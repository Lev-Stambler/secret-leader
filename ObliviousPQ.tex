\newcommand{\ct}{\texttt{ct}}
% https://eprint.iacr.org/2016/046.pdf
% and the functionality that it realizes is one that produces an encrypted random ordered list of the first k bla thingies
% We will start by describing the functionality of an MPC protocol for restricted computing environments which produces some random output list.

% TODO: ideal functionality
\subsection{Oblivious Priority Queue}
We will make extensive use of an oblivious priority queue which works over FHE.

\newcommand{\PQInsert}{\texttt{PQInsert}}
\newcommand{\PQFront}{\texttt{PQFront}}
\newcommand{\PQExtractFront}{\texttt{PQExtractFront}}
\newcommand{\PQ}{\texttt{PQ}}
\newcommand{\hybrid}[1]{\mathrm{\textbf{Hyb}}_{#1}}
% \newcommand{\simulator}{\mathcal{S}}

\begin{definition}[Oblivious Priority Queue]
	An oblivious priority queue is a priority queue which can be ``run'' by a party oblivious to the actual
	messages of the priority queue. More formally, an oblivious priority queue has functionality
	\begin{itemize}
		\item $\PQInsert(\ct, \PQ) = \PQ'$ which takes in a cipher text $\ct$ and current priority queue and outputs $\PQ'$ where 
		the ciphertext is properly inserted into the priority queue according to the underlying plaintext of $\ct$.
		\item $\PQFront(\PQ) = \ct$ which takes as input a priority queue and outputs the ciphertext $\ct$ who's plaintext is the minimum element in the queue.
		\item $\PQExtractFront(\PQ) = \ct, \PQ'$ which takes as input a priority queue and outputs the ciphertext with minimum plaintext as well as an updated queue without $\ct$.
	\end{itemize}
	We also require that for $N$ elements in $\PQ$, operation $Q \in \set{\PQInsert, \PQFront, \PQExtractFront}$,
	and input $\ct$ into $Q$ such that there is some simulator, $\simulatorPQ$ where
	% TODO: latex crypto package
	\begin{equation}
		\{\simulatorPQ(1^\lambda, N, Q, \PQ) \} \compInd	\{Q(\PQ, \ct) \}.
	\end{equation}
	In words, $\simulatorPQ$ can simulate a distribution of the output $\PQ$ independent of the input element.
	\LS{I am not quite sure about the above: definitions that I saw are for adaptive adversaries, but we are not adaptive.}
\end{definition}

\subsection{Multi SLE Algorithm}

For $n$ parties, we want to output a list of $k$ elements where each element is chosen randomly (without replacement) from a set of messages submitted by each party.
Let the function for each party denote $f_i$ and we will say that $f_i(\Enc(b_1), \Enc(b_2), ..., \Enc(b_n)) = y$ where $y \in \set{\bot, 1, ..., k}$
such that if $y=a$ then party $i$ is the leader for the $a$th round and if not $y = \bot$.

To build this functionality, we will use an oblivious streaming sampler. We can construct one as follows, closely following \cite{shi2020path} but modifying randomness sampling to use a PRF.
\begin{itemize}
	\item When an item, $\ct$, arrives as a TFHE cipher text, we will produce a random value, $\Enc(\alpha)$ by evaluating $\texttt{PRF}(k, \ct)$ in FHE.
	\item If the item is the $m$-th item and $m \leq k$, insert the item into the queue along with $\alpha$ as its label via $\PQInsert(\ct)$.
	\item If $m > k$, we will evaluate the PRF (in FHE) to get an encoded random coin which is 1 with probability $1/m$ and 0 otherwise.
	Then, run \cref{alg:ObIns} which will
	\begin{itemize}
		\item replace the smallest labeled item in the queue with the new item if the coin is 1.
		\item not replace the smallest item in the queue if the coin is 0.
	\end{itemize}
	\item At the end of the stream, we output all the items in the priority queue one by one using $\PQExtractFront$.
\end{itemize}
The simulation proof follows almost exactly as in \cite{shi2020path} but now we have to deal with the randomness from 
the PRF rather than public randomness. We can follow the proof directly but add one more hybrid where we replace the output of the PRF with a random oracle in the FHE cipher text space.
Then, we can use the PRF indistinguishably property to show that the hybrid is indistinguishable from the previous hybrid.\\

We can then implement multi secret leader election via the following algorithm:
\begin{itemize}
	\item Each party $i$ will submit a message $\Enc(b_i)$ to the oblivious streaming sampler for some secret $b_i$ which is a commitment to a secret known only by party $i$.
	\item After all parties submitted there messages, the oblivious streaming sampler will output a list of $k$ messages: $\Enc(b_{a_1}), \Enc(b_{a_2}), ..., \Enc(b_{a_k})$.
	\item Then, at least $t$ parties will submit decryption shares to decrypt $\Enc(b_{a_1}), ..., \Enc(b_{a_k})$.
	\item Each party will then build check if they one an election by seeing if their commitment is in the list of decrypted messages.
	A party can then prove that they won an election $a$ by showing that they know the opening to $b_a$.
\end{itemize}

\begin{algorithm}
	\caption{Oblivious Gated Insert}
	\label{alg:ObIns}
	\begin{algorithmic}
		\Require $\Enc(\text{coin}), \PQ, \ct_a$
		\Ensure $\PQ$'
		\State $\ct_b, \PQ' \gets \PQExtractFront(\PQ)$
		\State $\ct_{\text{new}} = \Enc(\text{coin}) \cdot \ct_a + (\Enc(\text{coin}) - 1) \cdot \ct_b$
		\State \Return $\PQInsert(ct_{\text{new}})$
	\end{algorithmic}
\end{algorithm}


\subsection{Correctness}
For correctness, we want to show that for each of the $k$ elections, a unique leader is chosen for each round
and that the probability of being selected leader is uniform out of the potential leaders.

More formally, let $E_i$ be the event in which party $i$ is elected for some election and
$E_{a, i}$ be the event which party $i$ is elected for the $a$-th election. We want 
\begin{align}
	&\Pr\left[E_i\right] = \frac{k}{n},\\
	&\Pr\left[E_{a, i} \mid E_i\right] = \frac{1}{k},\\
	&\Pr\left[E_{b, i} \mid E_{a, i}\right] = 0.
\end{align}
where $b \neq a, b \in [k]$.
\LS{I think that here we can prove the uniqueness and fairness property of multi-leader election}
\begin{theorem}[Correctness]
We have correctness...	
\begin{proof}
	By definition... should be pretty easy to show.
\end{proof}
\end{theorem}
\subsection{Simulation Security}
We will show security in a semi-honest model via simulation. We will show that each user's view of the protocol can be simulated by a polynomial-time algorithm that only has access to the user's input.
Note that view of publishing a message is the same as the view of as each user publishing a random LWE ciphertext assuming the TFHE is secure.
\\Then, upon evaluation, we need to show that the oblivious queue can be simulated.
Note that every step in the oblivious queue algorithm which does not involve $b_i$ involves manipulating
TFHE samples under a secret key unknown to the simulator. Thus, we can simulate all steps not involving $b_i$.
Every step in the oblivious queue algorithm which involves $b_i$ and some $b_j$, for $i \neq j$ requires 
a homomorphic evaluation where one of the inputs in encrypted under a secret key unknown to the simulator.
Thus, by the security of TFHE, the output of any of these steps is indistinguishable from a using a $b_j$ encoding a different message.

\subsubsection*{Proof of Simulation Security}
More formally, $\forall i \in [n]$, we can create a simulator, $\simulator_i$, which takes as input 
the user's secret value $b_i$ and publishes random TFHE encrypted message, $\Enc(b_j)$ for all $j \neq i$, for all other parties such that
\begin{equation}
	\label{eq:simulator}
	\{\simulator_i(1^\lambda, s_i, \comm_i, r_i, f_i(\Enc(b_1), ..., \Enc(b_n))\} \compInd \{\texttt{view}^\pi_1(b_i, y)\} 
\end{equation}
with non-negligible probability.

To achieve \cref{eq:simulator}, the simulator simply follows a simple procedure
\begin{enumerate}
	\item The simulator randomly samples $b_j$ for all $j \neq i$
	\item The simulator uses $\simulatorPQ$ to simulate the oblivious streaming sampler for all inputs $\Enc(b_j)$ for $j \neq i$.
	For input $\Enc(b_i)$, the simulator honestly runs the oblivious streaming sampler.
	\item After all $n$ parties have submitted their inputs, the simulator runs the oblivious streaming sampler to get the outputs $\Enc(b_{a_1}),... \Enc(b_{a_k})$.
	\item The simulator then pretends to ``decrypt'' $\Enc(b_{a_j})$ to get $b_{a_j}$ by setting the decryptions to a random value
	for $j \neq y$. If $y \neq \bot$, then the simulator sets the decryption of $b_{a_y} = b_i$.
	\item The simulator simply outputs the view of the protocol with the ``decryptions'' of $\Enc(b_j)$'s.
\end{enumerate}

We will now show that the above indeed is a polynomial time simulator with overwhelming probability.
\begin{lemma}
	The above algorithm satisfies the computational indistinguishably requirment of \cref{eq:simulator}.
	\begin{proof}
		First note that the above algorithm runs in polynomial time by inspection.
		
		We now show simulatability via hybrids.
		Note that we have that the encoded FHE elements (where party $i$ does not know the plaintext)
		are indistinguishable from encodings of random elements. We will now construct our hybrids:
		
		$\hybrid{1}$: Consider a hybrid where we replace the FHE encoded elements, $\Enc(b_j)$ for $j \neq i$ with random ciphertexts.
		By definition of IND-CPA security, this is indistinguishable from the original hybrid. We can then propogate the output of the oblivious priority queue
		to use these random elements.

		$\hybrid{2}$: Consider a hybrid where we replace the TFHE decoding of all winning elements which are not for the $y$-th round with a random value.
		As the $b_j$ for $j \neq i$ are random, this hybrid is indistinguishable from the previous hybrid.

		$\hybrid{3}$: Consider a hybrid where, if $y \neq \bot$, we replace $\Enc(b_{a_y})$ with some random cipher text. Note that because party 
		$i$ is not able to decrypt and arbitrary ciphertext, this hybrid is indistinguishable from the previous hybrid.

		$\hybrid{4}$: Consider a hybrid where we replace the the output of the priority queue with the output from simulating the priority queue with the random FHE encoded elements.
		Again, due to the IND-CPA property of FHE, this hybrid is also indistinguishable from the previous hybrid.
		
		Note that $\hybrid{4}$ is indistinguishable from the original hybrid and is the same as the output of the simulator.
	\end{proof}
\end{lemma}