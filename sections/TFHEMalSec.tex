\section{Malicious Adversary Secure Protocol}
\newcommand{\relEncr}{\proofRel_\texttt{Encr}}
\newcommand{\relProt}{\proofRel_\texttt{Prot}}
\newcommand{\relOpen}{\proofRel_\texttt{Open}}
\newcommand{\proofOpen}[1]{\pi_{#1}^{\texttt{Open}}}
\newcommand{\proofEncr}[1]{\pi_{#1}^{\texttt{Encr}}}
\newcommand{\proofProt}[1]{\pi_{#1}^{\texttt{Prot}}}
\newcommand{\inCTList}{\texttt{Inps}}
\newcommand{\decryptShares}{\texttt{Shares}}
Before describing the protocol, we need to define two relationships which will be used in
conjunction with two different NZIKs. A tuple $((\TFHEPk, \ct), \mu) \in \relEncr$ if
$\ct = \Enc(\mu)$ under TFHE public key $\TFHEPk$.
A tuple $((p_i, \inCTList, \Enc(s_\TFHE)), \TFHESk_i) \in \relProt$ if
% Hmmmm.... do we need verifiable encryption here? I think so...

% \begin{figure}
% 	\label{fig:protMal}
\centering
% \fbox{
\begin{mdframed}
	\textbf{The MSLE Protocol} $\pi_{\text{MSLE}}$:
	Initialize $\nBids = 0$. Fix some $k \in \N$ to denote the number of rounds.
	Initialize an empty set of tickets, $R$, an empty lookup of sets of parties in each election, $S$,
	an empty lookup of reservoir sampling data structures $\resSampling$,
	and a set of finished elections $\FinElect$. Initialize an empty lookup of election results, $E$.

	\begin{itemize}
		% \item $\init$
		%       \begin{itemize}
		% 	      \item Set $n = 0$
		% 	      \item Sample a random $\TFHE$ secret key, public key pair $\TFHESk, \TFHEPk$ and publish $\TFHEPk$.
		% 	      \item Sample some secret $s_{\texttt{TFHE}}$ and publish $\Enc(s_{\texttt{TFHE}})$.
		% 	            This will be the key to the PRF
		%       \end{itemize}

		% \item $\register$ from party $P_i$
		%       \begin{itemize}
		% 	      %  TODO: not within definition of TFHE, have to change up definition :((
		% 	      %  TODO: formalize secure channel?
		% 	      \item If party $P_i$ does not already have a TFHE share, create a TFHE share, $\TFHESk_i$.
		% 	            Then, sample randomness $r$ and publish a commitment, $c_{\TFHESk_i} = \comm(\TFHESk_i, r)$. Send $\TFHESk_i$ and $r$
		% 	            to party $P_i$ over a secure channel.

		% 	      \item $R \gets R \cup \{(i, \nBids)\}$, set $\nBids \gets \nBids + 1$.
		%       \end{itemize}
		      % \item Add $\Enc(c_i)$ to $C$.
		      % \item $n \gets n + 1$

		\item $(\registerElect, \eid)$ called by party $P_i$
		      Every honest party does the following:
		      \begin{itemize}
						\item Sample $s_i, r_i \randomGet \F_q, c_i \gets \comm(s_i, r_i), \ct_i \gets \Enc(c_i)$.
						Store $c_i$ and $\ct_i$ in a lookup table, $C$.
						\item Generate $\proofEncr{i} \gets \NZIKProve(\crs, \ct_i, c_i)$ to show $(c_i, \ct_i) \in \relEncr$ where $(w, x) \in \relEncr$ if $x = \Enc(w)$.
						\item Broadcast $(\registerElectAdd, \eid, \ct_i, \proofEncr{i})$ and run \\$(\registerElectAdd, \eid, \ct_i, \proofEncr{i})$.
			      % \item TODO: NEED TO CHECK VALID ENCRYPTION
			      % \item If $(i, w) \in S_\eid$ or $(i, w) \notin R$, broadcast $\bot$ and do nothing.
			      % \item Otherwise, if $\resSampling_\eid \notin \resSampling$, $\resSampling_\eid \gets \ResevInit(k)$
			      % \item The CRS will be used to run a PRF to get $\Enc(e_j), \Enc(\texttt{coin}_j) =$ \\$\Enc(\texttt{PRF}(c_j, s_\texttt{TFHE}))$
			      % \item $\Enc(c_j)$ will be fed to the encrypted streaming sampler along with randomness via
			      %       $\ResevInsertN{\eid}(\Enc(c_j), \Enc(e_j), \Enc(\texttt{coin}_j))$.
		      \end{itemize}
		
		\item $(\registerElectAdd, \eid, \ct_j, \proofEncr{j})$ called by party $P_j$.
					\begin{itemize}
			      \item If $\eid \in \FinElect$ or $\NZIKVerify(\NZIKCRS, \ct_j, \proofEncr{j}) \neq 1$, return and do nothing.
						\item If $b_\eid$ is not stored, $b_\eid \gets 0$
						\item $b_\eid \gets b_\eid + 1$
						% TODO: when call?
						\item If $\resSampling_\eid$ is not stored, $\resSampling_\eid \gets \ResevInit(k)$
						\item $\Enc(e_i), \Enc(\coin_i) \gets \TFHEEval(C_{PRF}, \crs, \ct_j)$
			      \item If $\inCTList$ is not stored, $\inCTList \gets [\;]$. 
						\item $\inCTList[j] \gets \Enc(c_j)$
					  \item Call $\resSampling_\eid \gets \TFHEEval(C, \resSampling_\eid, \ct_j, \Enc(e_i), \Enc(\texttt{coin}_i))$ where
					        $C$ is the reservoir sampling circuit for $\ResevInsert$. % TODO: is this right defn of resev insert?
					\end{itemize}

		\item $(\elect, \eid)$. called by party $P_i$. (TODO: add conditions)
		      \begin{itemize}
			      % \item If $\eid \in \FinElect$, then broadcast $\bot$ and do nothing.
						\item For $\ell \in [k]$, $\ct_\ell \gets \TFHEEval(C_\ResevOutput, \resSampling_\eid)$
					  \item Set $p_i \gets \TFHEPartDec(\ct_1, ..., \ct_k, \TFHESk_i)$
						\item Generate $\proofProt{i} \gets \NZIKProve(\crs, (p_i, \inCTList), \TFHESk_i) $
						for the relationship $\relProt$ where $(p_i, \inCTList) \in \relProt$ if... TODO:
						\item Broadcast $(\electDone, \eid, p_i, \proofProt{i})$ and call $(\electDone, \eid, p_i, \proofProt{i})$
					\end{itemize}
		\item $(\electDone, \eid, p_j, \proofProt{j})$ called by party $P_j$
					\begin{itemize}
						\item If $\NZIKVerify(\crs, (\inCTList, p_j), \proofProt{j}) \neq 1$, do nothing and return
						\item If $\electDone_j$ is stored, do nothing and return.
						\item Store $\electDone_j$
					  \item Retrieve $\partDecrySet^\eid$ if it is stored, otherwise set $\partDecrySet^\eid \gets \emptyset$.
					  \item Set $\partDecrySet^\eid \gets \partDecrySet^\eid \bigcup \{p_j\}$
					  \item If $|\partDecrySet^\eid| \geq t$, set and store $c_{a_1}, ..., c_{a_\ell} \gets \TFHEFinDec(\partDecrySet^\eid)$
					  \item Store $\partDecrySet^\eid$.
			      \item Send $(\texttt{prove}, sid, (p_i, \inCTList, \Enc(s_\TFHE)), \TFHESk_i)$ to $\NZIKIdeal$.
		      \end{itemize}
		      \pagebreak
		\item $\texttt{reveal}(\eid, \ell, \Enc(c'_i))$ from $P_i$
		      \begin{itemize}
			      %  TODO: formal with commitments...
			      \item $P_i$ submits a proof that they know the opening to $c_{a_\ell}$.
			            If this proof verifies, send out $(\texttt{result}, \eid, \ell, i)$ to all parties.
			            Otherwise, send out $(\texttt{rejected}, \eid, \ell, i)$ to all parties.
		      \end{itemize}
	\end{itemize}
\end{mdframed}
% \end{minipage}
% }
% 	\caption{Description of the MSLE protocol, secure against static malicious adversaries}
% 	\label{fig:protocolStatisMalMSLE}
% \end{figure}

\subsection{Static Malicious Adversary Security}
Consider parties $P_i$ to be corrupted where $i \in C$ and $C \subseteq [n]$ such that.
$|C| < t$.

\begin{lemma}[$\registerElect$ is simulation secure]
	Idea: if $\bot$, just do same
	If: $j \notin C$, just do same
	If: $j \in C$, then use the NIZK to extract out the commitment from the FHE ciphertext or bot
	Then: S follows protocol
\end{lemma}

\begin{lemma}[$\elect$ is simulation secure]
	Idea: if $\bot$, just do same
	If: $j \notin C$, just do same
	Output list from streaming sampler using TFHE sim, set to correct sequence (irrespective of advers)
	Ask advers to generate decryption shares for each corrupted party and broadcast proof
	For non-corrupted parties, use TFHE sim to generate decryption shares, simulate proofs
	For corrupted parties, ask advers to generate decryption shares and proofs
	If proofs check, use them, if not discard the decryption shares
\end{lemma}

\begin{lemma}[$\reveal$ is simulation secure]
	% TODO: change decommitment to NZIK? Or find sim sec comm scheme?
	% If we get that it checks out
	If an honest party sends the req, we simulate a commitment opening according to the stored commitment (that we know)
	If corrupted party sends req, pretending to know opening to not their commitment, we say $\bot$ (this is likely as we have the hiding property)
	If corrupted party sends req correctly, we use the de-commitment that they send

\end{lemma}
