\section{Encrypted Priority Queue}



\section{Data Independent Streaming Sampler}
\label{sec:streaming_sampler}
Our construction makes heavy use of a data independent streaming sampler which we will define below.
The streaming sampler relies on a data independent priority queue and in turn makes use of \LS{Cite Shi} protocol for an oblivious priority queue.

\subsection*{
	Encrypted Data Independent Queue
}
In this work, we will data independent queues as studied in \cite{toft2011secure, mitchell2014data, mazloom2023efficient}.
Data independent data structures are unique as their control flow and memory access do not depend on input data (\cite{mitchell2014data}).

We will use the data independent queue within public key threshold FHE such that we now have an encrypted priority queue, $\EncPQ$
with functionality
\begin{itemize}
	\item $\EncPQInsert$: Inserts a tag and value, $\Enc(p, x)$ into $\PQ$ according to the tag's priority.
	\item $\EncPQExtract$: Removes and returns the $\Enc(p, y)$ with highest tag priority.
	\item $\EncPQFront$: Returns the $\Enc(p, y)$ with highest tag priority without removing the element.
\end{itemize}

We further require that for $\Enc(p, y) \gets \EncPQExtract$ (and $\EncPQFront$) and $x_1, ..., x_n$ drawn from a random distribution,
\begin{align}
	\label{eq:gameEncPQ}
	\Bigl\lvert \Pr[\advers\left(x_1, \dots x_n, \Enc(p_1, x_1), ..., \Enc(p_n, x_n), \Enc(p, y), i\right) = 1] \\
	- \Pr[\advers\left(x_1, \dots x_n, \Enc(p_1, x_1), ..., \Enc(p_n, x_n), \Enc(p, y), j\right) = 1]
	\Bigr\rvert \leq \negl(\lambda) \notag
\end{align}
where $(p_i, x_i)$ and $(p_j, x_j)$ are the $i$th and $j$th submitted message to the priority queue for $i \neq j$.

\begin{lemma}
	Assuming the indistinguishablity of TFHE cipher texts, $\Enc(m_1) \dots \Enc(m_n)$
	where $m_i \neq m_j$ for $i \neq j$ and are known to the adversary, then \cref{eq:gameEncPQ} holds.

	\begin{proof}
		Assume towards contradiction that there exists an adversary $\advers$ which can distinguish between a random
		message to the priority queue and the output of $\EncPQExtract$ (resp. $\EncPQFront$).
		Then, we can construct an adversary $\advers'$ which can distinguish TFHE cipher text, $\ct_1 = \Enc(m_1), \dots, \ct_n = \Enc(m_n)$, as follows:
		\begin{enumerate}
			\item $\advers'$ simulates an encrypted priority queue, $\EncPQ$.
			\item $\advers'$ inserts $x_i = \Enc(\ct_i, r_i)$ into $\EncPQ$ for all $i \in [n]$.
			\item The adversary then calls $\Enc(p, y) \gets \EncPQExtract$ (resp. $\EncPQFront$) and uses $\advers$ to distinguish between $x_1, \dots x_n$.
			\item Output $m_i$ where $\Enc(p, y) = x_i$.
		\end{enumerate}
		Clearly, if the adversary can distinguish the priority queue outcomes, then using $\advers$, he can with non-negligible probability
		distinguish encryptions of $m_1, \dots, m_n$ as the relative ordering of the encryptions is preserved and thus the priority queue repeatably gives the same output.
	\end{proof}
\end{lemma}


\subsection*{Encrypted Streaming Sampler}
A streaming sampler should output a random subset of size $k$ from a stream of $n$ elements with a random ordering.
We can build a streaming sampler from a priority queue as follows:


\begin{algorithm}
	\caption{Encrypted Gated Insert}
	\label{alg:EncIns}
	\begin{algorithmic}
		\Require $\Enc(\text{coin}), \EncPQ, \ct_a$
		\Ensure $\EncPQ$'
		\State $\ct_b \gets \EncPQExtract(\PQ)$
		\State $\ct_{\text{new}} = \Enc(\text{coin}) \cdot \ct_a + (\Enc(\text{coin}) - 1) \cdot \ct_b$
		\State $\EncPQInsert(ct_{\text{new}})$
		\State \Return $\EncPQ$
	\end{algorithmic}
\end{algorithm}

We will use a slightly modified version of a streaming sampler where randomness is submitted alongside the message.
We can then implement the algorithm as follows
\begin{itemize}
	\item $\texttt{Insert}(\ct_i, \Enc(e_i))$ where $\ct_i$ is a cipher text and $\Enc(e_i)$ is encrypted randomness
	      \begin{itemize}
		      \item If the item is the $m$-th item and $m \leq k$, insert the item into the queue along with $e_i$ as its tag via $\EncPQInsert(\Enc(e_i), \ct)$.
		      \item If $m > k$, we will evaluate the PRF (in FHE) to get an encoded random coin which is 1 with probability $1/m$ and 0 otherwise.
		            Then, run \cref{alg:EncIns} which will
		            \begin{itemize}
			            \item replace the smallest labeled item in the queue with the new item if the coin is 1.
			            \item not replace the smallest item in the queue if the coin is 0.
		            \end{itemize}
	      \end{itemize}

	\item At the end of the stream, we output all the items in the priority queue, $\ct_{a_1} ... \ct_{a_k}$ in order
	      by repeatably calling $\EncPQExtract$.

\end{itemize}

We can formalize soundness as a game as follows, for all $\ell \in [k], i, j \in [n]$,
\begin{align}
	\Bigl\lvert
	\Pr[\advers(\ct_1, \dots, \ct_n, e_1, \dots, e_n, \ct_{a_1}, \dots, \ct_{a_k}, \ell, i) = 1] \\
	- \Pr[\advers(\ct_1, \dots, e_1, \dots, \ct_{a_1}, \dots, \ell, j) = 1]
	\Bigr\rvert \leq \negl(\lambda) \notag
\end{align}
In words, the adversary should not be able to guess with any advantage which of the $\ell$-th
outputted cipher text is associated with which inputted cipher text.

\begin{lemma}
	Our construction of a streaming sampler is sound.
	\begin{proof}
		By construction of the streaming sampler, we have that $\Pr[a_\ell = i] = 1/n$ and thus $\Pr[\Dec(\ct_{a_\ell}) = \Dec(\ct_i)] = 1/n$.
		% Now, we need to show that the adversary cannot distinguish between the output of the streaming sampler and a random subset of size $k$.
		...


		% TODO: this should not be too bad
		% Hmmm.... all randomness (add coins) could be input into the protocol, thus its repeatable?
	\end{proof}
\end{lemma}