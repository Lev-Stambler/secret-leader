\section{Data Independent Priority Queue}
Here we introduce a data independent priority queue. We will assume that there are no items in the queue
with equal priority. In practice, we can break ties by adding a unique identifier to each item.

\newcommand{\DIPQhead}{\texttt{Head}}
\newcommand{\DIPQstash}{\texttt{Stash}}
\newcommand{\DIPQcounter}{{\texttt{count}}}
\newcommand{\DIPQpartition}{\mathcal{P}}
\newcommand{\MergeFill}{\texttt{MergeFill}}
\newcommand{\Fill}{\texttt{Fill}}
\newcommand{\sqrtn}{{\sqrt{n}}}
\newcommand{\good}{\mathcal{G}}
\newcommand{\bad}{\mathcal{B}}

\begin{algorithm}
	\caption{$\MergeFill$}
	\begin{algorithmic}[1]
			\State \textbf{Input:} $\DIPQhead, \DIPQpartition_\DIPQcounter$
			\State $e_1, ..., e_\gamma = \texttt{sort}(\DIPQhead, \DIPQpartition_\DIPQcounter)$ where $\gamma = |\DIPQhead| + |\DIPQpartition_\DIPQcounter|$ \Comment Via a data-independent sort
			\State $\DIPQhead = \{e_1, ..., e_{\min\left(\sqrtn, \gamma\right)}\}$
			\State $\DIPQpartition_\DIPQcounter = \{e_{\sqrtn + 1}, ..., e_\gamma\}$
			\State \Return $\DIPQhead', \DIPQpartition_\DIPQcounter$
	\end{algorithmic}
\end{algorithm}

\begin{algorithm}
	\caption{$\Fill$}
	\begin{algorithmic}[1]
			\State \textbf{Input:} $\DIPQpartition_\DIPQcounter, \DIPQstash$
			\State $e_1, ..., e_\gamma = \DIPQpartition_\DIPQcounter \bigcup \DIPQstash$ where $\gamma = |\DIPQpartition_\DIPQcounter| + |\DIPQstash|$
			\State $\DIPQpartition_\DIPQcounter = \{e_1, ..., e_{\min(\sqrtn, \gamma)}\}$
			\State $\DIPQstash = \{e_{\sqrtn + 1}, ..., e_\gamma\}$
			\State \Return $\DIPQpartition_\DIPQcounter', \DIPQstash$
	\end{algorithmic}
\end{algorithm}


	
\begin{algorithm}
	\caption{Data Independent Priority Queue ($\DIPQ$)}
	\label{alg:DIQueue}
	\begin{algorithmic}[1]
		\Function{$\DIPQInit$}{}
		\State $\DIPQcounter \gets 0$
		\State $\DIPQhead \gets []$
		\State $\DIPQpartition_0, \cdots, \DIPQpartition_{\sqrtn-1} \gets \emptyset$
		\State $\DIPQstash \gets \emptyset$
		\EndFunction
		\Function{$\DIPQInsert$}{$p, v$}
		\If{$|\DIPQhead| < \sqrtn$}
			\State Append $(p, v)$ to $\DIPQhead$
		\Else
			\State $i, (p', v') = \texttt{GetLargest}(\DIPQhead)$ where $i$ is the index of $(p', v')$ in $\DIPQhead$
			\label{algline:HeadExtract}
			\State $I = 1$ if $p' < p$ else $0$
			\State $\DIPQstash[i] = I \cdot (p, v) + (1 - I) \cdot (p', v')$
			\State $\DIPQhead[i] = I \cdot (p', v') + (1 - I) \cdot (p, v)$
		\EndIf
		\State Call $\texttt{Order}()$ \label{algline:PQOrderIns}
		\EndFunction
		\Function{$\DIPQExtract$}{}
		 	% \State $i, (p, v) = \texttt{GetSmallest}(\DIPQhead)$ where $i$ is the index of $(p, v)$ in $\DIPQhead$
		 	\State $j, (p, v) = \texttt{GetSmallest}(\DIPQstash)$ where $j$ is the index of $(p', v')$ in $\DIPQstash$
		 	\State $k, (p', v') = \texttt{GetLargest}(\DIPQhead)$ where $k$ is the index of $(a, b)$ in $\DIPQhead$
			\State $I$ = 1 if $p' < p$ else 0
			\State $\DIPQstash[j] = (1 - I) \cdot (p', v') + I \cdot (p, v)$ \Comment Conditionally swap $(p, v)$ and $(p', v')$
			\State $\DIPQhead[k] = I \cdot (p', v') + (1 - I) \cdot (p, v)$ \Comment Conditionally swap $(p', v')$ and $(p, v)$
			\State $i, (p_r, v_r) = \texttt{GetSmallest}(\DIPQhead)$
			\State Call $\texttt{Order}()$
			\State \Return $(p_r, v_r)$
		\EndFunction
		\Function{$\DIPQOrder$}{}
			\State $\DIPQpartition_{\DIPQcounter}, \DIPQstash = \Fill(\DIPQpartition_\DIPQcounter, \DIPQstash)$
			\State $\DIPQhead, \DIPQpartition_\DIPQcounter = \MergeFill(\DIPQhead, \DIPQpartition_\DIPQcounter)$
			\State $\DIPQcounter \gets \DIPQcounter + 1 \mod \sqrtn$
		\EndFunction
	\end{algorithmic}
\end{algorithm}
% TODO: stash is always empty?? Maybe not...


\subsection{Invariants}
\label{subsec:invariants}
Let $i, j \in \N$ be the total number of insertions and extractions respectively. Note that if $i - j \leq \sqrtn$
and in the prior $i + j$ operations, the size of $\DIPQ$ never exceeded $\sqrtn$, then the priority queue is trivially correct because all elements remain in the head which is itself a priority queue.
Thus, we will assume that at some point in the sequence of insertions and extractions, we had that $|\DIPQ| > \sqrtn$.

Before specifying our invariants, we will add some notation. Let $\good$ ($\good$ for ``good'') be the largest set of smallest priorities in the head. More formally,
$\good$ is the largest subset of $\DIPQhead$ such that $\forall e \in \DIPQ, e \notin \good$, $a < e, \forall a \in \good$. Note that $|\good| \leq \sqrtn$.
Further, let $g$ be the smallest priority in $\DIPQ$ which is not in the head, $\DIPQhead$.
Note that $g \notin \good$.
% TODO: make explicit that g \notin G
% Confusing to call g the ``good'' element
% g is the smallest elem in the PQ not in G
% TODO: remove the forall e

We will show that the following invariants hold:
\begin{itemize}
	\item The $g$ is not in the prior $\sqrtn - |\good|$ iterated over partitions. Formally, $g \notin \bigcup_{q \in [\DIPQcounter - \sqrtn + |\good|, \DIPQcounter)} \DIPQpartition_q$
\end{itemize}

\begin{proof}
We proceed by joint induction on $i, j$ where $i - j \leq n$. 
We will first prove the base case where $i - j = \sqrtn + 1$ and $i - j \leq \sqrtn$
% TODO: define i_(t), j_(t): number of insert and extract at time t
% Say it does not come up before
for all prior operations.
Note that the operation has to be an insertion in-order for $i - j$ to increase. Thus, the head contains $\sqrtn$ elements, all of which are smaller than the element evicted by $\texttt{PQ.ExtractLargest}$ (line~\ref{algline:HeadExtract}) in the insertion operation.
We thus have that $|\good| = \sqrtn$ after the insertion. 
We can then see that the invariant holds trivially as $\DIPQcounter - \sqrtn + |\good| = \DIPQcounter - \sqrtn + \sqrtn = \DIPQcounter$ and thus
$\bigcup_{q \in [\DIPQcounter - \sqrtn + -\sqrtn, \DIPQcounter)} \DIPQpartition_q = \emptyset$.\\
% TODO: put here that we need to show that g is not in p_count...
% This helps with the motivation, etc.

Now, to prove the inductive case we need to show that after an operation, the new good set, $\good'$ has size
at least $|\good| - 1$ and that $g \notin \DIPQpartition_{\DIPQcounter + 1}$. Then, as we have the inductive hypothesis
that $g \notin \bigcup_{q \in [\DIPQcounter - \sqrtn + |\good|, \DIPQcounter)} \DIPQpartition_q$, 
we can show that $g \notin \bigcup_{q \in [\DIPQcounter + 1 - \sqrtn + |\good| - 1, \DIPQcounter + 1)} \DIPQpartition_q$.

We will proceed by assuming that the invariant hold for insertion count $i$ and extraction count $j$. We will show that they hold for $i + 1, j$ and $i, j + 1$.
\paragraph*{Case 1: $i + 1, j$}
\label{proof:case1}
Let $(p, v)$ be the inserted element. We will now show that the new good set, $\good'$
has size at least $|\good|$.
If $p < a$ for some $a \in \good$, then $p \in \good'$. As at most one element is evicted from $\good$,
we have that $|\good'| \geq |\good| - 1 + 1$. If $p > a$ for all $a \in \good$, then the $\good'$ does not lose any elements from $\good$ and so
$|\good'| \geq |\good|$.

Now, note that if $|\good'| = \sqrtn$, then the invariants hold trivially as $\bigcup_{j \in [\DIPQcounter + 1 - \sqrtn + \sqrtn, \DIPQcounter + 1)} \DIPQpartition_j = \emptyset$.  % Can we remove the G' \geq G
So, we assume that $|\good'| < \sqrtn$. Note that because we call $\DIPQOrder$ at the end of insertion (line~\ref{algline:PQOrderIns})
and $\sqrtn > |\good'| \geq |\good|$, $\DIPQhead$ either had an empty slot or an element $\beta$ such that $\beta > g$ prior to insertion.
If $p < g$, then $g' \gets p$ (where $g'$ is the updated $g$) and $(p, v)$ are moved into the head.
Otherwise, after calling $\DIPQOrder$, we have that $g$ is moved into $\DIPQhead'$ if $g \in \DIPQpartition_\DIPQcounter$ by the functionality of $\MergeFill$.
So, we can then see that $g' \notin \DIPQpartition'_\DIPQcounter$.
Finally, as $g' \leq g$ and no element smaller than or equal to $g$ is in $\bigcup_{q \in [\DIPQcounter - \sqrtn + |\good|, \DIPQcounter)} \DIPQpartition_q$ by the inductive hypothesis,
$g' \notin \bigcup_{q \in [\DIPQcounter + 1 - \sqrtn + |\good|, \DIPQcounter + 1)} \DIPQpartition_q$.\\

\paragraph*{Case 2: $i, j + 1$} Note that on an extraction from $\DIPQhead$, $|\good|$ may decrease by at most 1.
We can then see that the new good set $\good'$ has size at least $|\good| - 1$.
Again, for the case where the new good set, $\good'$, has size $\sqrtn$, the invariants hold trivially.
So, we assume that $|\good'| < \sqrtn$.

We must now show that $g$ is not in the updated current partition, $\DIPQpartition'_\DIPQcounter$, after $\DIPQOrder$ is called.
We can show this because we call $\MergeFill$ on the head and the current partition, $\DIPQpartition_\DIPQcounter$. I.e.\ if $\MergeFill$ introduced element $g$
into the head, $\good \notin \DIPQpartition'_\DIPQcounter$ and thus the invariants hold. Otherwise, if $\MergeFill$ did not introduce $g$ into the head, then $g \notin \DIPQpartition_\DIPQcounter$ and so $g \notin \DIPQpartition'_\DIPQcounter$
as elements in $\DIPQpartition_\DIPQcounter$ can only increase after $\MergeFill$.
% then $|\good'| = \sqrtn$ and thus the invariants hold trivially. Otherwise, we have that $g \notin \DIPQpartition_\DIPQcounter$ and so $g \notin \DIPQpartition'_\DIPQcounter$.
% we have that $|\good'| = \sqrtn - 1$ and thus there exists at least one element in the head, $\beta$, such that $\beta > g$.
% As $\MergeFill$ guarantees that $\forall e \in \DIPQpartition'_\DIPQcounter, e > \beta$, we have that $g$ is not in $\DIPQpartition'_\DIPQcounter$ and thus the invariants hold.
% If $|\good| < \sqrtn$, we either have that $|\DIPQhead| = \sqrtn$ or has empty slots as in \textbf{case 1}. In either case, after calling merge sort,
% we have that forall $e \in \DIPQpartition'_\DIPQcounter$, $e > g$ as if $g  \in \DIPQpartition_\DIPQcounter$, then $g$ is moved into the head.
By the inductive hypothesis and the above, we can see that $g \notin \bigcup_{q \in [\DIPQcounter + 1 - \sqrtn + |\good| - 1, \DIPQcounter + 1)} \DIPQpartition_q$ via an identical line of reasoning as in \textbf{case 1}.
Thus, we then have that invariant holds.\\


By induction, we have that the invariant holds for all $i, j$ when the number of elements in the priority queue passed $\sqrtn$ at some point in the sequence of operations.
\end{proof}

\subsection{Correctness Proof}
\subsubsection*{$\DIPQExtract$ Correctness}
Assuming that the invariants hold in \cref{subsec:invariants}, we will show that the algorithm is correct.
Note that if the total number of elements in the queue never exceeded $\sqrtn$ elements, correctness holds as elements are never moved out of the head which is itself a priority queue.
Otherwise, we will show that $\DIPQpartition_\DIPQcounter$ is always ``close enough'' to the next ``good'' element.
Whenever $\DIPQExtract$ is called, we have that $|\good|$ may decrease by 1. If $|\good|$ does decrease by 1,
we still have that the partition containing the next good element, $g$, will be reached in at most $|\good| - 1$ $\DIPQ$ operations.
So, by the call of $\MergeFill$ in $\DIPQ.\texttt{Order}()$, we will have that $g$ is in the head or the stash
after $|\good| - 1$ calls. 
We can then guarantee that the smallest element in the head and stash are at least as small as $g$.
If no insertions were called with element $e$ where $e < g$,
then after $|\good - 1|$ operations, our new $\good$ set, $\good'$, will have size of at least $|\good| - 1 - (|\good| - 1) + 1 = 1$
as $g$ will be moved into the head or $g$ will be in the stash. Thus, calling $\DIPQExtract$ will return the smallest element as we always have that $|\good| > 0$ or, if $|\good| = 0$, $g$, now the smallest element, is in the stash.
Otherwise, if $e$, such that $e < g$, is inserted within the $|\good| - 1$ operations, then $e$ will be in the head and be a part of the new $\good$ set, $\good'$.
Similarly, after $|\good| - 1$ operations, we will have that $|\good'| \geq 1$ and thus the smallest element will be returned by $\DIPQExtract$.

\subsubsection*{Size of Stash}
We will show that $|\DIPQstash| \leq \sqrtn$. As, $|\DIPQstash|$ and $\DIPQpartition_\alpha$, for all $\alpha \in [\sqrtn]$, does not grow if $|\DIPQhead| < \sqrtn$,
we will assume that $|\DIPQhead| = \sqrtn$. 
Note that we are guaranteed that the total number of elements in $\DIPQ$ is at most $n$ and that the capacity of each partition is $\DIPQpartition_\alpha = \sqrtn$.
So, the total capacity of all partitions is $n$ and, we have that there is always at least $\sqrtn$ empty slots in the partitions.
Thus, we have that no element which is added to the stash remains
in the stash for more than $\sqrtn$ calls to $\DIPQ.\texttt{Order}()$ as each call attempts to place an element from the stash into the current partition.
We can also note that the stash can only grow by at most 1 element per call to $\DIPQInsert$.
Thus, assume towards contradiction that $|\DIPQstash| > \sqrtn$. Then, we have that an element remained in the stash
for more than $\sqrtn$ calls to $\DIPQ.\texttt{Order}()$ which is a contradiction.
We can then see that $|\DIPQstash| \leq \sqrtn$.

\subsubsection*{Data Independence}
We will now show that the priority queue is data independent.
We can do this by simply noting that every operation is data independent.

\subsection{Time Complexities}

\subsection{Complexity}
% Asymptotic result, AKS sort. In practice, Bitonic (batcher) sort
% TODO: be careful
% TODO: say f(n) sort time
% TODO: careful that Fill and MergeFill and data independent
% TODO: say why Fill, MergeFill (Order), is data indep. Add subsection
Assume that we have a data independent sorting algorithm which takes $O(f(n))$ time where $f(n)$ is some function of $n$.
Now, we can show that the time complexity of $\DIPQInsert$ and $\DIPQExtract$ is $O(\sqrtn + f(n))$.
Note that $\DIPQOrder$ makes a call to $\MergeFill$ which does a sort and that the rest of the operations take at most $O(\sqrtn)$ time.
So, $\DIPQOrder$ takes $O(\sqrtn + f(n))$ time.
We can also see that in $\DIPQInsert$ and $\DIPQExtract$ all non-$\DIPQOrder$ operations take at most $O(\sqrtn)$ time.
So, $\DIPQInsert$ and $\DIPQExtract$ take $O(\sqrtn + f(n))$ time.
