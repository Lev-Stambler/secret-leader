\section{Preliminaries}
\subsection{Threshold FHE}
\cite{jain2017threshold,boneh2018threshold} defines threshold FHE encryption. For the sake of completeness, we will define it here.

\begin{definition}[TFHE \cite{jain2017threshold}]
	Let $P = \set{P_1, ..., P_N}$	 be a set of $N$ parties and $\mathbb{S}$ be a class of
	access structures on $P$. A TFHE scheme for $\mathbb{S}$ is a tuple of PPT algorithms
	$$
		(\TFHESetup, \TFHEEncrypt, \TFHEEval, \TFHEPartDec, \TFHEFinDec)
	$$ such that the following specifications are met

	\begin{itemize}
		\item $(\TFHEPk, \TFHESk_1, ..., \TFHESk_N) \gets \TFHESetup(1^\lambda, 1^s, \mathbb{A})$:
		      Takes as input a security parameter $\lambda$, a depth bound on the circuit, and a structure $\mathbb{A} \in \mathbb{S}$.
		      Outputs a public key $\TFHEPk$ and a secret key $\TFHESk_i$ for each party $P_i$.
		\item $\ct \gets \TFHEEncrypt(\TFHEPk, \mu)$: Takes as input a public key and a message $\mu \in \set{0, 1}$ and outputs a ciphertext $\ct$.
		\item $\hat{\ct} \gets \TFHEEval(C, \ct_1, ..., \ct_k)$: Takes as input a circuit $C$ of depth at most $d$ and $k$ ciphertexts $\ct_1, ..., \ct_k$.
		      Outputs a ciphertext $\hat{\ct} = C(\ct_1, ..., \ct_k)$.
		\item $p_i \gets \TFHEPartDec(\ct, \TFHESk_i)$: Takes as input a ciphertext $\ct$ and a secret key $\TFHESk_i$ and outputs a partial decryption $p_i$.
		\item $\hat\mu \gets \TFHEFinDec(B)$: Takes as input a set $B = \set{p_i}_{i \in S}$ for some $S \subseteq [N]$ and deterministically
		      outputs a message $\hat\mu \in \set{0, 1, \bot}$.
	\end{itemize}
	% Let $\mathcal{F}$ be a FHE scheme with key generation algorithm 
\end{definition}

Further, we remember the definitions of evaluation correctness, semantic security, and simulation security as outlined in,
\cite{boneh2020single}.

\begin{definition}[Evaluation Correctness \cite{jain2017threshold}].
	We have that $\TFHE$ scheme is correct if
	for all $\lambda$, depth bounds $d$, access structure $\mathbb{A}$, circuit $C : \set{0, 1}^k \rightarrow \set{0, 1}$
	of depth at most $d$, $S \in \mathbb{A}$, and $\mu_i \in \set{0, 1}$, we have the following.
	For $(\TFHEPk, \TFHESk_1, ..., \TFHESk_N) \gets \TFHESetup(1^\lambda, 1^d, \mathbb{A})$,
	$\ct_i \gets \TFHEEncrypt(\TFHEPk, \mu_i)$ for $i \in [k]$, $\hat{\ct} \gets \TFHEEval(\TFHEPk, C, \ct_1, ..., \ct_k)$,
	\begin{equation*}
		\Pr\left[\TFHEFinDec(\TFHEPk,
			\set{\TFHEPartDec(\TFHEPk, \hat{\ct}, \TFHESk_i)}_{i \in S}) = C(\mu_1, ..., \mu_k)\right] = 1 - \negl(\lambda).
	\end{equation*}
\end{definition}

\begin{definition}[Semantic Security \cite{jain2017threshold}]
	\label{def:semanticSecurityTFHE}
	We have that a $\TFHE$ scheme satisfies semantic security for for all $\lambda$, and depth bound $d$ if the following holds. There is a stateful PPT algorithm
	$\mathcal{S} = (\mathcal{S}_1, \mathcal{S}_2)$ such that for any PPT adversary $\mathcal{A}$,
	the following experiment outputs 1 with negligible probability in $\lambda$:
	\begin{enumerate}
		\item On input $1^\lambda$ and depth $1^d$, the adversary outputs $\mathbb{A} \in \mathbb{S}$
		\item The challenger runs $(\TFHEPk, \TFHESk_1, ..., \TFHESk_N) \gets \TFHESetup(1^\lambda, 1^d, \mathbb{A})$ and provides $\TFHEPk$ to $\advers$.
		\item $\advers$ outputs a set $S \subseteq \set{P_1, ..., P_N}$ such that $S \notin \mathbb{A}$.
		\item The challenger provides $\set{\TFHESk_i}_{i \in S}$ and $\TFHEEncrypt(\TFHEPk, \mu)$ to $\advers$ where $\mu \overset{\$}{\gets} \set{0, 1}$.
		\item $\advers$ outputs a guess $\mu'$. The experiment outputs 1 if $\mu' = \mu$.
	\end{enumerate}
\end{definition}

\begin{definition}[Simulation Security \cite{jain2017threshold}]
	We say that a $\TFHE$ scheme is simulation secure if for all $\lambda$, depth bound $d$, and access structure $\mathbb{A}$
	if there exists a stateful PPT simulator, $\mathcal{S}$, such that for any PPT adversary $\mathcal{A}$,
	we have that the experiments $\exptReal(1^\lambda, 1^d)$ and $\exptSim(1^\lambda, 1^d)$ are statisically close
	as a function of $\lambda$. The experiments are defined as follows:

	% TODO: independent Latex?
	\begin{itemize}
		\item
		      $\exptReal(1^\lambda, 1^d)$: \begin{enumerate}
			      \item On input the security parameter $1^\lambda$ and depth bound $d$, the adversary outputs $\mathbb{A} \in \mathbb{S}$.
			      \item Run $\TFHESetup(1^\lambda, 1^d, \mathbb{A})$ to obtain $(\TFHEPk, \TFHESk_1, ..., \TFHESk_N)$.
						The adversary is given $\TFHEPk$.
						\item The adversary outputs a set $S \subseteq \set{P_1, ..., P_N}$ such that $S \notin \mathbb{A}$
						together with plaintest messages $\mu_1, ..., \mu_k \in \set{0, 1}$. The adversary is handed over $\{sk_i\}_{i \in S}$
						\item For each $\mu_i$, the adversary is given $\TFHEEncrypt(\TFHEPk, \mu_i) \rightarrow \ct_i$.
						\item The adversary issues a polynomial number of queries, $(S_i \subseteq \set{P_1, ..., P_N}, C_i)$.
						for circuites $C_i: \set{0, 1}^k \rightarrow \set{0, 1}$. After each query the adversary receuves for $l \in S_i$ the value
						$$
						\TFHEPartDec(\TFHEEval(C_i, \ct_1, ..., \ct_k), \TFHESk_l) \rightarrow p_l
						$$
						\item $\advers$ outputs $\texttt{out}$, the experiment's output.
						% TODO?
		      \end{enumerate}
		\item
		      $\exptSim(1^\lambda, 1^d)$: \begin{enumerate}
			      \item On input the security parameter $1^\lambda$ and depth bound $d$, the adversary outputs $\mathbb{A} \in \mathbb{S}$.
			      \item Run $\TFHESetup(1^\lambda, 1^d, \mathbb{A})$ to obtain $(\TFHEPk, \TFHESk_1, ..., \TFHESk_N)$.
						The adversary is given $\TFHEPk$.
						\item $\advers$ outputs a set $S^* \subseteq \set{P_1, ..., P_N}$ such that $S \notin \mathbb{A}$
						and plaintexts $\mu_1, ..., \mu_k \in \set{0, 1}$. The simulator is given $\TFHEPk, \mathbb{A}, S^*$ as input
						and outputs $\set{sk_i}_{i \in S^*}$ and the state $\texttt{state}$. The adversary is given $\{sk_i\}_{i \in S^*}$
						\item For each $\mu_i$, the adversay is given $\TFHEEncrypt(\TFHEPk, \mu_i) \rightarrow \ct_i$.
						\item $\advers$ issues a polynomial number of queries of the form $(S_i \subseteq \set{P_1, ..., P_N}, C_i)$
						\item for circuits $C_i: \set{0, 1}^k \rightarrow \set{0, 1}$. After each query, the simulator computes
						$$
							\simTFHE(C_i, \set{\ct_l}_{l=1}^k, C_i(\mu_1, ..., \mu_k), \texttt{state}) \rightarrow \set{p_l}_{l \in S_i}
						$$
						and sends $\set{p_l}_{l \in S_i}$ to the adversary.
						\item $\advers$ outputs $\texttt{out}$, the experiment's output.

		      \end{enumerate}
	\end{itemize}

\end{definition}

\subsection{Non Interactive Zero-Knowledge}
The following definition is taken almost verbatim from \cite{catalano2022adaptively}.
A non-Interactive Zero-Knowledge (NZIK) proof system for relationship $\proofRel$
is a tuple of PPT algorithms $(\NZIKGen, \NZIKProve, \NZIKVerify)$ such that $\NZIKGen$ generates a common reference string, $\NZIKCRS$,
$\NZIKProve(\NZIKCRS, x, w)$ given $(x, w) \in \proofRel$, outputs a proof $\proofElem$,
and $\NZIKVerify(\NZIKCRS, x, \proofElem)$ outputs 1 if $(x, w) \in \proofRel$ and 0 otherwise.
A $\NZIK$ is correct is for every $\NZIKCRS$ and all $(x, w) \in \proofRel$, we have that\\
$\NZIKVerify(\NZIKCRS, x, \NZIKProve(\NZIKCRS, x, w)) = 1$ holds with probability 1.
We also require that our NZIKs satisfy the notions of weak simulation extractability \cite{sahai1999non} and 
zero-knowledge \cite{feige1990multiple}.

Weak simulation extractability guarentees the extractability of proofs produced
by the adversary that are not equal to proofs previously observed. Thus, we make each
proof ``unique'' by implicitly adding a session ID to the statement. For more of a commentary,
see section 2.7 of \cite{catalano2022adaptively}. We will not detail how to handle these session IDs.

% TODO: remove above?
We recall the notion of simulation security from \cite{sahai1999non}:

\begin{definition}[NZIK simulation security]
	We say that a proof system for relationship $\proofRel$ is simulation secure if proof system
	$(\NZIKGen, \NZIKProve, \NZIKVerify)$ is a non-interactive proof system and $\simulator_1, \simulator_2$ are PPT algorithms
	such that for all PPT adversaries $\advers_1, \advers_2$ we have that \\
	$ \left|\Pr[\exptReal(\lambda) = 1] - \Pr[\exptSim(\lambda) = 1]\right|$ is negligible in $\lambda$.
	The experiments are defined as follows:
	\begin{itemize}
		\item $\exptReal$:
		\begin{enumerate}
			\item $\NZIKCRS \gets \NZIKGen(1^\lambda)$
			\item $\advers_1$ outputs $(x, w, \texttt{state}_1)$
			\item $\proofElem \gets \NZIKProve(\NZIKCRS, x, w)$
			\item return $\advers_2(\proofElem, \texttt{state}_1)$
		\end{enumerate}
		\item $\exptSim$:
		\begin{enumerate}
			\item $\NZIKCRS \gets \simulator_1(1^\lambda)$
			\item $\advers_1$ outputs $(x, w, \texttt{state})$
			\item $\proofElem \gets \simulator_2(\NZIKCRS, x, w)$
			\item return $\advers_2(\proofElem, \texttt{state})$
		\end{enumerate}
	\end{itemize}
\end{definition}
% Oh wait we do not need this

We also have that a NZIK has ideal functionality $\NZIKIdeal$ which is defined as follows:
\begin{figure}[H]
	\fbox{
		\begin{minipage}{1\textwidth}
			\textbf{The $\NZIKIdeal$ functionality for zero-knowledge}:\\
			Upon receiving $(\texttt{prove}, sid, x, w)$ from $P_i$, with $sid$ being used for the first time,
			if $(x, w) \in \proofRel$, broadcast $(\texttt{proof}, sid, i, \proofElem)$, otherwise broadcast $\bot$.
		\end{minipage}
	}
\end{figure}

\subsection{Data Independent Priority Queue}
In this work, we will use data independent queues as studied in \cite{toft2011secure, mitchell2014data, mazloom2023efficient}.
Data independent data structures are unique as their control flow and memory access do not depend on input data (\cite{mitchell2014data}).

\begin{definition}[Word RAM model \cite{mitchell2014data}]
	In the word RAM model, the RAM has a constant number of public and secret registers and can perform arbitrary operations on a constant number of registers in constant time.
\end{definition}

% TODO: slightly modified definition by me. Check with Mark and come back to it
\begin{definition}[Data Independent Data Structure \cite{mitchell2014data}]
	In the word RAM model, a data independent data structure is a collection of algorithms where
	all the algorithms uses RAM such that the RAM can only set its	control flow based on registers that are public.
\end{definition}

Data independent queues are especially useful as they allow for efficient computation within MPC and FHE as control flow is not dependent
on underlying ciphertexts data. We use a data independent queue as outlined in \cite{mazloom2023efficient}
which allows for
\begin{itemize}
	\item $\PQInsert$: Inserts a tag and value, $(p, x)$ into $\PQ$ according to the tag's priority.
	\item $\PQExtract$: Removes and returns the $(p, y)$ with highest tag priority.
	\item $\PQFront$: Returns the $(p, y)$ with highest tag priority without removing the element.
\end{itemize}
Moreover, we note that the order is stable. I.e.\ the first inserted among equal tagged elements has a higher priority.


\subsection{Resevoir Sampling}
Resevoir sampling is an online algorithm which allows for randomly selecting
$k$ elements from a stream of $n$ elements while using $\tilde{O}(k)$ space.
Algorithm R (\cite{vitter1985random}) is a simple algorithm which relies on a priority queue with interface:
\begin{itemize}
	\item $\ResevInit(k)$ initialize the resevoir sampling algorithm and data structure
	\item $\ResevInsert(\mu_i, e_i, \texttt{coin}_i)$ where $\mu_i$ is $i$-th item, $e_i$ is independentally sampled randomness,
	      and $\texttt{coin}_i$ is a random coin with probability $1/m$ of equaling 1.
	      \begin{itemize}
		      \item If $i \leq k$, insert the item into the queue along with $e_i$ as its tag via $\PQInsert(e_i, \mu_i)$.
		      \item If $i > k$ and $\texttt{coin}_i = 1$, replace the smallest labeled item in the queue with the new item if the coin is 1.
		      \item If $i > k$ and $\texttt{coin}_i = 0$, do nothing.
	      \end{itemize}

	\item  $\mu_{a_1}, \mu_{a_2}, ..., \mu_{a_k} \gets \ResevOutput()$ where $a_1, ..., a_k$ are a uniformly random ordered susbset of $[n]$
	      \begin{itemize}
		      \item Call $\PQExtract$ $k$ times setting $\mu_{a_\ell}$ to the $\ell$-th call to $\PQExtract$ where $\ell \in [k]$.
	      \end{itemize}
\end{itemize}

