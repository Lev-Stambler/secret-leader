\section{Semi Honest Reservoir Sampling Based Protocol}
\label{sec:msle_protocol}
\newcommand{\partDecrySet}{\mathcal{P}}
We outline a semi-honest protocol in \cref{fig:protocolMSLE} in the common reference string model for the MSLE functionality.

We define the setup, $\setup$, for the protocol as follows:
\begin{itemize}
	\item Set $k_{\texttt{TFHE}} \overset{\$}{\leftarrow} U_q.$
	\item Publish CRS, $\crs = \Enc(k_{\texttt{TFHE}})$.
	      % This will be the key to the PRF
\end{itemize}



\begin{figure}[H]
	\centering
	\fbox{
		\begin{minipage}{1\textwidth}
			\textbf{The MSLE Protocol}, Party $P_i$ realizing $\idealMSLE$ in the semi-honest setting:
			\\
			% Fix some $k \in \N$ to denote the number of rounds.
			% Initialize an empty set of tickets, $R$.
			% Initialize an empty lookup of sets of parties in each election, $S$.
			Each party has as input a TFHE secret key share $\TFHESk_i$ of secret key $\TFHESk$
			with associated public key $\TFHEPk$.
			Initialize a set of finished elections $\FinElect$.
			% Initialize an empty lookup of election results, $E$.
			% Start with some fixed TFHE secret key $\TFHESk$ and public key $\TFHEPk$ where.
			% each party, $P_i$ has a share of $\TFHESk_i$, $\TFHESk_i$.
			% TODO: formal

			% \item $\register$ from party $P_i$
			%       \begin{itemize}
			% 	      %  TODO: not within definition of TFHE, have to change up definition :((
			% 	      %  TODO: formalize secure channel?
			% 	      \item If party $P_i$ does not already have a TFHE share, create a TFHE share, $\TFHESk_i$, for $P_i$
			% 	            and send the share over a secure channel to $P_i$.

			% 	      \item $R \gets R \cup \{(i, \nBids)\}$, set $\nBids \gets \nBids + 1$.
			%       \end{itemize}
			% \item Add $\Enc(c_i)$ to $C$.
			% \item $n \gets n + 1$

			\begin{itemize}
				\item $(\registerElect, \eid)$
				      \begin{itemize}
					      \item Sample $s_i, r_i \randomGet \F_q$, $\ct_i \gets \Enc(c_i)$
					      \item Broadcast $(\registerElectAdd, \eid, \ct_i)$ to all parties and run \\$(\registerElectAdd, \eid, \ct_i)$
				      \end{itemize}
				\item $(\registerElectAdd, \eid, \ct_j)$ from party $P_j$ for $j \in [n]$
				      \begin{itemize}
					      \item If $b_\eid$ is not stored, set $b_\eid = 0$
					      \item $b_\eid \gets b_\eid+1$
					            %  Not needed because semi-honest case
					            % \item If $\eid \in \FinElect$, do nothing.
					            % \item If $(i, w) \in S_\eid$ or $(i, w) \notin R$, send $\bot$ and do nothing.
					      \item If $\resSampling_\eid$ is not stored, $\resSampling_\eid \gets \ResevInit(k)$
					      \item Set $\Enc(e_i), \Enc(\texttt{coin}_i) \gets \TFHEEval(C_{PRF}, \crs, \ct_j)$
					            where $C$ is the PRF circuit with polynomial stretch and $\crs$ is the key to the PRF. 
					            % TODO: formally define PRF
					      \item Call $\resSampling_\eid \gets \TFHEEval(C, \resSampling_\eid, \ct_j, \Enc(e_i), \Enc(\texttt{coin}_i))$ where
					            $C$ is the reservoir sampling circuit for $\ResevInsert$. % TODO: is this right defn of resev insert?
					            $\ResevInsertN{\eid}(\Enc(c_j), \Enc(e_i), \Enc(\texttt{coin}_i))$.
					      \item Store $\resSampling_\eid$.
				      \end{itemize}

				\item $(\elect, \eid)$ called by user $P_i$ when  $b_\eid = n$
				      \begin{itemize}
					      % \item If $\eid \in \FinElect$, then return $\bot$ and do nothing.
					      \item For $\ell \in [k]$, set $\ct_{\ell} \gets \TFHEEval(C_\ResevOutput, \resSampling_\eid)$
					      \item Set $p_i \gets \TFHEPartDec(\ct_1, ..., \ct_k, \TFHESk_i)$
					      \item Broadcast $(\electDone, \eid, p_i)$ and call $(\electDone, \eid, p_i)$
				      \end{itemize}
				\item $(\electDone, \eid, p_j)$ from party $P_j$
				      \begin{itemize}
					      \item Retrive $\partDecrySet^\eid$ if it is stored, otherwise set $\partDecrySet^\eid \gets \emptyset$.
					      \item Set $\partDecrySet^\eid \gets \partDecrySet^\eid \bigcup \{p_j\}$
					      \item If $|\partDecrySet^\eid| = n$, set $c_{a_1}, ..., c_{a_\ell} \gets \TFHEFinDec(\partDecrySet^\eid)$
					      \item Store $\partDecrySet^\eid$.
				      \end{itemize}
				\item $(\reveal, \eid, \ell)$ from party $P_i$ if $c_{a_\ell} = c_i$.
				      \begin{itemize}
					      \item Broadcast $(\texttt{result}, \eid, \ell, i)$
				      \end{itemize}
				      % \item $(\revealClaim, \eid, \ell, j)$ from party $P_j$
				      % 			\begin{itemize}
				      % 				\item Broadcast $(\texttt{result}, \eid, \ell, j)$
				      % 			\end{itemize}
			\end{itemize}
		\end{minipage}
	}
	\caption{Description of the MSLE protocol}
	\label{fig:protocolMSLE}
\end{figure}


\subsection{Semi Honest Simulation Security}


\newcommand{\simCMSLE}{\texttt{Sim}_C}

We will now show that the above protocol is semi-honest simulation secure.

% TODO: one election??
\begin{theorem}[Semi-honest simulation security]
	Assuming the existence of a PRF with polynomial stretch, a $\TFHE$ scheme with semantic security (\cref{def:semanticSecurityTFHE}) and simulation security (\cref{def:simSecTFHE}),
	then the protocol outline in \cref{fig:protocolMSLE} is semi-honest simulation secure.
	\begin{proof}
		We will show that the simulator outlined in \cref{fig:simMSLE} is a simulator for the view of corrupt
		parties $C \subset [n]$ and $|C| < t$ by showing that the simulator's view and the protocol's view are indistinguishable for all calls
		via the following three lemmas.

		% TODO: ordering?
		\begin{lemma}
			The views of $\registerElect, \registerElectAdd$ are simulated by the simulator.
			\begin{proof}
				% Note that the view of the protocol is $(s_i, r_i, \ct_i)$ for $\registerElect$.
				The simulator firsts simulates the view of the protocol for all calls to $\registerElect$ from $P_i$, $i \in C$.
				Note that the simulator uses the same inputs, $s_i, r_i$ as in the protocol and thus the view is identical.
				Moreover, the view of $(\registerElectAdd, \eid, \ct'_i)$ is identical to the real protocol as $\ct'_i = \ct_i$.
				For $j \notin C$, note that $c'_j \compInd c_j$ by the hiding property of commitments and thus
				$\ct'_j \compInd \ct_j$.
				Then, assuming that prior calls to $\registerElect$ by $P_j$ are simulated by the simulator,
				the view of $(\registerElectAdd, \eid, \ct'_j)$ is indistinguishable to the real protocol as
				$\registerElectAdd$ is completely determined by prior calls to $\registerElectAdd$ and $\ct'_j$.

				Let $\resSampling_{\eid, \simCMSLE}$ denote the reservoir sampling data-structure in the simulator.
				We then note that, after $\alpha$, for $\alpha \in [n]$,  calls to $\registerElectAdd$, the simulator's $\resSampling_{\eid, \simCMSLE} \compInd \resSampling_\eid$
				of the protocol as $\resSampling$ is completly determined by the calls to $\registerElectAdd$.
			\end{proof}
		\end{lemma}
		\begin{lemma}
			The views of $\elect, \electDone$ are simulated by the simulator.
			\begin{proof}
				When $\elect$ is called by $P_i$ for $i \in C$, we have that for $\ct'_\ell = \TFHEEval(C_\ResevOutput, \resSampling_{\eid, \simCMSLE})$, $\ct'_\ell \compInd \ct_\ell$ for all $\ell \in [k]$.
				This is because $\resSampling_{\eid, \simCMSLE}$ is indistinguishable from that of the real protocol.
				We also have that, by the ideal functionality of $\elect$, $\set{a'_\ell}_{\ell \in [k]}$ is a random subset of $[n]$ with random order with size $k$.
				Then, note that $e_i, \texttt{coin}_i = PRF(k_\TFHE, c_i)$ in the protocol is indistinguishable from a randomly sampled $e_i, \texttt{coin}_i$
				by the semantic security of PRFs. Thus, by the correctness of $\ResevOutput(\resSampling)$, $\set{a_\ell}_{\ell \in [k]}$ is a randomly chosen ordered subset of $[n]$.
				So then, $\set{a'_\ell}_{\ell \in [k]} \compInd \set{a_\ell}_{\ell \in [k]}$.
				Remembering that $c'_\beta \compInd c_\beta$ for all $\beta \in [n]$, we have that $c'_{a_1}, ..., c'_{a_k} \compInd c_{a_1}, ..., c_{a_k}$.
				By the simulation security of $\TFHE$, we have that the decryption shares, $p'_1,..., p'_n = \simTFHE(C_{\texttt{Ident}}, (\set{ct'_\ell}, c'_{a_1}, ..., c'_{a_k}), [n], \texttt{st})$, for all $i \in [n]$ are simulated by the simulator such that $p'_i \compInd p_i$ in the real protocol.
				And thus, all calls to $\electDone$ are indistinguishable from the real protocol as $\electDone$ is completely
				determined by $\set{p_i}_{i \in [n]}$.
			\end{proof}
		\end{lemma}
		\begin{lemma}[$\reveal$ is simulation secure]
			\begin{proof}
				Finally, note that $\reveal$ in the simulator is identical to that of the real protocol
				as in the semi-honest setting, only parties which have won an election will call $\reveal$.
			\end{proof}
		\end{lemma}
		We thus have that the view of the simulator is identical to that of the real protocol by the above three lemmas.
	\end{proof}
\end{theorem}


\begin{figure}
	\centering
	\fbox{
		\begin{minipage}{1\textwidth}
			\textbf{Simulator for Threshold MSLE} $\simCMSLE$ where $C \subseteq [n]$ and $|C| < t$:\\
			% Initialize an empty set of tickets, $R$, an empty lookup of sets of parties in each election, $S$,
			Initializie a set of finished elections $\FinElect$ and set a random tape for the simulator.
			% an empty lookup of election results, $E$,
			% and a set $S \gets \emptyset$ to denote the set of sets of active participants in each round.
			% The simulator knows, input for party $P_i$, $\TFHESk_i$ and $s_i$.

			\begin{itemize}
				% \item $\register$ from party $P_j$
				%       \begin{itemize}
				% 	      \item $R \gets R \cup \{(i, \nBids)\}$, set $\nBids \gets \nBids + 1$.
				% 	            %  Can we use a different secret key here?
				% 	      \item Simulate the secure channel communication with party $P_j$ if $i \neq j$.
				% 	      \item Run $\register$ honestly if $i = j$ by sending $\TFHESk_i$ to $P_i$
				% 	            % TODO: formal
				%       \end{itemize}
				\item $(\registerElect, \eid)$ for party $P_j$
				      \begin{itemize}
					      \item If $j \in C$ Follow the protocol's $\registerElect$ using $P_j$'s $s_j, r_j, c_j, \ct_j$ from the real protocol and call $(\registerElectAdd, \eid, \ct_j)$.
					            Set $c'_j \gets c_j$ and store $c'_j$.
					      \item If $j \notin C$, $s'_j, r'_j \randomGet \F_q, c'_j \gets \comm(s'_j, r'_j), \ct'_j \gets \Enc(c'_j)$
					            Store $c'_j$ and follow the protocol by calling $(\registerElectAdd, \eid, \ct'_j)$.
				      \end{itemize}

				      % \item $(\registerElectAdd, \eid)$ from party $P_j$.
				      %       \begin{itemize}
				      % 	      \item If $j \notin C$, $s'_j, r'_j \randomGet \F_q, c'_j \gets \comm(s'_j, r'_j), \ct'_j \gets \Enc(c'_j)$.
				      % 				 	Store $c'_j$ and follow the protocol for $\registerElectAdd$ directly using $\ct'_j$ as input.
				      % 				\item If $j \in C$, use $\ct_j$ from the real protocol and follow the protocol for $\registerElectAdd$ directly.
				      %       \end{itemize}

				\item $(\elect, \eid, \texttt{out})$ from party $P_j$ where $\texttt{out}$ is $\idealMSLE.\elect$ output $(\texttt{outcome}, \eid, q)_i$ for all $i \in [n]$ where $q \in \set{1, ..., k, \bot}$.

				      \begin{itemize}
					      \item For $\ell \in [k]$, set $\ct'_\ell \gets \TFHEEval(C_\ResevOutput, \resSampling_\eid)$
					      \item Let $a'_1, ..., a'_k \in [n]$ such that if $q_i = \ell$ then $a'_\ell = i$.
					      \item Call $p'_1, ..., p'_{n} \gets \simTFHE(C_\texttt{Ident}, (\set{\ct'_\ell}, c'_{a'_1}, ..., c'_{a'_\ell}), [n], \texttt{st})$
					            % Note that this sets the decryption of the output of the TFHE circuit sampling to $c'_{a'_\ell}$ for $\ell \in [k]$.
					            % \item Store $E_\eid = \left\{c'_{a'_1}, ..., c'_{a'_k}\right\}$
					      \item Call $(\electDone, \eid, p_j)$ from the protocol.
				      \end{itemize}
				      % \item $(\electDone, \eid)$ from party $P_j$
				      % 			\begin{itemize}
				      % 				\item If $j \in C$, then follow the protocol for $\electDone$ directly using $p_j$.
				      % 			\end{itemize}
				\item $(\texttt{reveal}, \eid, \ell)$ for $P_j$ and $\idealMSLE.\reveal$ output $(\texttt{result}, \eid, \ell, j)$ or $(\texttt{reject}, \eid, \ell, j)$.
				      \begin{itemize}
					      % TODO: faulty? oh yeah not here...
					      % \item If $\idealMSLE.\reveal$ outputs $\texttt{results}$
					      \item Broadcast $(\texttt{result}, \eid, \ell, j)$
				      \end{itemize}
			\end{itemize}
		\end{minipage}
	}
	\caption{Description of the MSLE protocol}
	\label{fig:simMSLE}
\end{figure}

% We will now show that $\simIMSLE$ is indeed a simulator for the view of $\pi_{\text{MSLE}}$
% for $\registerElect$, $\elect$, and $\reveal$.

% \begin{lemma}
% 	[$\registerElect$ is simulation secure]
% 	\label{lemma:regElectSemiHonest}
% 	\begin{proof}
% 		The simulator gets as input party $i$'s secret value, $s_i$, party $i$'s $\TFHE$ share $\TFHESk_i$,
% 		$\eid$, and ticket number $w$. We note that if $\eid \in \FinElect$ or $(i, w) \notin R$, then the simulator can simply output $\bot$ and
% 		is identical to the real protocol.
% 		Otherwise, if party $P_i$ calls $\registerElect$, then the simulator knows the input of $P_i$ and can simulate the protocol
% 		honestly. If party $P_j$ call $\registerElect$ for some $j \neq i$, then the view of the protocol
% 		is that of $$(\Enc(c_j), \Enc(e_i, \texttt{coin}_i), \ResevInsertN{\eid}(\Enc(c_j, e_i, \texttt{coin}_i))).$$
% 		Note that this view is completely determined by $\Enc(c_j)$. Also, note that $c_j$ is drawn
% 		from a random distribution and is not in the view of the real protocol. Thus,
% 		by semantic security of $\TFHE$ (\cref{def:semanticSecurityTFHE}) we have that
% 		$\Enc(c_j) \compInd \Enc(c'_j)$ where $c'_j$ is a commitment to a random value.
% 	\end{proof}
% \end{lemma}

% \begin{lemma}[$\elect$ is simulation secure]
% 	\begin{proof}
% 		If the view of the real protocol is $\bot$ because $\eid$ was already called, then the simulator can simply output $\bot$
% 		and is thus identical to the real protocol.
% 		Otherwise, note that the view of the protocol is
% 		$$
% 			(c_{a_1}, ..., c_{a_k}, p_1, ..., p_t, \texttt{view}(C(\Enc(c_{a_1}, e_1, \texttt{coin}_1), ..., \Enc(c_{a_k}, e_k, \texttt{coin}_k) ))).
% 		$$
% 		We will now show that simulator can simulate the above view.
% 		The simulator needs to ``fix'' the output of $\elect$ to yield
% 		a list of commitments such that, if $a'_\ell = j$ if party $j$ wins election $\ell$.
% 		The TFHE simulator can simply output the decryption shares, $p_1, ... p_t$ using
% 		$$\simTFHE(C, \set{\Enc(c'_j, e_j, \texttt{coin}_j)}_{j \in [\totalReg]}, c'_1, ..., c'_{a_\ell}, S, \texttt{st}).$$
% 		Note that the reservoir sampling circuit is indeed simulated by $\simTFHE$
% 		and the decryption shares are simulated as well such that the cipher texts decrypt
% 		to $c'_{a'_1}, ..., c'_{a'_k}$.
% 		We also have that, by the ideal functionality of $\elect$, $\set{(a'_\ell, .)}_{\ell \in [k]}$ is a randomly chosen
% 		ordered subset from $S_\eid$. We can note that, by the correctness of reservoir sampling,
% 		the output of $\ResevOutputN{\eid}()$ in the protocol is a randomly chosen ordered subset from $S_\eid$.
% 		We then have that the distribution of the simulator's $\set{(a'_\ell, .)}_{\ell \in [k]} \equiv \set{(a_\ell, .)}_{\ell \in [k]}$
% 		where $a_\ell$ is the party that won election $\ell$ in the protocol.
% 		We then have that, by the simulation security of $\registerElect$ (\cref{lemma:regElectSemiHonest}), $c'_{a'_1}, ..., c'_{a'_k} \compInd c_{a_1}, ..., c_{a_k}$ .
% 	\end{proof}
% \end{lemma}

% \begin{lemma}[$\reveal$ is simulation secure]
% 	\begin{proof}
% 		Note that the simulator can simply run $\reveal$ honestly as the simulator has knowledge of the openings to all of the commitments $c'_j$ used
% 		and thus has an identical view to that of the real protocol.
% 	\end{proof}
% \end{lemma}