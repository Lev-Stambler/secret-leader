\section{MSLE Protocol}
\label{sec:msle_protocol}
We use a similar notion of ideal functionality for a multi-secret leader election from the ideal
functionality of single secret leader election of \LS{CITE}.

\begin{figure}[ht]
	\centering
	\fbox{
		\begin{minipage}{1\textwidth}
			\textbf{The MSLE functionality} $\idealMSLE$:
			Initialize $E, R \gets \emptyset, \gets 0$. Fix some $k \in \N$ to denote the number of rounds. Upon receiving,
			\begin{itemize}
				\item $\register$ from party $P_i$, set $R \gets R \cup \{(i, n\}$, broadcast $(\texttt{register}, i)$ to all parties and set $n \gets n + 1$
				\item $\elect$ Elect $k$ leaders from $S \subseteq R$ parties. If $|S| \geq k$ and $\eid$ was not requested before,
				      randomly sample $W^{\eid} \subseteq S$ where $|W^{\eid}| = k$.
				      Then, assign a random ordering to $W^\eid$ to get ordered set $E^\eid$.
				      Next, send $(\texttt{outcome}, \eid, a)$ to $P_j$ for all $E^\eid_a = (j, \cdot)$
				      and $(\texttt{outcome}, \eid, \bot)$ to $P_i$ if $(i, \cdot) \notin E^\eid$. Store $E \gets E \bigcup \{E^\eid\}$.
				\item $\reveal$ from $P_i$: for $E_\eid \in E$, if $i = E^\eid_a$, broadcast $(\texttt{result}, \eid, \ell, i)$.
				      Otherwise, broadcast $(\texttt{rejected}, \eid, \ell, i)$.
			\end{itemize}
		\end{minipage}
	}
	\caption{Description of the MSLE functionality, heavily based on the description of SSLE in \LS{CITE Dario and CFG21}}.
	\label{fig:my_label}
\end{figure}

\begin{figure}
	\centering
	\fbox{
		\begin{minipage}{1\textwidth}
			\textbf{The MSLE Protocol} $\pi_{\text{MSLE}}$:
			Initialize $n = 0$. Fix some $k \in \N$ to denote the number of rounds.
			Also, sample ;
			this will be the key FHE to the PRF.

			\begin{itemize}
				\item $\init$
				      \begin{itemize}
					      \item Set $n = 0$
					      \item Sample some secret $s_{\texttt{TFHE}}$ and publicly publicly publish $\Enc(s_{\texttt{TFHE}})$.
					            This will be the key to the PRF
				      \end{itemize}

				\item $\register(\Enc(c_i))$
				      \begin{itemize}
					      %  TODO: not within definition of TFHE, have to change up definition :((
					      \item If party $P$ does not already have a TFHE share, create a TFHE share for $P_i$.
					            % \item Add $\Enc(c_i)$ to $C$.
					            % \item $n \gets n + 1$
				      \end{itemize}
				\item $\electReal$ where $c_i = \comm(s_i)$ for secret $s_i$ to party $P_i$
				      \begin{itemize}
					      \item An encrypted streaming sampler will be publicly initialized
					      \item the CRS will be used to run a PRF to get $\Enc(e_i) = \Enc(\texttt{PRF}(c_i, s_\texttt{TFHE}))$ for $i \in S$
					      \item All $\Enc(c_{S_i})$ will be fed to the encrypted streaming sampler along with randomness $\Enc(e_i)$
					      \item The encrypted streaming sampler will output a list of $k$ messages: $\Enc(c_{a_1}), \Enc(c_{a_2}), ..., \Enc(c_{a_k})$.
					      \item Then, at least $t$ parties will submit decryption shares to get $c_{a_1}, ..., c_{a_k}$.
					      \item Each party will then check if they won an election by seeing if their commitment is in the list of decrypted messages.
				      \end{itemize}
				\item $\texttt{reveal}(\eid, \ell, \Enc(c'_i))$ from $P_i$
				      \begin{itemize}
					      \item $P_i$ submits a proof that they know the opening to $c_{a_\ell}$.
								If this proof verifies, send out $(\texttt{result}, \eid, \ell, i)$ to all parties.
					      Otherwise, send out $(\texttt{rejected}, \eid, \ell, i)$ to all parties.
				      \end{itemize}
			\end{itemize}
		\end{minipage}
	}
	\caption{Description of the MSLE protocol}
	\label{fig:protocolMSLE}
\end{figure}

\subsection{Simulation Security}
To show simulation security, we will prove that, given a party's input and output in the ideal model,
a simulator can simulate the distribution of the view in the protocol for the party. Note that because the
protocol is deterministic, it suffices to prove simulation for only the party's output.

More formally we will show that for party $i$,
\begin{equation}
	\label{eq:simSecReg}
	\simulator_i\left(s_i, r_i, \idealMSLE.\register_i\right) \compInd \texttt{view}_{\register_i}((s_0, r_0), ..., (s_n, r_n)),
\end{equation}
\begin{equation}
	\label{eq:simSecElect}
	\simulator_i\left(s_i, b_i, r_i, \idealMSLE.\elect_i\right) \compInd \texttt{view}_{\elect_i}((s_0, b_0), ..., (s_n, b_n)),
\end{equation}
and
\begin{equation}
	\label{eq:simSecReveal}
	\simulator_i\left(b_{a, 1}, \dots, b_{a, k}, s_i, r_i, \idealMSLE.\reveal_i\right) \compInd \texttt{view}_{\reveal_i}(),
\end{equation}

\begin{lemma}
	We will first show that \cref{eq:simSecReg} is simulation secure.
	\begin{proof}
		The view of each party $i$ for $\register$ can be expressed as
		\begin{equation*}
			(r_i, s_i, c_i, C, \Enc(c_i), n)
		\end{equation*}
		We can create a simulator $\simulator_i$ that takes as input $s_i, r_i, n$ and outputs
		an indistinguishable view
		\begin{enumerate}
			\item Sample something? Does the simulator have access to C?
		\end{enumerate}
	\end{proof}
\end{lemma}


\begin{lemma}
	We will now show that \cref{eq:simSecElect} is simulation secure.
	\begin{proof}
		The view of each party $i$ for $\elect$ can be expressed as
		\begin{align*}
			(r_i,  s_i, \Enc(b_{1}), \dots \Enc(b_n), \Enc(r'_1), \dots, \Enc(r'_n), \\
			\Enc(b_{a_1}), \dots, \Enc(b_{a_k}), \sigma_1, \dots \sigma_t, b_1, \dots, b_n, y)
		\end{align*}
		where $r'_i$ is the randomness from the streaming sampler, $\sigma_1, \dots \sigma_t$ are the decryption shares,
		and $y \in \set{\bot, 1, ..., k}$ representing whether a party won election $1$ through $k$ or not ($\bot$).
		Then, we have the simulator proceed in the following manner:
		\begin{enumerate}
			\item The simulator sets a random, local tape
			      % TODO: question, are these c_js part of the protocol/ commitment or is it valid to do these samples?
			      % (I.e. they are not messages)...
			\item The simulator samples $c'_j$ for all $j \neq i$ where $c'_j$ is a commitment to a random value
			\item The simulator samples some randomness $r$ and computes $\Enc(e_1), \dots, \Enc(e_n) = \Enc(\texttt{PRF}(r, s_\texttt{TFHE}))$
			\item The simulator creates an encrypted priority queue $\EncPQ$ and simulates the encrypted streaming sampler for all inputs $\Enc(c'_j)$ for $j \in [i]$.
			\item The simulator runs the encrypted streaming sampler to get the outputs $\Enc(c'_{a_1}),... \Enc(c'_{a_k})$.
			\item The simulator then chooses a random, ordered subset $S \subseteq [n]$ where $|S| = k$.
			      If there is some $w$ such that $S_w = i$, then set $y' = w$. Otherwise, set $y' = \bot$.
			\item The simulator then sets the decryptions of $\Enc(c'_{a_\ell})$ to $c'_{S_\ell}$.
			      The simulator also creates shares $\sigma_j$ such that $\Enc(c'_{a_\ell})$ decrypts to $c'_{S_\ell}$.
			\item The simulator then outputs $y'$

		\end{enumerate}
		We now use a sequence of hybrids to show that the view of the real protocol is indistinguishable from
		that of the simulated one
		\begin{itemize}
			\item $\hybrid{0}$: The real protocol
			\item $\hybrid{1}$: As $\hybrid{0}$ but, for all $j \neq i$, $\Enc(c_j)$ are replaced with $\Enc(c'_j)$, where $c'_j$ is a commitment to a random value.
			      We can see that $\hybrid{0} \equiv \hybrid{1}$ by the hiding property of commitments.
			\item $\hybrid{2}$: As $\hybrid{1}$ but replace $\Enc(c_{a_1}),... \Enc(c_{a_k})$ with
			      $\Enc(c'_{a_1}),... \Enc(c'_{a_k})$, the output of sampling the encrypted streaming sampler with $\Enc(c'_j)$.
			\item $\hybrid{3}$: As $\hybrid{2}$ but replace $\Dec(\Enc(c'_{a_1})),... \Dec(\Enc(c'_{a_k}))$ with
			      $c'_{S_1},... c'_{S_k}$, where $S$ is the random subset chosen by the simulator.
			      Replace $\sigma_j$ with shares $\sigma'_j$ such that $\Enc(c'_{a_\ell})$ decrypts to $c'_{S_\ell}$.
			\item $\hybrid{4}$: As $\hybrid{3}$ but replace $y$ with $y'$.

			      % 	% TODO: better define threshold decryption as separate thing
			      % 	% TODO: remove separate ticket registration
			\item $\hybrid{5}$: The simulated protocol
		\end{itemize}
		% TODO: can I have proof be within bullet points
		% Question: can a simulator have access to information from prior calls? 
		% Ohhhhhhhhhhhhhhhh... I see output $y$ has to follow distribution as well.


	\end{proof}
\end{lemma}

\begin{lemma}
	We will now show that \cref{eq:simSecReveal} is simulation secure.
	\begin{proof}

	\end{proof}
\end{lemma}

