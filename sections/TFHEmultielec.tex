\section{TFHE MSLE Protocol}
\label{sec:msle_protocol}
We use a similar notion of ideal functionality for a multi-secret leader election from the ideal
functionality of single secret leader election of \LS{CITE}.

\newcommand{\FinElect}{\mathcal{E}}
\newcommand{\nBids}{b}
\begin{figure}[ht]
	\centering
	\fbox{
		\begin{minipage}{1\textwidth}
			\textbf{The MSLE functionality} $\idealMSLE$:
			Initialize $E, R \gets \emptyset, \nBids \gets 0$. Initialize $S$ to denote the set of sets
			of active participants in each round.
			Set $\FinElect \gets \emptyset$ to denote the set of finished elections.
			Fix some $k \in \N$ to denote the number of rounds. Upon receiving,
			\begin{itemize}
				\item $\register$ from party $P_i$, set $R \gets R \cup \{(i, n)\}$, broadcast $(\texttt{register}, i)$ to all parties and set $\nBids \gets \nBids + 1$
				\item $\registerElect(\eid, w)$ from party $P_i$.
				      If $\eid \in \FinElect$ send $\bot$ to $P_i$ and do nothing.
				      If $(i, w) \notin R$ or $(i, w) \in S_\eid$, send $\bot$ to $P_i$ and do nothing.
				      If $S_\eid$ is not defined, set $S_\eid \gets \{(i, w)\}$.
				      Otherwise, set $S_\eid \gets S_\eid \cup \{(i, w)\}$.
				\item $\elect(\eid)$ Elect $k$ leaders from $S_\eid \subseteq R$ parties.
				      If $|S| \geq k$ and $\eid \notin \FinElect$,
				      randomly sample $W^{\eid} \subseteq S_\eid$ where $|W^{\eid}| = k$.
				      Then, assign a random ordering to $W^\eid$ to get ordered set $E^\eid$.
				      Next, send $(\texttt{outcome}, \eid, a)$ to $P_j$ for all $E^\eid_a = (j, \cdot)$
				      and $(\texttt{outcome}, \eid, \bot)$ to $P_i$ if $(i, \cdot) \notin E^\eid$. Store $E \gets E \bigcup \{E^\eid\}$.
				      Set $\FinElect \gets \FinElect \cup \{\eid\}$.
				\item $\reveal$ from $P_i$: for $E_\eid \in E$, if $i = E^\eid_\ell$, broadcast $(\texttt{result}, \eid, \ell, i)$.
				      Otherwise, broadcast $(\texttt{rejected}, \eid, \ell, i)$.
			\end{itemize}
		\end{minipage}
	}
	\caption{Description of the MSLE functionality, heavily based on the description of SSLE in \LS{CITE Dario and CFG21}}.
	\label{fig:my_label}
\end{figure}

\newcommand{\resSampling}{\mathcal{R}}

\begin{figure}
	\centering
	\fbox{
		\begin{minipage}{1\textwidth}
			\textbf{The MSLE Protocol} $\pi_{\text{MSLE}}$:
			Initialize $\nBids = 0$. Fix some $k \in \N$ to denote the number of rounds.
			Initialize an empty set of tickets, $R$.
			Initialize an empty lookup of sets of parties in each election, $S$.
			Initialize an empty lookup of reservoir sampling data structures $\resSampling$ and a set of finished elections $\FinElect$
			Initialize an empty lookup of election results, $E$.

			Initialize $S \gets \emptyset$ to denote the set of sets of active participants in each round.

			\begin{itemize}
				\item $\init$
				      \begin{itemize}
					      \item Set $n = 0$
					      \item Sample a random $\TFHE$ secret key, public key pair $\TFHESk, \TFHEPk$ and publish $\TFHEPk$.
					      \item Sample some secret $s_{\texttt{TFHE}}$ and publish $\Enc(s_{\texttt{TFHE}})$.
					            This will be the key to the PRF
				      \end{itemize}

				\item $\register$ from party $P_i$
				      \begin{itemize}
					      %  TODO: not within definition of TFHE, have to change up definition :((
					      %  TODO: formalize secure channel?
					      \item If party $P_i$ does not already have a TFHE share, create a TFHE share, $\TFHESk_i$, for $P_i$
					            and send the share over a secure channel to $P_i$.

					      \item $R \gets R \cup \{(i, \nBids)\}$, set $\nBids \gets \nBids + 1$.
				      \end{itemize}
				      % \item Add $\Enc(c_i)$ to $C$.
				      % \item $n \gets n + 1$

				\item $\registerElect(\eid, \Enc(c_i), w)$ from party $P_i$ where $c_i = \comm(s_i)$ for a randomly sampled secret $s_i$.
				      \begin{itemize}
					      \item If $\eid \in \FinElect$, then send $\bot$ to $P_i$ and do nothing.
					      \item If $(i, w) \in S_\eid$ or $(i, w) \notin R$, send $\bot$ and do nothing.
					      \item Otherwise, if $\resSampling_\eid \notin \resSampling$, $\resSampling_\eid \gets \ResevInit(k)$
					      \item The CRS will be used to run a PRF to get $\Enc(e_i), \Enc(\texttt{coin}_i) = \Enc(\texttt{PRF}(c_i, s_\texttt{TFHE}))$ for $i \in S$
					      \item $\Enc(c_i)$ will be fed to the encrypted streaming sampler along with randomness via
					            $\ResevInsertN{\eid}(\Enc(c_i), \Enc(e_i), \Enc(\texttt{coin}_i))$.
				      \end{itemize}

				\item $\elect(\eid)$
				      \begin{itemize}
					      \item If $\eid \in \FinElect$, then return $\bot$ and do nothing.
					      \item The encrypted streaming sampler will output a list of $k$ messages via calling $\ResevOutputN{\eid}()$ $k$ times: $\Enc(c_{a_1}), \Enc(c_{a_2}), ..., \Enc(c_{a_k})$.
					            %  TODO: formal with TFHE decryption shares
					      \item Then, at least $t$ parties will submit decryption shares, $p_i = \TFHEPartDec(\Enc(c_{a_1}), ..., \Enc(c_{a_2}))$ to get $c_{a_1}, ..., c_{a_k} = \TFHEFinDec(\{p_i\})$.
					      \item Add $E_\eid = \left\{c_{a_1}, ..., c_{a_k}\right\}$ to $E$.
				      \end{itemize}
				\item $\texttt{reveal}(\eid, \ell, \Enc(c'_i))$ from $P_i$
				      \begin{itemize}
					      %  TODO: formal with commitments...
					      \item $P_i$ submits a proof that they know the opening to $c_{a_\ell}$.
					            If this proof verifies, send out $(\texttt{result}, \eid, \ell, i)$ to all parties.
					            Otherwise, send out $(\texttt{rejected}, \eid, \ell, i)$ to all parties.
				      \end{itemize}
			\end{itemize}
		\end{minipage}
	}
	\caption{Description of the MSLE protocol}
	\label{fig:protocolMSLE}
\end{figure}


\subsection{Semi Honest Simulation Security}


We can show semi-honest simulation security by showing that the view of each party $i$ in the real protocol
can be simulated throughout the course of one election and then that this simulation can be extended for a polynomial number of elections.

\newcommand{\totalReg}{\overline{n}}
\newcommand{\totalRev}{\overline{k}}
Let $\totalReg$ be the total number of calls to $\registerElect(\eid, ., .)$ before the election for $\eid$, $\elect(\eid)$
is called and that $k \leq \totalReg \leq n$ as if $\totalReg < k$, then $\elect$ returns $\bot$ and is trivial to simulate.
Further, let $\totalRev$ be the total number of calls to $\texttt{reveal}(\eid, ., .)$ which did not reject, again we have that $\totalRev \leq k$.
We will now show that a simulator can simulate the view of each party $i$ after $\totalReg$ calls to $\registerElect(\eid, .)$, a call to $\elect(\eid)$, and $\totalRev$ calls to $\reveal(\eid, .)$
for a fixed $\eid$.

\newcommand{\simIMSLE}{\texttt{Sim}_i}
% The simulator, $\simIMSLE$, will proceed as follows with input $s_i, \TFHESk_i, \eid, w$ and output of election
% $(\texttt{outcome}, \eid, q)_i$ for $i \in [\totalReg]$ and $(\texttt{result}, \eid, \ell, j)$ for $\ell \in K$
% where $K \subseteq [k]$ and $|K| = \totalRev$:

% \begin{enumerate}
% 	\item $\simIMSLE$ sets a random, local tape
% 	\item The simulator samples $c'_j$ for all $j \neq i$ where $c'_j$ is a commitment to a random value.
% 				The simulator also computes $c_i = \comm(s_i)$.
% 	\item The simulator runs $\registerElect(\eid, \Enc(c'_j), w_j)$ for $i \in [\totalReg]$ as in the protocol.
% 	\item The simulator runs the first two steps of $\elect(\eid)$, having the reservoir sample output $\Enc(c'_{a_1}), ..., \Enc(c'_{a_k})$.
% 	The simulator then gives $\simTFHE(C, \set{\Enc{c'_j}}_{j \in [\totalReg]}, c'_1, ..., c'_{\totalReg}, S, \texttt{st})$
% 	to the TFHE simulator to get a list of partial decryptions, $p_1, ..., p_{t - 1}$ where $S$ is a qualified set % TODO: CHECK CAN USE!
% 	and $C$ is the reservoir sampling circuit with PRF seed $s_\TFHE$ hardcoded.
% 	Note that this sets the decryption of the output of the reservoir sampling to $c'_{a_\ell}$ for $\ell \in [k]$.
% 	\item For all calls to $\texttt{reveal}(\eid, \ell, \Enc(c'_j))$ for $\ell \in K$, the simulator 
% 	simply runs the protocol as the simulator has knowledge of the openings to all of the commitments $c'_j$ used.
% \end{enumerate}

% Proof sketch
% First show that \Enc(c_j) == \Enc(c'_j)
% Then by defn, register_elect(..) == register_elect(..)
% Then, by defn Resev.output(..) == Resev.output(..)... No NEED HERE b/c TFHE does that for you
% Then, by the simulator for TFHE, we have that TFHEPartDec(..) == TFHEPartDec(..)
% Then, by defn, elect(..) == elect(..)

% One additional thing thats good is to use **all the security**
% If no hiding used, we can have a non-hiding commitment and proof still goes through


\begin{figure}
	\centering
	\fbox{
		\begin{minipage}{1\textwidth}
			\textbf{Simulator for Threshold MSLE} $\simIMSLE$:
			Initialize $\nBids = 0$. Fix some $k \in \N$ to denote the number of rounds.
			Initialize an empty set of tickets, $R$, an empty lookup of sets of parties in each election, $S$,
			a set of finished elections $\FinElect$,
			an empty lookup of election results, $E$,
			and a set $S \gets \emptyset$ to denote the set of sets of active participants in each round.
			Moreover, set a random tape for the simulator.

			The simulator knows, input for party $P_i$, $\TFHESk_i$ and $s_i$.

			\begin{itemize}
				\item $\register$ from party $P_j$
				      \begin{itemize}
					      \item $R \gets R \cup \{(i, \nBids)\}$, set $\nBids \gets \nBids + 1$.
									%  Can we use a different secret key here?
					      \item Simulate the secure channel communication with party $P_j$ if $i \neq j$.
								\item Run $\register$ honestly if $i = j$ by sending $\TFHESk_i$ to $P_i$
					            % TODO: formal
				      \end{itemize}

				\item $\registerElect(\eid, w, s_i, \TFHESk_i)$ from party $P_j$.
				      \begin{itemize}
					      \item If $j \neq i$, the simulator samples a random value $s'_j$ and follows the protocol directly using $s'_\TFHE$ and $\Enc(c'_j)$ where $c'_j = \comm(s'_j)$.
					            Store $c'_j$ in a lookup table as well as its committed message.
								\item If $j = i$, the simulator will use the commitment $c_i = \comm(s_i)$. And run $\registerElect$ as is in the protocol
				      \end{itemize}

				      % TODO: not really... we need to make sure that the wins and losses end up correct here...			

				\item $\elect(\eid, \texttt{out}, s_i, \TFHESk_i)$ where the $\texttt{out}$ is output $\bot$ or $(\texttt{outcome}, \eid, q)_i$ for all $i \in [n]$ where $q \in \set{1, ..., k, \bot}$.
				      \begin{itemize}
								\item Let $\totalReg$ be the total number of successful calls to $\registerElect(\eid, ., .)$ before the election for $\eid$
					      \item If the output is $\bot$, return $\bot$.
					      \item Let $a_1, ..., a_k$ be the ordered set of parties that won the election. I.e.\ if $q_i = \ell$ then $a_\ell = i$.
								\item The simulator then gives $\simTFHE(C, \set{\Enc(c'_j)}_{j \in [\totalReg]}, c'_1, ..., c'_{a_\ell}, S, \texttt{st})$
	to the TFHE simulator to get a list of partial decryptions, $p_1, ..., p_{t - 1}$ where $S$ is a qualified set % TODO: CHECK CAN USE!
	and $C$ is the reservoir sampling circuit with PRF seed $s_\TFHE$ hardcoded.
	Note that this sets the decryption of the output of the TFHE circuit sampling to $c'_{a_\ell}$ for $\ell \in [k]$.
					      \item Add $E_\eid = \left\{c'_{a_1}, ..., c'_{a_k}\right\}$ to $E$.
				      \end{itemize}
				\item $\texttt{reveal}(\eid, \ell, \Enc(c'_j))$ from $P_j$ and output $\bot$ or $(\texttt{result}, \eid, \ell, j)$.
				      \begin{itemize}
					      % TODO: faulty? oh yeah not here...
					      \item As the simulator has knowledge of the openings to all of the commitments $c'_j$ used, the simulator
								to honestly run $\reveal(\eid, \ell, \Enc(c'_j))$ with commitment $c'_j$ and opening $s'_j$.
								\item Note that in the semi-honest setting, the output is never $\bot$
				      \end{itemize}
			\end{itemize}
		\end{minipage}
	}
	\caption{Description of the MSLE protocol}
	\label{fig:protocolMSLE}
\end{figure}

We will now show that $\simIMSLE$ is indeed a simulator for the view of $\pi_{\text{MSLE}}$
for $\registerElect$, $\elect$, and $\reveal$.

\begin{lemma}
	\label{lemma:regElectSemiHonest}
	[$\registerElect$ is simulation secure]
	\begin{proof}
		The simulator gets as input party $i$'s secret value, $s_i$, party $i$'s $\TFHE$ share $\TFHESk_i$,
		$\eid$, and ticket number $w$. We note that if $\eid \in \FinElect$ or $(i, w) \not R$, then the simulator can simply output $\bot$ and
		is identical to the real protocol.
		Otherwise, if party $P_i$ calls $\registerElect$, then the simulator knows the input of $P_i$ and can simulate the protocol
		honestly. If party $P_j$ call $\registerElect$ for some $j \neq i$, then the view of the protocol
		is that of $(\Enc(c_j), \Enc(e_i, \texttt{coin}_i), \ResevInsertN{\eid}(\Enc(c_j, e_i, \texttt{coin}_i)))$.
		Note that this view is completely determined by $\Enc(c_j)$. Also, note that $c_j$ is drawn
		from a random distribution and is not in the view of the real protocol. Thus, 
		by semantic security of $\TFHE$ (TODO: add citation and refrence to prelis) we have that
		$\Enc(c_j) \compInd \Enc(c'_j)$ where $c'_j$ is a commitment to a random value.
	\end{proof}
\end{lemma}

\begin{lemma}[$\elect$ is simulation secure]
\begin{proof}
	If the view of the real protocol is $\bot$ because $\eid$ was already called, then the simulator can simply output $\bot$
	and is thus identical to the real protocol.
	Otherwise, note that the simulator needs to ``fix'' the output of $\elect$ to yield
	a list of commitments such that, if $a_\ell = j$ if party $j$ wins election $\ell$.
	Thus, the simulator can simply output the decryption shares $\simTFHE(C, \set{\Enc{c'_j}}_{j \in [\totalReg]}, c'_1, ..., c'_{a_\ell}, S, \texttt{st})$.
	Note that the reservoir sampling circuit is indeed simulated by $\simTFHE$
	and the decryption shares are simulated as well such that the cipher texts decrypt
	to $c'_{a_1}, ..., c'_{a_k} \compInd c_{a_1}, ..., c_{a_k}$ by \cref{lemma:regElectSemiHonest}.
\end{proof}	
\end{lemma}

\begin{lemma}[$\reveal$ is simulation secure]
	\begin{proof}
		Note that the simulator can simply run $\reveal$ honestly as the simulator has knowledge of the openings to all of the commitments $c'_j$ used
		and is thus identical to the view of the real protocol.
	\end{proof}
	
\end{lemma}


% To show simulation security, we will prove that, given a party's input and output in the ideal model,
% a simulator can simulate the distribution of the view in the protocol for the party.

% More formally we will show that for party $i$,
% \begin{equation}
% 	\label{eq:simSecReg}
% 	\simulator_i\left(s_i, r_i, \idealMSLE.\register_j\right) \compInd \texttt{view}_{\register_j}(),
% \end{equation}
% \begin{equation}
% 	\label{eq:simSecRegisterElect}
% 	\simulator_i\left(s_i, b_i, r_i, \idealMSLE.\registerElect_j\right) \compInd \texttt{view}_{\registerElect}(\eid, \Enc(c_j)),
% \end{equation}
% \begin{equation}
% 	\label{eq:simSecElect}
% 	\simulator_i\left(s_i, b_i, r_i, \idealMSLE.\elect_i\right) \compInd \texttt{view}_{\elect_i}(\eid),
% \end{equation}
% and
% \begin{equation}
% 	\label{eq:simSecReveal}
% 	\simulator_i\left(b_{a, 1}, \dots, b_{a, k}, s_i, r_i, \idealMSLE.\reveal_i\right) \compInd \texttt{view}_{\reveal_i}(),
% \end{equation}

% \begin{lemma}
% 	[$\registerElect$ is simulation secure]
% 	We will now show that \cref{eq:simSecRegisterElect} holds.
% 	\begin{proof}
% 		The simulator, $\mathcal{S}_i$, gets as input party $i$'s secret value, $s_i$, party $i$'s $\TFHE$ share $\TFHESk_i$,
% 		$\eid$, and ticket number $w$. We note that if $\eid \in \FinElect$, then the simulator can simply output $\bot$ and
% 		is identical to the real protocol.
% 	\end{proof}
% \end{lemma}

% \begin{lemma}
% 	We will first show that \cref{eq:simSecReg} is simulation secure.
% 	\begin{proof}
% 		The view of each party $i$ for $\register$ can be expressed as
% 		\begin{equation*}
% 			(r_i, s_i, c_i, C, \Enc(c_i), n)
% 		\end{equation*}
% 		We can create a simulator $\simulator_i$ that takes as input $s_i, r_i, n$ and outputs
% 		an indistinguishable view
% 		\begin{enumerate}
% 			\item Sample something? Does the simulator have access to C?
% 		\end{enumerate}
% 	\end{proof}
% \end{lemma}


% \begin{lemma}
% 	We will now show that \cref{eq:simSecElect} is simulation secure.
% 	\begin{proof}
% 		The view of each party $i$ for $\elect$ can be expressed as
% 		\begin{align*}
% 			(r_i,  s_i, \Enc(b_{1}), \dots \Enc(b_n), \Enc(r'_1), \dots, \Enc(r'_n), \\
% 			\Enc(b_{a_1}), \dots, \Enc(b_{a_k}), \sigma_1, \dots \sigma_t, b_1, \dots, b_n, y)
% 		\end{align*}
% 		where $r'_i$ is the randomness from the streaming sampler, $\sigma_1, \dots \sigma_t$ are the decryption shares,
% 		and $y \in \set{\bot, 1, ..., k}$ representing whether a party won election $1$ through $k$ or not ($\bot$).
% 		Then, we have the simulator proceed in the following manner:
% 		\begin{enumerate}
% 			\item The simulator sets a random, local tape
% 			      % TODO: question, are these c_js part of the protocol/ commitment or is it valid to do these samples?
% 			      % (I.e. they are not messages)...
% 			\item The simulator samples $c'_j$ for all $j \neq i$ where $c'_j$ is a commitment to a random value
% 			\item The simulator samples some randomness $r$ and computes $\Enc(e_1), \dots, \Enc(e_n) = \Enc(\texttt{PRF}(r, s_\texttt{TFHE}))$
% 			\item The simulator creates an encrypted priority queue $\EncPQ$ and simulates the encrypted streaming sampler for all inputs $\Enc(c'_j)$ for $j \in [i]$.
% 			\item The simulator runs the encrypted streaming sampler to get the outputs $\Enc(c'_{a_1}),... \Enc(c'_{a_k})$.
% 			\item The simulator then chooses a random, ordered subset $S \subseteq [n]$ where $|S| = k$.
% 			      If there is some $w$ such that $S_w = i$, then set $y' = w$. Otherwise, set $y' = \bot$.
% 			\item The simulator then sets the decryptions of $\Enc(c'_{a_\ell})$ to $c'_{S_\ell}$.
% 			      The simulator also creates shares $\sigma_j$ such that $\Enc(c'_{a_\ell})$ decrypts to $c'_{S_\ell}$.
% 			\item The simulator then outputs $y'$

% 		\end{enumerate}
% 		We now use a sequence of hybrids to show that the view of the real protocol is indistinguishable from
% 		that of the simulated one
% 		\begin{itemize}
% 			\item $\hybrid{0}$: The real protocol
% 			\item $\hybrid{1}$: As $\hybrid{0}$ but, for all $j \neq i$, $\Enc(c_j)$ are replaced with $\Enc(c'_j)$, where $c'_j$ is a commitment to a random value.
% 			      We can see that $\hybrid{0} \equiv \hybrid{1}$ by the hiding property of commitments.
% 			\item $\hybrid{2}$: As $\hybrid{1}$ but replace $\Enc(c_{a_1}),... \Enc(c_{a_k})$ with
% 			      $\Enc(c'_{a_1}),... \Enc(c'_{a_k})$, the output of sampling the encrypted streaming sampler with $\Enc(c'_j)$.
% 			\item $\hybrid{3}$: As $\hybrid{2}$ but replace $\Dec(\Enc(c'_{a_1})),... \Dec(\Enc(c'_{a_k}))$ with
% 			      $c'_{S_1},... c'_{S_k}$, where $S$ is the random subset chosen by the simulator.
% 			      Replace $\sigma_j$ with shares $\sigma'_j$ such that $\Enc(c'_{a_\ell})$ decrypts to $c'_{S_\ell}$.
% 			\item $\hybrid{4}$: As $\hybrid{3}$ but replace $y$ with $y'$.

% 			      % 	% TODO: better define threshold decryption as separate thing
% 			      % 	% TODO: remove separate ticket registration
% 			\item $\hybrid{5}$: The simulated protocol
% 		\end{itemize}
% 		% TODO: can I have proof be within bullet points
% 		% Question: can a simulator have access to information from prior calls? 
% 		% Ohhhhhhhhhhhhhhhh... I see output $y$ has to follow distribution as well.


% 	\end{proof}
% \end{lemma}

% \begin{lemma}
% 	We will now show that \cref{eq:simSecReveal} is simulation secure.
% 	\begin{proof}

% 	\end{proof}
% \end{lemma}

