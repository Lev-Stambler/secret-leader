\documentclass[11pt]{article}
\usepackage{Custom}
% \usepackage{algorithmic}
\usepackage{CSTheoryToolkitCMUStyle}
\usepackage{adjustbox}
\usepackage{cryptocode}
\usepackage{cleveref}

\newcommand{\myname}{Authors here}

%%%%% Section-renaming code by egreg
\makeatletter
% we use \prefix@<level> only if it is defined
\renewcommand{\@seccntformat}[1]{%
  \ifcsname prefix@#1\endcsname
    \csname prefix@#1\endcsname
  \else
    \csname the#1\endcsname\quad
  \fi}
% Now we define our homework section prefixes
\makeatother
%%%%%

\begin{document}

\title{Multiple Secret Leaders}

\author{\myname}

\date{\today}
\maketitle

\newcommand{\npar}{n}
\newcommand{\kld}{k}
\newcommand{\comm}{\texttt{comm}}
\newcommand{\Enc}{\texttt{Enc}_{\mathrm{TFHE}}}
\newcommand{\Dec}{\texttt{Dec}_{\mathrm{TFHE}}}
\newcommand{\LS}[1]{\textcolor{red}{LS: #1}}
\newcommand{\compInd}{\stackrel{\mathclap{\normalfont\mbox{c}}}{\equiv}}
\newcommand{\advers}{\mathcal{A}}
\newcommand{\simulator}{\text{Sim}}
\newcommand{\simulatorPQ}{\text{Sim}_\text{PQ}}
\newcommand{\negl}{\texttt{negl}}
\newcommand{\idealMSLE}{\mathcal{F}_{\text{MSLE}}}
\newcommand{\idealStreamSample}{\mathcal{F}_{\text{SAMPLE}}}
\newcommand{\hybrid}[1]{\mathrm{\textbf{Hyb}}_{#1}}
\newcommand{\eid}{{eid}}
\newcommand{\init}{\texttt{initialize}}
\newcommand{\register}{\texttt{register}}
\newcommand{\elect}{\texttt{elect}(\eid, S)}
\newcommand{\electReal}{\texttt{elect}(\eid, S, \Enc(c_{S_1}), ..., \Enc(c_{S_{|S|}}))}
\newcommand{\reveal}{\texttt{reveal}(\eid, \ell)}
\newcommand{\extractMin}{\texttt{ExtractMin}}
\newcommand{\getMin}{\texttt{GetMin}}
\newcommand{\ct}{\texttt{ct}}

\newcommand{\PQInsert}{\texttt{PQ.Insert}}
\newcommand{\PQExtract}{\texttt{PQ.ExtractFront}}
\newcommand{\PQFront}{\texttt{PQ.Front}}
\newcommand{\PQ}{\texttt{PQ}}

\newcommand{\TFHE}{\texttt{TFHE}}
\newcommand{\TFHESetup}{\texttt{TFHE.Setup}}
\newcommand{\TFHEEncrypt}{\texttt{TFHE.Encrypt}}
\newcommand{\TFHEEval}{\texttt{TFHE.Eval}}
\newcommand{\TFHEPartDec}{\texttt{TFHE.PartDec}}
\newcommand{\TFHEFinDec}{\texttt{TFHE.FinDec}}
\newcommand{\TFHEPk}{pk}
\newcommand{\TFHESk}{sk}

\newcommand{\EncPQInsert}{\texttt{EncPQ.Insert}}
\newcommand{\EncPQExtract}{\texttt{EncPQ.ExtractFront}}
\newcommand{\EncPQFront}{\texttt{EncPQ.Front}}
\newcommand{\EncPQ}{\texttt{EncPQ}}

\newcommand{\ResevInsert}{\texttt{Resevoir.Insert}}
\newcommand{\ResevOutput}{\texttt{Resevoir.Output}}

\algrenewcommand\algorithmicrequire{\textbf{Input:}}
\algrenewcommand\algorithmicensure{\textbf{Output:}}

% \begin{abstract}
% \end{abstract}
\section{Some Notation}
\begin{enumerate}
	\item We will have $\npar$ parties
	\item We will have $\kld$ leaders elected
	\item We will have a ``bid'' published by a user $i \in [\npar]$ be denoted as $b_i$
	\item We will denote a commitment as $\comm_i$
	\item We will denote some generic CRHF as $h$
	\item We will say $\Enc$ and $\Dec$ for TFHE encoding and decoding respectively
\end{enumerate}
% \section{Sketching Stuff Out}
% Say we want to elect $\kld$ leaders (maybe with or without repetition) secretly out of $\npar$
% parties, we can run SSLE $\kld$ times but that'd be painful. Let's not do that.

% \subsection{Simple sorting}
% \label{sec:simpleSort}
% A simple approach with runtime (assuming $\kld \leq \npar$) $O(\kld + \npar \log \npar)$
% is to use sorting. At a high level, the idea is to somehow randomly sort the $\npar$ parties 
% then elect the top $\kld$ parties in the sort.
% To do this we can use TFHE in a very similar way to the SSLE paper.

% \begin{algorithm}
% 	\begin{algorithmic}
%     \Require List of bids, $b_1, b_2, ..., b_n$
% 		\State $\alpha \gets \texttt{ROM}(b_1, b_2, ...)$
% 		\For{$i \in [n]$}
% 		\State $b'_i \gets (\Enc(r_i) - \alpha, \Enc(\comm_i))$
% 		\EndFor
% 		\State $\texttt{sorted} = [b'_{j_1}, b'_{j_2}, ...] \gets \texttt{Sort}_{\mathrm{TFHE}}([b'_1, b'_2, ...])$
% 		\State \Return first $k$ elements of $\texttt{sorted}$
% 	\end{algorithmic}
% 	\caption{Simple Sorting Evaluation}
% 	\label{alg:simpleSortEval}
% \end{algorithm}


% \begin{enumerate}
% 	\item \textbf{Setup}: Same exact setup with TFHE in SSLE. Setup TFHE and distribute shares.
% 	      Each party also has some secret $s_i$ and commitment $\comm_i = \comm(s_i)$.
% 	\item \textbf{Publish Bid}: Each party samples $r_i \leftarrow U$ and publish $b_i = (\Enc(r_i), \Enc(\comm_i))$.
% 	\item \textbf{Evaluation}: This is done by one party whom may even be an external party. See \cref{alg:simpleSortEval} for details.
% 	\item \textbf{Decoding}: At least $t$ parties publish their threshold decryptions of the return from the evaluation.
% 	\item \textbf{Proving}: For a party $i$ to prove that they are the selected leader for the $j$th slot, they prove that they know the secret to produce $\comm_i$
% 	      where $\comm_i$ is the decrypted commitment for the $i$th slot.
% \end{enumerate}

% \subsection{Distributed/ Streaming Simple Sorting}
% Assume that $n, k$ are a power of 2. We will keep a lot of the same from sorting
% but now look at selection as a sort of tournament. As a note, security is slightly reduced here as
% a participating party can know that they did not win before the final outcome of the election. I do not know if this matters.

% For completeness, we will write out each step again


% \begin{algorithm}
% 	\caption{$\texttt{PlayGame}$. Homomorphically plays a game to decide which incoming bid ``wins''}
% 	\label{alg:playGame}
% 	\begin{algorithmic}
% 		\Require Point $\alpha \in \F_q$, \\two bids, $b_1 = (\Enc(r_1), \Enc(\comm_1)), (\Enc(r_2), \Enc(\comm_2))$
% 		\Ensure Winning bid. The winning bid should be indistinguishable from a random bid to the non-players
% 		\State $\texttt{closer} \gets \Enc(r_1 - \alpha > r_2 - \alpha)$
% 		\State \Return $\texttt{closer} \cdot b_1 + (1 - \texttt{closer}) \cdot b_2$
% 	\end{algorithmic}
% \end{algorithm}



% \begin{algorithm}
% 	\caption{$\texttt{EvalStream}$. Evaluates a stream of bids as they come in.}
% 	\label{alg:distrEval}
% 	\begin{algorithmic}
% 		% TODO: how to name?
% 		% Name evalStream
% 		\Require $L$, set of $(\texttt{bid}, \texttt{level})$ pairs. New bid and level, $(b, \ell)$. Also, stop level $\ell_{\mathrm{stop}}$
% 		\Ensure New list $L$ with remaining evaluations
		
% 		\If {$\text{there is some for some }b', (b', \ell) \in L$}
% 		\State $\alpha \gets \texttt{ROM}(b, b', \ell)$
% 		\State $\overline{b} \gets \texttt{PlayGame}(b, b')$
% 		\State $L' \gets L \setminus \set{(b', \ell)}$
% 		\If $\ell_{\mathrm{stop}} = \ell - 1$
% 		\State \Return $L \bigcup \set{(\overline{b}, \ell - 1)}$
% 		\Else
% 		\State \Return $\texttt{EvalStream}(L', (\overline{b}, \ell - 1))$
% 		\EndIf
% 		\Else
% 		\State $L' \gets L \bigcup \set{(b, \ell)}$
% 		\State \Return $L'$
% 		\EndIf
% 	\end{algorithmic}
% \end{algorithm}

% \begin{enumerate}
% 	\item \textbf{Setup}: Same exact setup with TFHE in SSLE. Setup TFHE and distribute shares.
% 	      Each party also has some secret $s_i$ and commitment $\comm_i = \comm(s_i)$.
% 	\item \textbf{Publish Bid}: Each party samples $r_i \leftarrow U$ and publish $b_i = (\Enc(r_i), \Enc(\comm_i))$ each tagged with level $\log_2 n + 1$.
% 	\item \textbf{Evaluation}: This step is different than that in the simple sorting version (\cref{sec:simpleSort}).
% 	      Now, we can evaluate in a streaming fashion and even distribute evaluation (though we'll not go into details about the distributed implementation).
% 	      See \cref{alg:distrEval} for the algorithm. The idea is that we keep a list of bids and their levels;
% 	      when a new bid comes in, we greedily remove any bids from the list which we can. We can run \cref{alg:distrEval}
% 	      until there is only one bid left in the list or until all bids up to a level have been processed.
% 	      So, say that we have $k = 4$, we can run the algorithm until we are two levels away (in a tournament tree) from the final level.
% 	      More formally, when we have processed all bids up to level $\ell_{\mathrm{stop}} = \log_2 k + 2$ (and we only have bids at level $\log_2 k + 1$ in the list) we return the list and stop processing.
% 	      When there are $k$ elements in this list, all at level $\log_2 k + 1$, we are done.

% 		    \LS{I may have an off by one error here though I do not think so.}
	      
% 	\item \textbf{Decoding}: At least $t$ parties publish their threshold decryptions of the return from the evaluation
% 	\item \textbf{Proving}: For a party $i$ to prove that they are the selected leader for the $j$th slot, they prove that they know the secret to produce $\comm_i$
% 	      where $\comm_i$ is the decrypted commitment for the $i$th slot
% \end{enumerate}

% \section{Oblivious PQ}
% \newcommand{\ct}{\texttt{ct}}
% https://eprint.iacr.org/2016/046.pdf
% and the functionality that it realizes is one that produces an encrypted random ordered list of the first k bla thingies
% We will start by describing the functionality of an MPC protocol for restricted computing environments which produces some random output list.

% TODO: ideal functionality
\subsection{Oblivious Priority Queue}
We will make extensive use of an oblivious priority queue which works over FHE.

\newcommand\compInd{\stackrel{\mathclap{\normalfont\mbox{c}}}{\approx}}
\newcommand{\PQInsert}{\texttt{PQInsert}}
\newcommand{\PQFront}{\texttt{PQFront}}
\newcommand{\PQExtractFront}{\texttt{PQExtractFront}}
\newcommand{\simulator}{\text{Sim}}
\newcommand{\simulatorPQ}{\text{Sim}_\text{PQ}}
\newcommand{\PQ}{\texttt{PQ}}
\newcommand{\hybrid}[1]{\mathrm{\textbf{Hyb}}_{#1}}
% \newcommand{\simulator}{\mathcal{S}}

\begin{definition}[Oblivious Priority Queue]
	An oblivious priority queue is a priority queue which can be ``run'' by a party oblivious to the actual
	messages of the priority queue. More formally, an oblivious priority queue has functionality
	\begin{itemize}
		\item $\PQInsert(\ct, \PQ) = \PQ'$ which takes in a cipher text $\ct$ and current priority queue and outputs $\PQ'$ where 
		the ciphertext is properly inserted into the priority queue according to the underlying plaintext of $\ct$.
		\item $\PQFront(\PQ) = \ct$ which takes as input a priority queue and outputs the ciphertext $\ct$ who's plaintext is the minimum element in the queue.
		\item $\PQExtractFront(\PQ) = \ct, \PQ'$ which takes as input a priority queue and outputs the ciphertext with minimum plaintext as well as an updated queue without $\ct$.
	\end{itemize}
	We also require that for $N$ elements in $\PQ$, operation $Q \in \set{\PQInsert, \PQFront, \PQExtractFront}$,
	and input $\ct$ into $Q$ such that there is some simulator, $\simulatorPQ$ where
	% TODO: latex crypto package
	\begin{equation}
		\{\simulatorPQ(1^\lambda, N, Q, \PQ) \} \compInd	\{Q(\PQ, \ct) \}.
	\end{equation}
	In words, $\simulatorPQ$ can simulate a distribution of the output $\PQ$ independent of the input element.
	\LS{I am not quite sure about the above: definitions that I saw are for adaptive adversaries, but we are not adaptive.}
\end{definition}

\subsection{Multi SLE Algorithm}

For $n$ parties, we want to output a list of $k$ elements where each element is chosen randomly (without replacement) from a set of messages submitted by each party.
Let the function for each party denote $f_i$ and we will say that $f_i(\Enc(b_1), \Enc(b_2), ..., \Enc(b_n)) = y$ where $y \in \set{\bot, 1, ..., k}$
such that if $y=a$ then party $i$ is the leader for the $a$th round and if not $y = \bot$.

To build this functionality, we will use an oblivious streaming sampler. We can construct one as follows, closely following \cite{shi2020path} but modifying randomness sampling to use a PRF.
\begin{itemize}
	\item When an item, $\ct$, arrives as a TFHE cipher text, we will produce a random value, $\Enc(\alpha)$ by evaluating $\texttt{PRF}(k, \ct)$ in FHE.
	\item If the item is the $m$-th item and $m \leq k$, insert the item into the queue along with $\alpha$ as its label via $\PQInsert(\ct)$.
	\item If $m > k$, we will evaluate the PRF (in FHE) to get an encoded random coin which is 1 with probability $1/m$ and 0 otherwise.
	Then, run \cref{alg:ObIns} which will
	\begin{itemize}
		\item replace the smallest labeled item in the queue with the new item if the coin is 1.
		\item not replace the smallest item in the queue if the coin is 0.
	\end{itemize}
	\item At the end of the stream, we output all the items in the priority queue one by one using $\PQExtractFront$.
\end{itemize}
The simulation proof follows almost exactly as in \cite{shi2020path} but now we have to deal with the randomness from 
the PRF rather than public randomness. We can follow the proof directly but add one more hybrid where we replace the output of the PRF with a random oracle in the FHE cipher text space.
Then, we can use the PRF indistinguishably property to show that the hybrid is indistinguishable from the previous hybrid.\\

We can then implement multi secret leader election via the following algorithm:
\begin{itemize}
	\item Each party $i$ will submit a message $\Enc(b_i)$ to the oblivious streaming sampler for some secret $b_i$ which is a commitment to a secret known only by party $i$.
	\item After all parties submitted there messages, the oblivious streaming sampler will output a list of $k$ messages: $\Enc(b_{a_1}), \Enc(b_{a_2}), ..., \Enc(b_{a_k})$.
	\item Then, at least $t$ parties will submit decryption shares to decrypt $\Enc(b_{a_1}), ..., \Enc(b_{a_k})$.
	\item Each party will then build check if they one an election by seeing if their commitment is in the list of decrypted messages.
	A party can then prove that they won an election $a$ by showing that they know the opening to $b_a$.
\end{itemize}

\begin{algorithm}
	\caption{Oblivious Gated Insert}
	\label{alg:ObIns}
	\begin{algorithmic}
		\Require $\Enc(\text{coin}), \PQ, \ct_a$
		\Ensure $\PQ$'
		\State $\ct_b, \PQ' \gets \PQExtractFront(\PQ)$
		\State $\ct_{\text{new}} = \Enc(\text{coin}) \cdot \ct_a + (\Enc(\text{coin}) - 1) \cdot \ct_b$
		\State \Return $\PQInsert(ct_{\text{new}})$
	\end{algorithmic}
\end{algorithm}


\subsection{Correctness}
For correctness, we want to show that for each of the $k$ elections, a unique leader is chosen for each round
and that the probability of being selected leader is uniform out of the potential leaders.

More formally, let $E_i$ be the event in which party $i$ is elected for some election and
$E_{a, i}$ be the event which party $i$ is elected for the $a$-th election. We want 
\begin{align}
	&\Pr\left[E_i\right] = \frac{k}{n},\\
	&\Pr\left[E_{a, i} \mid E_i\right] = \frac{1}{k},\\
	&\Pr\left[E_{b, i} \mid E_{a, i}\right] = 0.
\end{align}
where $b \neq a, b \in [k]$.
\LS{I think that here we can prove the uniqueness and fairness property of multi-leader election}
\begin{theorem}[Correctness]
We have correctness...	
\begin{proof}
	By definition... should be pretty easy to show.
\end{proof}
\end{theorem}
\subsection{Simulation Security}
We will show security in a semi-honest model via simulation. We will show that each user's view of the protocol can be simulated by a polynomial-time algorithm that only has access to the user's input.
Note that view of publishing a message is the same as the view of as each user publishing a random LWE ciphertext assuming the TFHE is secure.
\\Then, upon evaluation, we need to show that the oblivious queue can be simulated.
Note that every step in the oblivious queue algorithm which does not involve $b_i$ involves manipulating
TFHE samples under a secret key unknown to the simulator. Thus, we can simulate all steps not involving $b_i$.
Every step in the oblivious queue algorithm which involves $b_i$ and some $b_j$, for $i \neq j$ requires 
a homomorphic evaluation where one of the inputs in encrypted under a secret key unknown to the simulator.
Thus, by the security of TFHE, the output of any of these steps is indistinguishable from a using a $b_j$ encoding a different message.

\subsubsection*{Proof of Simulation Security}
More formally, $\forall i \in [n]$, we can create a simulator, $\simulator_i$, which takes as input 
the user's secret value $b_i$ and publishes random TFHE encrypted message, $\Enc(b_j)$ for all $j \neq i$, for all other parties such that
\begin{equation}
	\label{eq:simulator}
	\{\simulator_i(1^\lambda, s_i, \comm_i, r_i, f_i(\Enc(b_1), ..., \Enc(b_n))\} \compInd \{\texttt{view}^\pi_1(s_i, \comm_i, r_i, y)\} 
\end{equation}
with non-negligible probability.

To achieve \cref{eq:simulator}, the simulator simply follows a simple procedure
\begin{enumerate}
	\item The simulator randomly samples $b_j$ for all $j \neq i$
	\item The simulator uses $\simulatorPQ$ to simulate the oblivious streaming sampler for all inputs $\Enc(b_j)$ for $j \neq i$.
	For input $\Enc(b_i)$, the simulator honestly runs the oblivious streaming sampler.
	\item After all $n$ parties have submitted their inputs, the simulator runs the oblivious streaming sampler to get the outputs $\Enc(b_{a_1}),... \Enc(b_{a_k})$.
	\item If the output of the oblivious streaming sampler 
	\item If $y \neq \bot$, the simulator then ``decrypts'' $\Enc(b_{a_y})$ to get $b_i$ and also pretends to decrypt the rest of the $\Enc(b_{a_j})$'s by setting the decryption to a random value.
	It $y = \bot$, then the simulator will ``decrypt'' all $\Enc(b_{a_j})$'s to random values.
	\item The simulator simply outputs the view of the protocol with the ``decryptions'' of $\Enc(b_j)$'s.
\end{enumerate}

We will now show that the above indeed is a polynomial time simulator with overwhelming probability.
\begin{lemma}
	The above algorithm satisfies the computational indistinguishably requirment of \cref{eq:simulator}.
	\begin{proof}
		First we will show that $\mathcal{S}_i$ runs in polynomial time. There are only $k + 1$ possible outcomes for $y$,
		each with inverse polynomial probability. Thus, with overwhelming probability, the simulator can sample the $b_i$'s where the outcome is $y$.

		We now show simulatability via hybrids.
		Note that by the fairness property, the simulator has a non-negligible probability of being the true winner of the election.
		Thus, we will assume that the simulator is the true winner of election $y$ if $y \neq \bot$. If $y = \bot$, then we assume that $i$ did not win any election.
		Moreover, we also have that the encoded FHE elements (where party $i$ does not know the plaintext)
		are indistinguishable from encodings of random elements. We will now construct our hybrids:
		
		$\hybrid{1}$: Consider a hybrid where we replace the FHE encoded elements, $\Enc(b_j)$ for $j \neq i$ with random elements.
		By definition of IND-CPA security, this is indistinguishable from the original hybrid. We can then propogate the output of the oblivious priority queue
		to use these random elements.

		$\hybrid{2}$: Consider a hybrid where we replace the TFHE decoding of all elements which do not equal to $y$ with a random value.
		As the $b_j$ for $i \neq i$ are random, this hybrid is indistinguishable from the previous hybrid.

		$\hybrid{3}$: Consider a hybrid where we replace the the output of the priority queue with the output from simulating the priority queue with the random FHE encoded elements.
		
		Note that $\hybrid{3}$ is indistinguishable from the original hybrid and is the same as the output of the simulator.
	\end{proof}
\end{lemma}
% \newcommand{\idealMSLE}{\mathcal{F}_{\text{MSLE}}}
\newcommand{\idealStreamSample}{\mathcal{F}_{\text{SAMPLE}}}
\newcommand{\eid}{{eid}}
\newcommand{\register}{\texttt{register}}
\newcommand{\elect}{\texttt{elect}(\eid)}
\newcommand{\reveal}{\texttt{reveal}(\eid, a)}

\section{Data Independent Streaming Sampler}
\label{sec:streaming_sampler}
Our construction makes heavy use of a data independent streaming sampler which we will define below.
The streaming sampler relies on a data independent priority queue and in turn makes use of \LS{Cite Shi} protocol for an oblivious priority queue.

\subsection*{Data Independent Queue Ideal Functionality}

\begin{figure}[ht]
	\centering
	\fbox{
		\begin{minipage}{1\textwidth}
			\textbf{Encrypted Data Independent Queue Ideal Functionality} $\idealMSLE$:
			Initialize $L$ to be an empty list
			\begin{itemize}
				\item $\texttt{Insert}(\Enc(p), \Enc(x))$ upon receiving $(\Enc(p), \Enc(x))$, place $(\Enc(p), \Enc(x))$
				into an initially empty list $L$ and broadcast $(\Enc(p), \Enc(x))$ to all parties as well as termination of insert.
				\item $y \gets \texttt{ExtractMin}()$ Delete $(\Enc(p), \Enc(x))$ with the lowest $p$ value.
				Then, set $y$ to be a re-encryption of $x$ such that $y$ is indistinguishable from all other elements in $L$.

				\item $y \gets \texttt{GetMin}()$ Get $(\Enc(p), \Enc(x))$ with the lowest $p$ value.
				Then, set $y$ to be a re-encryption of $x$ such that the encryption of $x$ and $y$ are indistinguishable.
			\end{itemize}
		\end{minipage}
	}
	\caption{Description of the MSLE functionality, heavily based on the description of SSLE in \LS{CITE Dario and CFG21}}.
	\label{fig:dataIndepQ}
\end{figure}
% https://eprint.iacr.org/2011/081.pdf
% See page 8, extend 

\subsection*{Streaming Sampler Ideal Functionality}
The ideal functionality of a data independent sampler is given by $\idealStreamSample$ (TODO: cite Elaine)
which outputs a random subset of size $k$ from a stream $S$ with a random ordering.

\begin{figure}[ht]
	\centering
	\fbox{
		\begin{minipage}{1\textwidth}
			\textbf{Encrypted Data Independent Streaming Sampler Ideal Functionality} $\idealMSLE$:
			Initialize $L$ to be an empty list
			\begin{itemize}
				\item $\texttt{Insert}(\Enc(x))$ upon receiving $\Enc(x)$, place $\Enc(x)$
				into an initially empty list $L$ 
				\item $S \gets \texttt{Sample}()$ Randomly sample an ordered and re-encrypted subset $S \subseteq L$ such that $|S| = k$, $S$ has a random ordering
				and $S_a$ is indistinguishable from all elements in $L$.
			\end{itemize}
		\end{minipage}
	}
	\caption{Description of ideal sampling functionality}.
	\label{fig:samplerIdeal}
\end{figure}
%

\subsection*{Protocol for Streaming Sampler}
This protocol is identical to that of (TODO: cite Elaine) except that the 
we replace the oblivious queue with the stronger notion of a data independent queue and public coin randomness
with randomness generated within a PRF.

\section{MSLE Protocol}
\label{sec:msle_protocol}
We use a similar notion of ideal functionality for a multi-secret leader election from the ideal
functionality of single secret leader election of \LS{CITE}.

\begin{figure}[ht]
	\centering
	\fbox{
		\begin{minipage}{1\textwidth}
			\textbf{The MSLE functionality} $\idealMSLE$:
			Initialize $E, R \gets \emptyset, \gets 0$. Fix some $k \in \N$ to denote the number of rounds. Upon receiving,
			\begin{itemize}
				\item $\register$ from party $P_i$, set $R \gets R \cup \{(i, n)\}$, broadcast $(\texttt{register}, i)$ to all parties and set $n \gets n + 1$
				\item $\elect$ $k$ leaders from all honest parties. If $|R| \geq k$ and $\eid$ was not requested before,
				randomly sample $W^{\eid} \subseteq R$ where $|W^{\eid}| = k$.
				Then, assign a random ordering to $W^\eid$ to get ordered set $E^\eid$.
				Next, send $(\texttt{outcome}, \eid, a)$ to $P_j$ for all $E^\eid_a = (j, \cdot)$
				and $(\texttt{outcome}, \eid, \bot)$ to $P_i$ if $(i, \cdot) \notin E^\eid$. Store $E \gets E \bigcup \{E^\eid\}$.
				\item $\reveal$ from $P_i$: for $E_\eid \in E$, if $i = E^\eid_a$, broadcast $(\texttt{result}, \eid, a, i)$.
				Otherwise, broadcast $(\texttt{rejected}, \eid, a, i)$.
			\end{itemize}
		\end{minipage}
	}
	\caption{Description of the MSLE functionality, heavily based on the description of SSLE in \LS{CITE Dario and CFG21}}.
	\label{fig:my_label}
\end{figure}

We wil realize MSLE via the following algorithm:
\begin{itemize}
	\item Each party $i$ will submit a message $\Enc(b_i)$ to the oblivious streaming sampler for some secret $b_i$ which is a commitment to a secret known only by party $i$.
	\item After all parties submitted there messages, the oblivious streaming sampler will output a list of $k$ messages: $\Enc(b_{a_1}), \Enc(b_{a_2}), ..., \Enc(b_{a_k})$.
	\item Then, at least $t$ parties will submit decryption shares to decrypt $\Enc(b_{a_1}), ..., \Enc(b_{a_k})$.
	\item Each party will then build check if they won an election by seeing if their commitment is in the list of decrypted messages.
	A party can then prove that they won an election $a$ by showing that they know the opening to $b_a$.
\end{itemize}

\subsection{Correctness}

\subsection{Simulation Security}
To show simulation security, we will prove that, given a party's input and output in the ideal model,
a simulator can simulate the distribution of the view in the protocol for the party. Note that because the 
protocol is deterministic, it suffices to prove simulation for only the party's output.

More formally we will show that for party $i$,
\begin{equation}
	\label{eq:simSecReg}
	\simulator_i\left(s_i, r_i, \idealMSLE.\register_i\right) \compInd \texttt{view}_{\register_i}((s_0, r_0), ..., (s_n, r_n)),
\end{equation}
\begin{equation}
	\label{eq:simSecElect}
	\simulator_i\left(s_i, b_i, r_i, \idealMSLE.\elect_i\right) \compInd \texttt{view}_{\elect_i}((s_0, b_0), ..., (s_n, b_n)),
\end{equation}
and
\begin{equation}
	\label{eq:simSecReveal}
	\simulator_i\left(b_{a, 1}, \dots, b_{a, k}, s_i, r_i, \idealMSLE.\reveal_i\right) \compInd \texttt{view}_{\reveal_i}(),
\end{equation}

\begin{lemma}
	We will first show that \cref{eq:simSecReg} is simulation secure.
	\begin{proof}
		
	\end{proof}
\end{lemma}


\begin{lemma}
	We will now show that \cref{eq:simSecElect} is simulation secure.
	\begin{proof}

	\end{proof}
\end{lemma}

\begin{lemma}
	We will now show that \cref{eq:simSecReveal} is simulation secure.
	\begin{proof}

	\end{proof}
\end{lemma}


\section{Preliminaries}
\subsection{Threshold FHE}
\cite{jain2017threshold,boneh2018threshold} defines threshold FHE encryption. For the sake of completeness, we will define it here.

\begin{definition}[TFHE \cite{jain2017threshold}]
	Let $P = \set{P_1, ..., P_N}$	 be a set of $N$ parties and $\mathbb{S}$ be a class of
	access structures on $P$. A TFHE scheme for $\mathbb{S}$ is a tuple of PPT algorithms
	$$
		(\TFHESetup, \TFHEEncrypt, \TFHEEval, \TFHEPartDec, \TFHEFinDec)
	$$ such that the following specifications are met

	\begin{itemize}
		\item $(\TFHEPk, \TFHESk_1, ..., \TFHESk_N) \gets \TFHESetup(1^\lambda, 1^s, \mathbb{A})$:
		      Takes as input a security parameter $\lambda$, a depth bound on the circuit, and a structure $\mathbb{A} \in \mathbb{S}$.
		      Outputs a public key $\TFHEPk$ and a secret key $\TFHESk_i$ for each party $P_i$.
		\item $\ct \gets \TFHEEncrypt(\TFHEPk, \mu)$: Takes as input a public key and a message $\mu \in \set{0, 1}$ and outputs a ciphertext $\ct$.
		\item $\hat{\ct} \gets \TFHEEval(C, \ct_1, ..., \ct_k)$: Takes as input a circuit $C$ of depth at most $d$ and $k$ ciphertexts $\ct_1, ..., \ct_k$.
		      Outputs a ciphertext $\hat{\ct} = C(\ct_1, ..., \ct_k)$.
		\item $p_i \gets \TFHEPartDec(\ct, \TFHESk_i)$: Takes as input a ciphertext $\ct$ and a secret key $\TFHESk_i$ and outputs a partial decryption $p_i$.
		\item $\hat\mu \gets \TFHEFinDec(B)$: Takes as input a set $B = \set{p_i}_{i \in S}$ for some $S \subseteq [N]$ and deterministically
		      outputs a message $\hat\mu \in \set{0, 1, \bot}$.
	\end{itemize}
	% Let $\mathcal{F}$ be a FHE scheme with key generation algorithm 
\end{definition}

Further we remember the definitions of evaluation correctness and simulation security as outlined in,
\cite{boneh2020single}.

\begin{definition}[Evaluation Correctness \cite{jain2017threshold}].
	We have that $\TFHE$ scheme is correct if
	for all $\lambda$, depth bounds $d$, access structure $\mathbb{A}$, circuit $C : \set{0, 1}^k \rightarrow \set{0, 1}$
	of depth at most $d$, $S \in \mathbb{A}$, and $\mu_i \in \set{0, 1}$, we have the following.
	For $(\TFHEPk, \TFHESk_1, ..., \TFHESk_N) \gets \TFHESetup(1^\lambda, 1^d, \mathbb{A})$,
	$\ct_i \gets \TFHEEncrypt(\TFHEPk, \mu_i)$ for $i \in [k]$, $\hat{\ct} \gets \TFHEEval(\TFHEPk, C, \ct_1, ..., \ct_k)$,
	\begin{equation*}
		\Pr\left[\TFHEFinDec(\TFHEPk,
			\set{\TFHEPartDec(\TFHEPk, \hat{\ct}, \TFHESk_i)}_{i \in S}) = C(\mu_1, ..., \mu_k)\right] = 1 - \negl(\lambda).
	\end{equation*}
\end{definition}

\begin{definition}[Semantic Security \cite{jain2017threshold}]
	\label{def:semanticSecurityTFHE}
	We have that a $\TFHE$ scheme satisfies semantic security for for all $\lambda$, and depth bound $d$ if the following holds. There is a stateful PPT algorithm
	$\mathcal{S} = (\mathcal{S}_1, \mathcal{S}_2)$ such that for any PPT adversary $\mathcal{A}$,
	the following experiment outputs 1 with negligible probability in $\lambda$:
	\begin{enumerate}
		\item On input $1^\lambda$ and depth $1^d$, the adversary outputs $\mathbb{A} \in \mathbb{S}$
		\item The challenger runs $(\TFHEPk, \TFHESk_1, ..., \TFHESk_N) \gets \TFHESetup(1^\lambda, 1^d, \mathbb{A})$ and provides $\TFHEPk$ to $\advers$.
		\item $\advers$ outputs a set $S \subseteq \set{P_1, ..., P_N}$ such that $S \notin \mathbb{A}$.
		\item The challenger provides $\set{\TFHESk_i}_{i \in S}$ and $\TFHEEncrypt(\TFHEPk, \mu)$ to $\advers$ where $\mu \overset{\$}{\gets} \set{0, 1}$.
		\item $\advers$ outputs a guess $\mu'$. The experiment outputs 1 if $\mu' = \mu$.
	\end{enumerate}
\end{definition}

\begin{definition}[Simulation Security \cite{jain2017threshold}]
	We say that a $\TFHE$ scheme is simulation secure if for all $\lambda$, depth bound $d$, and access structure $\mathbb{A}$
	if there exists a stateful PPT simulator, $\mathcal{S}$, such that for any PPT adversary $\mathcal{A}$,
	we have that the experiments $\exptReal(1^\lambda, 1^d)$ and $\exptSim(1^\lambda, 1^d)$ are statisically close
	as a function of $\lambda$. The experiments are defined as follows:

	% TODO: independent Latex?
	\begin{itemize}
		\item
		      $\exptReal(1^\lambda, 1^d)$: \begin{enumerate}
			      \item On input the security parameter $1^\lambda$ and depth bound $d$, the adversary outputs $\mathbb{A} \in \mathbb{S}$.
			      \item Run $\TFHESetup(1^\lambda, 1^d, \mathbb{A})$ to obtain $(\TFHEPk, \TFHESk_1, ..., \TFHESk_N)$.
						The adversary is given $\TFHEPk$.
						\item The adversary outputs a set $S \subseteq \set{P_1, ..., P_N}$ such that $S \notin \mathbb{A}$
						together with plaintest messages $\mu_1, ..., \mu_k \in \set{0, 1}$. The adversary is handed over $\{sk_i\}_{i \in S}$
						\item For each $\mu_i$, the adversary is given $\TFHEEncrypt(\TFHEPk, \mu_i) \rightarrow \ct_i$.
						\item The adversary issues a polynomial number of queries, $(S_i \subseteq \set{P_1, ..., P_N}, C_i)$.
						for circuites $C_i: \set{0, 1}^k \rightarrow \set{0, 1}$. After each query the adversary receuves for $l \in S_i$ the value
						$$
						\TFHEPartDec(\TFHEEval(C_i, \ct_1, ..., \ct_k), \TFHESk_l) \rightarrow p_l
						$$
						\item $\advers$ outputs $\texttt{out}$, the experiment's output.
						% TODO?
		      \end{enumerate}
		\item
		      $\exptSim(1^\lambda, 1^d)$: \begin{enumerate}
			      \item On input the security parameter $1^\lambda$ and depth bound $d$, the adversary outputs $\mathbb{A} \in \mathbb{S}$.
			      \item Run $\TFHESetup(1^\lambda, 1^d, \mathbb{A})$ to obtain $(\TFHEPk, \TFHESk_1, ..., \TFHESk_N)$.
						The adversary is given $\TFHEPk$.
						\item $\advers$ outputs a set $S^* \subseteq \set{P_1, ..., P_N}$ such that $S \notin \mathbb{A}$
						and plaintexts $\mu_1, ..., \mu_k \in \set{0, 1}$. The simulator is given $\TFHEPk, \mathbb{A}, S^*$ as input
						and outputs $\set{sk_i}_{i \in S^*}$ and the state $\texttt{state}$. The adversary is given $\{sk_i\}_{i \in S^*}$
						\item For each $\mu_i$, the adversay is given $\TFHEEncrypt(\TFHEPk, \mu_i) \rightarrow \ct_i$.
						\item $\advers$ issues a polynomial number of queries of the form $(S_i \subseteq \set{P_1, ..., P_N}, C_i)$
						\item for circuits $C_i: \set{0, 1}^k \rightarrow \set{0, 1}$. After each query, the simulator computes
						$$
							\simTFHE(C_i, \set{\ct_l}_{l=1}^k, C_i(\mu_1, ..., \mu_k), \texttt{state}) \rightarrow \set{p_l}_{l \in S_i}
						$$
						and sends $\set{p_l}_{l \in S_i}$ to the adversary.
						\item $\advers$ outputs $\texttt{out}$, the experiment's output.

		      \end{enumerate}
	\end{itemize}

\end{definition}

\subsection{Data Independent Priority Queue}
In this work, we will use data independent queues as studied in \cite{toft2011secure, mitchell2014data, mazloom2023efficient}.
Data independent data structures are unique as their control flow and memory access do not depend on input data (\cite{mitchell2014data}).

\begin{definition}[Word RAM model \cite{mitchell2014data}]
	In the word RAM model, the RAM has a constant number of public and secret registers and can perform arbitrary operations on a constant number of registers in constant time.
\end{definition}

% TODO: slightly modified definition by me. Check with Mark and come back to it
\begin{definition}[Data Independent Data Structure \cite{mitchell2014data}]
	In the word RAM model, a data independent data structure is a collection of algorithms where
	all the algorithms uses RAM such that the RAM can only set its	control flow based on registers that are public.
\end{definition}

Data independent queues are especially useful as they allow for efficient computation within MPC and FHE as control flow is not dependent
on underlying ciphertexts data. We use a data independent queue as outlined in \cite{mazloom2023efficient}
which allows for
\begin{itemize}
	\item $\PQInsert$: Inserts a tag and value, $(p, x)$ into $\PQ$ according to the tag's priority.
	\item $\PQExtract$: Removes and returns the $(p, y)$ with highest tag priority.
	\item $\PQFront$: Returns the $(p, y)$ with highest tag priority without removing the element.
\end{itemize}
Moreover, we note that the order is stable. I.e.\ the first inserted among equal tagged elements has a higher priority.


\subsection{Resevoir Sampling}
Resevoir sampling is an online algorithm which allows for randomly selecting
$k$ elements from a stream of $n$ elements while using $\tilde{O}(k)$ space.
Algorithm R (\cite{vitter1985random}) is a simple algorithm which relies on a priority queue with interface:
\begin{itemize}
	\item $\ResevInit(k)$ initialize the resevoir sampling algorithm and data structure
	\item $\ResevInsert(\mu_i, e_i, \texttt{coin}_i)$ where $\mu_i$ is $i$-th item, $e_i$ is independentally sampled randomness,
	      and $\texttt{coin}_i$ is a random coin with probability $1/m$ of equaling 1.
	      \begin{itemize}
		      \item If $i \leq k$, insert the item into the queue along with $e_i$ as its tag via $\PQInsert(e_i, \mu_i)$.
		      \item If $i > k$ and $\texttt{coin}_i = 1$, replace the smallest labeled item in the queue with the new item if the coin is 1.
		      \item If $i > k$ and $\texttt{coin}_i = 0$, do nothing.
	      \end{itemize}

	\item  $\mu_{a_1}, \mu_{a_2}, ..., \mu_{a_k} \gets \ResevOutput()$ where $a_1, ..., a_k$ are a uniformly random ordered susbset of $[n]$
	      \begin{itemize}
		      \item Call $\PQExtract$ $k$ times setting $\mu_{a_\ell}$ to the $\ell$-th call to $\PQExtract$ where $\ell \in [k]$.
	      \end{itemize}
\end{itemize}


\section{TFHE MSLE Protocol}
\label{sec:msle_protocol}
We use a similar notion of ideal functionality for a multi-secret leader election from the ideal
functionality of single secret leader election of \cite{catalano2022adaptively}
except that we add a $\registerElect$ phase for each election.

\newcommand{\FinElect}{\mathcal{E}}
\newcommand{\nBids}{b}
\begin{figure}[ht]
	\centering
	\fbox{
		\begin{minipage}{1\textwidth}
			\textbf{The MSLE functionality} $\idealMSLE$:
			Initialize $E, R \gets \emptyset, \nBids \gets 0$. Initialize $S$ to denote the set of sets
			of active participants in each round.
			Set $\FinElect \gets \emptyset$ to denote the set of finished elections.
			Fix some $k \in \N$ to denote the number of rounds. Upon receiving,
			\begin{itemize}
				\item $\register$ from party $P_i$, set $R \gets R \cup \{(i, b)\}$, broadcast $(\texttt{register}, i)$ to all parties and set $\nBids \gets \nBids + 1$
				\item $\registerElect(\eid, w)$ from party $P_i$.
				      If $\eid \in \FinElect$ send $\bot$ to $P_i$ and do nothing.
				      If $(i, w) \notin R$ or $(i, w) \in S_\eid$, send $\bot$ to $P_i$ and do nothing.
				      If $S_\eid$ is not defined, set $S_\eid \gets \{(i, w)\}$.
				      Otherwise, set $S_\eid \gets S_\eid \cup \{(i, w)\}$.
				\item $\elect(\eid)$ Elect $k$ leaders from $S_\eid \subseteq R$ parties.
				      If $|S| \geq k$ and $\eid \notin \FinElect$,
				      randomly sample $W^{\eid} \subseteq S_\eid$ where $|W^{\eid}| = k$.
				      Then, assign a random ordering to $W^\eid$ to get ordered set $E^\eid$.
				      Next, send $(\texttt{outcome}, \eid, a)$ to $P_j$ for all $E^\eid_a = (j, \cdot)$
				      and $(\texttt{outcome}, \eid, \bot)$ to $P_i$ if $(i, \cdot) \notin E^\eid$. Store $E \gets E \bigcup \{E^\eid\}$.
				      Set $\FinElect \gets \FinElect \cup \{\eid\}$.
				\item $\reveal$ from $P_i$: for $E_\eid \in E$, if $i = E^\eid_\ell$, broadcast $(\texttt{result}, \eid, \ell, i)$.
				      Otherwise, broadcast $(\texttt{rejected}, \eid, \ell, i)$.
			\end{itemize}
		\end{minipage}
	}
	\caption{Description of the Multi Secret Leader Election functionality}.
	\label{fig:my_label}
\end{figure}

\newcommand{\resSampling}{\mathcal{R}}

\begin{figure}
	\centering
	\fbox{
		\begin{minipage}{1\textwidth}
			\textbf{The MSLE Protocol} $\pi_{\text{MSLE}}$:
			Initialize $\nBids = 0$. Fix some $k \in \N$ to denote the number of rounds.
			Initialize an empty set of tickets, $R$.
			Initialize an empty lookup of sets of parties in each election, $S$.
			Initialize an empty lookup of reservoir sampling data structures $\resSampling$ and a set of finished elections $\FinElect$
			Initialize an empty lookup of election results, $E$.

			Initialize $S \gets \emptyset$ to denote the set of sets of active participants in each round.

			\begin{itemize}
				\item $\init$
				      \begin{itemize}
					      \item Set $n = 0$
					      \item Sample a random $\TFHE$ secret key, public key pair $\TFHESk, \TFHEPk$ and publish $\TFHEPk$.
					      \item Sample some secret $s_{\texttt{TFHE}}$ and publish $\Enc(s_{\texttt{TFHE}})$.
					            This will be the key to the PRF
				      \end{itemize}

				\item $\register$ from party $P_i$
				      \begin{itemize}
					      %  TODO: not within definition of TFHE, have to change up definition :((
					      %  TODO: formalize secure channel?
					      \item If party $P_i$ does not already have a TFHE share, create a TFHE share, $\TFHESk_i$, for $P_i$
					            and send the share over a secure channel to $P_i$.

					      \item $R \gets R \cup \{(i, \nBids)\}$, set $\nBids \gets \nBids + 1$.
				      \end{itemize}
				      % \item Add $\Enc(c_i)$ to $C$.
				      % \item $n \gets n + 1$

				\item $\registerElect(\eid, \Enc(c_i), w)$ from party $P_i$ where $c_i = \comm(s_i)$ for a randomly sampled secret $s_i$.
				      \begin{itemize}
					      \item If $\eid \in \FinElect$, then send $\bot$ to $P_i$ and do nothing.
					      \item If $(i, w) \in S_\eid$ or $(i, w) \notin R$, send $\bot$ and do nothing.
					      \item Otherwise, if $\resSampling_\eid \notin \resSampling$, $\resSampling_\eid \gets \ResevInit(k)$
					      \item The CRS will be used to run a PRF to get $\Enc(e_i), \Enc(\texttt{coin}_i) = \Enc(\texttt{PRF}(c_i, s_\texttt{TFHE}))$ for $i \in S$
					      \item $\Enc(c_i)$ will be fed to the encrypted streaming sampler along with randomness via
					            $\ResevInsertN{\eid}(\Enc(c_i), \Enc(e_i), \Enc(\texttt{coin}_i))$.
				      \end{itemize}

				\item $\elect(\eid)$
				      \begin{itemize}
					      \item If $\eid \in \FinElect$, then return $\bot$ and do nothing.
					      \item The encrypted streaming sampler will output a list of $k$ messages via calling $\ResevOutputN{\eid}()$ $k$ times: $\Enc(c_{a_1}), \Enc(c_{a_2}), ..., \Enc(c_{a_k})$.
					            %  TODO: formal with TFHE decryption shares
					      \item Then, at least $t$ parties will submit decryption shares, $p_i = \TFHEPartDec(\Enc(c_{a_1}), ..., \Enc(c_{a_2}))$ to get $c_{a_1}, ..., c_{a_k} = \TFHEFinDec(\{p_i\})$.
					      \item Add $E_\eid = \left\{c_{a_1}, ..., c_{a_k}\right\}$ to $E$.
				      \end{itemize}
				\item $\texttt{reveal}(\eid, \ell, \Enc(c'_i))$ from $P_i$
				      \begin{itemize}
					      %  TODO: formal with commitments...
					      \item $P_i$ submits a proof that they know the opening to $c_{a_\ell}$.
					            If this proof verifies, send out $(\texttt{result}, \eid, \ell, i)$ to all parties.
					            Otherwise, send out $(\texttt{rejected}, \eid, \ell, i)$ to all parties.
				      \end{itemize}
			\end{itemize}
		\end{minipage}
	}
	\caption{Description of the MSLE protocol}
	\label{fig:protocolMSLE}
\end{figure}


\subsection{Semi Honest Simulation Security}


We can show semi-honest simulation security by showing that the view of each party $i$ in the real protocol
can be simulated throughout the course of one election and then that this simulation can be extended for a polynomial number of elections.

\newcommand{\totalReg}{\overline{n}}
\newcommand{\totalRev}{\overline{k}}
% Let $\totalReg$ be the total number of calls to $\registerElect(\eid, ., .)$ before the election for $\eid$, $\elect(\eid)$
% is called and that $k \leq \totalReg \leq n$ as if $\totalReg < k$, then $\elect$ returns $\bot$ and is trivial to simulate.
% Further, let $\totalRev$ be the total number of calls to $\texttt{reveal}(\eid, ., .)$ which did not reject, again we have that $\totalRev \leq k$.
% We will now show that a simulator can simulate the view of each party $i$ after $\totalReg$ calls to $\registerElect(\eid, .)$, a call to $\elect(\eid)$, and $\totalRev$ calls to $\reveal(\eid, .)$
% for a fixed $\eid$.

\newcommand{\simIMSLE}{\texttt{Sim}_i}
% The simulator, $\simIMSLE$, will proceed as follows with input $s_i, \TFHESk_i, \eid, w$ and output of election
% $(\texttt{outcome}, \eid, q)_i$ for $i \in [\totalReg]$ and $(\texttt{result}, \eid, \ell, j)$ for $\ell \in K$
% where $K \subseteq [k]$ and $|K| = \totalRev$:

% \begin{enumerate}
% 	\item $\simIMSLE$ sets a random, local tape
% 	\item The simulator samples $c'_j$ for all $j \neq i$ where $c'_j$ is a commitment to a random value.
% 				The simulator also computes $c_i = \comm(s_i)$.
% 	\item The simulator runs $\registerElect(\eid, \Enc(c'_j), w_j)$ for $i \in [\totalReg]$ as in the protocol.
% 	\item The simulator runs the first two steps of $\elect(\eid)$, having the reservoir sample output $\Enc(c'_{a_1}), ..., \Enc(c'_{a_k})$.
% 	The simulator then gives $\simTFHE(C, \set{\Enc{c'_j}}_{j \in [\totalReg]}, c'_1, ..., c'_{\totalReg}, S, \texttt{st})$
% 	to the TFHE simulator to get a list of partial decryptions, $p_1, ..., p_{t - 1}$ where $S$ is a qualified set % TODO: CHECK CAN USE!
% 	and $C$ is the reservoir sampling circuit with PRF seed $s_\TFHE$ hardcoded.
% 	Note that this sets the decryption of the output of the reservoir sampling to $c'_{a_\ell}$ for $\ell \in [k]$.
% 	\item For all calls to $\texttt{reveal}(\eid, \ell, \Enc(c'_j))$ for $\ell \in K$, the simulator 
% 	simply runs the protocol as the simulator has knowledge of the openings to all of the commitments $c'_j$ used.
% \end{enumerate}

% Proof sketch
% First show that \Enc(c_j) == \Enc(c'_j)
% Then by defn, register_elect(..) == register_elect(..)
% Then, by defn Resev.output(..) == Resev.output(..)... No NEED HERE b/c TFHE does that for you
% Then, by the simulator for TFHE, we have that TFHEPartDec(..) == TFHEPartDec(..)
% Then, by defn, elect(..) == elect(..)

% One additional thing thats good is to use **all the security**
% If no hiding used, we can have a non-hiding commitment and proof still goes through


\begin{figure}
	\centering
	\fbox{
		\begin{minipage}{1\textwidth}
			\textbf{Simulator for Threshold MSLE} $\simIMSLE$:
			Initialize $\nBids = 0$. Fix some $k \in \N$ to denote the number of rounds.
			Initialize an empty set of tickets, $R$, an empty lookup of sets of parties in each election, $S$,
			a set of finished elections $\FinElect$,
			an empty lookup of election results, $E$,
			and a set $S \gets \emptyset$ to denote the set of sets of active participants in each round.
			Moreover, set a random tape for the simulator.

			The simulator knows, input for party $P_i$, $\TFHESk_i$ and $s_i$.

			\begin{itemize}
				\item $\register$ from party $P_j$
				      \begin{itemize}
					      \item $R \gets R \cup \{(i, \nBids)\}$, set $\nBids \gets \nBids + 1$.
					            %  Can we use a different secret key here?
					      \item Simulate the secure channel communication with party $P_j$ if $i \neq j$.
					      \item Run $\register$ honestly if $i = j$ by sending $\TFHESk_i$ to $P_i$
					            % TODO: formal
				      \end{itemize}

				\item $\registerElect(\eid, w, s_i, \TFHESk_i)$ from party $P_j$.
				      \begin{itemize}
					      \item If $j \neq i$, the simulator samples a random value $s'_j$ and follows the protocol directly using $s'_\TFHE$ and $\Enc(c'_j)$ where $c'_j = \comm(s'_j)$.
					            Store $c'_j$ in a lookup table as well as its committed message.
					      \item If $j = i$, the simulator will use the commitment $c_i = \comm(s_i)$. And run $\registerElect$ as is in the protocol
				      \end{itemize}

				      % TODO: not really... we need to make sure that the wins and losses end up correct here...			

				\item $\elect(\eid, \texttt{out}, s_i, \TFHESk_i)$ where the $\texttt{out}$ is output $\bot$ or $(\texttt{outcome}, \eid, q)_i$ for all $i \in [n]$ where $q \in \set{1, ..., k, \bot}$.
				      \begin{itemize}
					      \item Let $\totalReg$ be the total number of successful calls to $\registerElect(\eid, ., .)$ before the election for $\eid$
					      \item If the output is $\bot$, return $\bot$.
					      \item Let $a'_1, ..., a'_k$ be the ordered set of parties that won the election. I.e.\ if $q_i = \ell$ then $a'_\ell = i$.
					      \item The simulator then gives $\simTFHE(C, \set{\Enc(c'_j)}_{j \in [\totalReg]}, c'_{a'_1}, ..., c'_{a'_\ell}, S, \texttt{st})$
					            to the TFHE simulator to get a list of partial decryptions, $p_1, ..., p_{t}$ where $S$ is a qualified set % TODO: CHECK CAN USE!
					            and $C$ is the reservoir sampling circuit with PRF seed $s_\TFHE$ hardcoded.
					            Note that this sets the decryption of the output of the TFHE circuit sampling to $c'_{a'_\ell}$ for $\ell \in [k]$.
					      \item Add $E_\eid = \left\{c'_{a'_1}, ..., c'_{a'_k}\right\}$ to $E$.
				      \end{itemize}
				\item $\texttt{reveal}(\eid, \ell, \Enc(c'_j))$ from $P_j$ and output $\bot$ or $(\texttt{result}, \eid, \ell, j)$.
				      \begin{itemize}
					      % TODO: faulty? oh yeah not here...
					      \item As the simulator has knowledge of the openings to all of the commitments $c'_j$ used, the simulator
					            to honestly run $\reveal(\eid, \ell, \Enc(c'_j))$ with commitment $c'_j$ and opening $s'_j$.
					      \item Note that in the semi-honest setting, the output is never $\bot$
				      \end{itemize}
			\end{itemize}
		\end{minipage}
	}
	\caption{Description of the MSLE protocol}
	\label{fig:protocolMSLE}
\end{figure}

We will now show that $\simIMSLE$ is indeed a simulator for the view of $\pi_{\text{MSLE}}$
for $\registerElect$, $\elect$, and $\reveal$.

\begin{lemma}
	[$\registerElect$ is simulation secure]
	\label{lemma:regElectSemiHonest}
	\begin{proof}
		The simulator gets as input party $i$'s secret value, $s_i$, party $i$'s $\TFHE$ share $\TFHESk_i$,
		$\eid$, and ticket number $w$. We note that if $\eid \in \FinElect$ or $(i, w) \notin R$, then the simulator can simply output $\bot$ and
		is identical to the real protocol.
		Otherwise, if party $P_i$ calls $\registerElect$, then the simulator knows the input of $P_i$ and can simulate the protocol
		honestly. If party $P_j$ call $\registerElect$ for some $j \neq i$, then the view of the protocol
		is that of $$(\Enc(c_j), \Enc(e_i, \texttt{coin}_i), \ResevInsertN{\eid}(\Enc(c_j, e_i, \texttt{coin}_i))).$$
		Note that this view is completely determined by $\Enc(c_j)$. Also, note that $c_j$ is drawn
		from a random distribution and is not in the view of the real protocol. Thus,
		by semantic security of $\TFHE$ (\cref{def:semanticSecurityTFHE}) we have that
		$\Enc(c_j) \compInd \Enc(c'_j)$ where $c'_j$ is a commitment to a random value.
	\end{proof}
\end{lemma}

\begin{lemma}[$\elect$ is simulation secure]
	\begin{proof}
		If the view of the real protocol is $\bot$ because $\eid$ was already called, then the simulator can simply output $\bot$
		and is thus identical to the real protocol.
		Otherwise, note that the view of the protocol is
		$$
			(c_{a_1}, ..., c_{a_k}, p_1, ..., p_t, \texttt{view}(C(\Enc(c_{a_1}, e_1, \texttt{coin}_1), ..., \Enc(c_{a_k}, e_k, \texttt{coin}_k) ))).
		$$
		We will now show that simulator can simulate the above view.	
		The simulator needs to ``fix'' the output of $\elect$ to yield
		a list of commitments such that, if $a'_\ell = j$ if party $j$ wins election $\ell$.
		The TFHE simulator can simply output the decryption shares, $p_1, ... p_t$ using 
		$$\simTFHE(C, \set{\Enc(c'_j, e_j, \texttt{coin}_j)}_{j \in [\totalReg]}, c'_1, ..., c'_{a_\ell}, S, \texttt{st}).$$
		Note that the reservoir sampling circuit is indeed simulated by $\simTFHE$
		and the decryption shares are simulated as well such that the cipher texts decrypt
		to $c'_{a'_1}, ..., c'_{a'_k}$.
		We also have that, by the ideal functionality of $\elect$, $\set{(a'_\ell, .)}_{\ell \in [k]}$ is a randomly chosen
		ordered subset from $S_\eid$. We can note that, by the correctness of reservoir sampling,
		the output of $\ResevOutputN{\eid}()$ in the protocol is a randomly chosen ordered subset from $S_\eid$.
		We then have that the distribution of the simulator's $\set{(a'_\ell, .)}_{\ell \in [k]} \equiv \set{(a_\ell, .)}_{\ell \in [k]}$
		where $a_\ell$ is the party that won election $\ell$ in the protocol.
		We then have that, by the simulation security of $\registerElect$ (\cref{lemma:regElectSemiHonest}), $c'_{a'_1}, ..., c'_{a'_k} \compInd c_{a_1}, ..., c_{a_k}$ .
	\end{proof}
\end{lemma}

\begin{lemma}[$\reveal$ is simulation secure]
	\begin{proof}
		Note that the simulator can simply run $\reveal$ honestly as the simulator has knowledge of the openings to all of the commitments $c'_j$ used
		and thus has an identical view to that of the real protocol.
	\end{proof}
\end{lemma}


% To show simulation security, we will prove that, given a party's input and output in the ideal model,
% a simulator can simulate the distribution of the view in the protocol for the party.

% More formally we will show that for party $i$,
% \begin{equation}
% 	\label{eq:simSecReg}
% 	\simulator_i\left(s_i, r_i, \idealMSLE.\register_j\right) \compInd \texttt{view}_{\register_j}(),
% \end{equation}
% \begin{equation}
% 	\label{eq:simSecRegisterElect}
% 	\simulator_i\left(s_i, b_i, r_i, \idealMSLE.\registerElect_j\right) \compInd \texttt{view}_{\registerElect}(\eid, \Enc(c_j)),
% \end{equation}
% \begin{equation}
% 	\label{eq:simSecElect}
% 	\simulator_i\left(s_i, b_i, r_i, \idealMSLE.\elect_i\right) \compInd \texttt{view}_{\elect_i}(\eid),
% \end{equation}
% and
% \begin{equation}
% 	\label{eq:simSecReveal}
% 	\simulator_i\left(b_{a, 1}, \dots, b_{a, k}, s_i, r_i, \idealMSLE.\reveal_i\right) \compInd \texttt{view}_{\reveal_i}(),
% \end{equation}

% \begin{lemma}
% 	[$\registerElect$ is simulation secure]
% 	We will now show that \cref{eq:simSecRegisterElect} holds.
% 	\begin{proof}
% 		The simulator, $\mathcal{S}_i$, gets as input party $i$'s secret value, $s_i$, party $i$'s $\TFHE$ share $\TFHESk_i$,
% 		$\eid$, and ticket number $w$. We note that if $\eid \in \FinElect$, then the simulator can simply output $\bot$ and
% 		is identical to the real protocol.
% 	\end{proof}
% \end{lemma}

% \begin{lemma}
% 	We will first show that \cref{eq:simSecReg} is simulation secure.
% 	\begin{proof}
% 		The view of each party $i$ for $\register$ can be expressed as
% 		\begin{equation*}
% 			(r_i, s_i, c_i, C, \Enc(c_i), n)
% 		\end{equation*}
% 		We can create a simulator $\simulator_i$ that takes as input $s_i, r_i, n$ and outputs
% 		an indistinguishable view
% 		\begin{enumerate}
% 			\item Sample something? Does the simulator have access to C?
% 		\end{enumerate}
% 	\end{proof}
% \end{lemma}


% \begin{lemma}
% 	We will now show that \cref{eq:simSecElect} is simulation secure.
% 	\begin{proof}
% 		The view of each party $i$ for $\elect$ can be expressed as
% 		\begin{align*}
% 			(r_i,  s_i, \Enc(b_{1}), \dots \Enc(b_n), \Enc(r'_1), \dots, \Enc(r'_n), \\
% 			\Enc(b_{a_1}), \dots, \Enc(b_{a_k}), \sigma_1, \dots \sigma_t, b_1, \dots, b_n, y)
% 		\end{align*}
% 		where $r'_i$ is the randomness from the streaming sampler, $\sigma_1, \dots \sigma_t$ are the decryption shares,
% 		and $y \in \set{\bot, 1, ..., k}$ representing whether a party won election $1$ through $k$ or not ($\bot$).
% 		Then, we have the simulator proceed in the following manner:
% 		\begin{enumerate}
% 			\item The simulator sets a random, local tape
% 			      % TODO: question, are these c_js part of the protocol/ commitment or is it valid to do these samples?
% 			      % (I.e. they are not messages)...
% 			\item The simulator samples $c'_j$ for all $j \neq i$ where $c'_j$ is a commitment to a random value
% 			\item The simulator samples some randomness $r$ and computes $\Enc(e_1), \dots, \Enc(e_n) = \Enc(\texttt{PRF}(r, s_\texttt{TFHE}))$
% 			\item The simulator creates an encrypted priority queue $\EncPQ$ and simulates the encrypted streaming sampler for all inputs $\Enc(c'_j)$ for $j \in [i]$.
% 			\item The simulator runs the encrypted streaming sampler to get the outputs $\Enc(c'_{a_1}),... \Enc(c'_{a_k})$.
% 			\item The simulator then chooses a random, ordered subset $S \subseteq [n]$ where $|S| = k$.
% 			      If there is some $w$ such that $S_w = i$, then set $y' = w$. Otherwise, set $y' = \bot$.
% 			\item The simulator then sets the decryptions of $\Enc(c'_{a_\ell})$ to $c'_{S_\ell}$.
% 			      The simulator also creates shares $\sigma_j$ such that $\Enc(c'_{a_\ell})$ decrypts to $c'_{S_\ell}$.
% 			\item The simulator then outputs $y'$

% 		\end{enumerate}
% 		We now use a sequence of hybrids to show that the view of the real protocol is indistinguishable from
% 		that of the simulated one
% 		\begin{itemize}
% 			\item $\hybrid{0}$: The real protocol
% 			\item $\hybrid{1}$: As $\hybrid{0}$ but, for all $j \neq i$, $\Enc(c_j)$ are replaced with $\Enc(c'_j)$, where $c'_j$ is a commitment to a random value.
% 			      We can see that $\hybrid{0} \equiv \hybrid{1}$ by the hiding property of commitments.
% 			\item $\hybrid{2}$: As $\hybrid{1}$ but replace $\Enc(c_{a_1}),... \Enc(c_{a_k})$ with
% 			      $\Enc(c'_{a_1}),... \Enc(c'_{a_k})$, the output of sampling the encrypted streaming sampler with $\Enc(c'_j)$.
% 			\item $\hybrid{3}$: As $\hybrid{2}$ but replace $\Dec(\Enc(c'_{a_1})),... \Dec(\Enc(c'_{a_k}))$ with
% 			      $c'_{S_1},... c'_{S_k}$, where $S$ is the random subset chosen by the simulator.
% 			      Replace $\sigma_j$ with shares $\sigma'_j$ such that $\Enc(c'_{a_\ell})$ decrypts to $c'_{S_\ell}$.
% 			\item $\hybrid{4}$: As $\hybrid{3}$ but replace $y$ with $y'$.

% 			      % 	% TODO: better define threshold decryption as separate thing
% 			      % 	% TODO: remove separate ticket registration
% 			\item $\hybrid{5}$: The simulated protocol
% 		\end{itemize}
% 		% TODO: can I have proof be within bullet points
% 		% Question: can a simulator have access to information from prior calls? 
% 		% Ohhhhhhhhhhhhhhhh... I see output $y$ has to follow distribution as well.


% 	\end{proof}
% \end{lemma}

% \begin{lemma}
% 	We will now show that \cref{eq:simSecReveal} is simulation secure.
% 	\begin{proof}

% 	\end{proof}
% \end{lemma}




\bibliographystyle{alpha}
\bibliography{bib/ref}

\end{document}

