\documentclass[11pt]{article}
\usepackage{Custom}
% \usepackage{algorithmic}
\usepackage{CSTheoryToolkitCMUStyle}
\usepackage{adjustbox}
\usepackage{cryptocode}
\usepackage{cleveref}

\newcommand{\myname}{Authors here}

%%%%% Section-renaming code by egreg
\makeatletter
% we use \prefix@<level> only if it is defined
\renewcommand{\@seccntformat}[1]{%
  \ifcsname prefix@#1\endcsname
    \csname prefix@#1\endcsname
  \else
    \csname the#1\endcsname\quad
  \fi}
% Now we define our homework section prefixes
\makeatother
%%%%%

\begin{document}

\title{Multiple Secret Leaders}

\author{\myname}

\date{\today}
\maketitle

\newcommand{\npar}{n}
\newcommand{\kld}{k}
\newcommand{\comm}{\texttt{comm}}
\newcommand{\Enc}{\texttt{Enc}_{\mathrm{TFHE}}}
\newcommand{\Dec}{\texttt{Dec}_{\mathrm{TFHE}}}
\newcommand{\LS}[1]{\textcolor{red}{LS: #1}}
\newcommand{\compInd}{\stackrel{\mathclap{\normalfont\mbox{c}}}{\approx}}
\newcommand{\advers}{\mathcal{A}}
\newcommand{\simulator}{\text{Sim}}
\newcommand{\simulatorPQ}{\text{Sim}_\text{PQ}}
\newcommand{\negl}{\texttt{negl}}

\algrenewcommand\algorithmicrequire{\textbf{Input:}}
\algrenewcommand\algorithmicensure{\textbf{Output:}}

% \begin{abstract}
% \end{abstract}
\section{Some Notation}
\begin{enumerate}
	\item We will have $\npar$ parties
	\item We will have $\kld$ leaders elected
	\item We will have a ``bid'' published by a user $i \in [\npar]$ be denoted as $b_i$
	\item We will denote a commitment as $\comm_i$
	\item We will denote some generic CRHF as $h$
	\item We will say $\Enc$ and $\Dec$ for TFHE encoding and decoding respectively
\end{enumerate}
\section{Sketching Stuff Out}
Say we want to elect $\kld$ leaders (maybe with or without repetition) secretly out of $\npar$
parties, we can run SSLE $\kld$ times but that'd be painful. Let's not do that.

\subsection{Simple sorting}
\label{sec:simpleSort}
A simple approach with runtime (assuming $\kld \leq \npar$) $O(\kld + \npar \log \npar)$
is to use sorting. At a high level, the idea is to somehow randomly sort the $\npar$ parties 
then elect the top $\kld$ parties in the sort.
To do this we can use TFHE in a very similar way to the SSLE paper.

\begin{algorithm}
	\begin{algorithmic}
    \Require List of bids, $b_1, b_2, ..., b_n$
		\State $\alpha \gets \texttt{ROM}(b_1, b_2, ...)$
		\For{$i \in [n]$}
		\State $b'_i \gets (\Enc(r_i) - \alpha, \Enc(\comm_i))$
		\EndFor
		\State $\texttt{sorted} = [b'_{j_1}, b'_{j_2}, ...] \gets \texttt{Sort}_{\mathrm{TFHE}}([b'_1, b'_2, ...])$
		\State \Return first $k$ elements of $\texttt{sorted}$
	\end{algorithmic}
	\caption{Simple Sorting Evaluation}
	\label{alg:simpleSortEval}
\end{algorithm}


\begin{enumerate}
	\item \textbf{Setup}: Same exact setup with TFHE in SSLE. Setup TFHE and distribute shares.
	      Each party also has some secret $s_i$ and commitment $\comm_i = \comm(s_i)$.
	\item \textbf{Publish Bid}: Each party samples $r_i \leftarrow U$ and publish $b_i = (\Enc(r_i), \Enc(\comm_i))$.
	\item \textbf{Evaluation}: This is done by one party whom may even be an external party. See \cref{alg:simpleSortEval} for details.
	\item \textbf{Decoding}: At least $t$ parties publish their threshold decryptions of the return from the evaluation.
	\item \textbf{Proving}: For a party $i$ to prove that they are the selected leader for the $j$th slot, they prove that they know the secret to produce $\comm_i$
	      where $\comm_i$ is the decrypted commitment for the $i$th slot.
\end{enumerate}

\subsection{Distributed/ Streaming Simple Sorting}
Assume that $n, k$ are a power of 2. We will keep a lot of the same from sorting
but now look at selection as a sort of tournament. As a note, security is slightly reduced here as
a participating party can know that they did not win before the final outcome of the election. I do not know if this matters.

For completeness, we will write out each step again


\begin{algorithm}
	\caption{$\texttt{PlayGame}$. Homomorphically plays a game to decide which incoming bid ``wins''}
	\label{alg:playGame}
	\begin{algorithmic}
		\Require Point $\alpha \in \F_q$, \\two bids, $b_1 = (\Enc(r_1), \Enc(\comm_1)), (\Enc(r_2), \Enc(\comm_2))$
		\Ensure Winning bid. The winning bid should be indistinguishable from a random bid to the non-players
		\State $\texttt{closer} \gets \Enc(r_1 - \alpha > r_2 - \alpha)$
		\State \Return $\texttt{closer} \cdot b_1 + (1 - \texttt{closer}) \cdot b_2$
	\end{algorithmic}
\end{algorithm}



\begin{algorithm}
	\caption{$\texttt{EvalStream}$. Evaluates a stream of bids as they come in.}
	\label{alg:distrEval}
	\begin{algorithmic}
		% TODO: how to name?
		% Name evalStream
		\Require $L$, set of $(\texttt{bid}, \texttt{level})$ pairs. New bid and level, $(b, \ell)$. Also, stop level $\ell_{\mathrm{stop}}$
		\Ensure New list $L$ with remaining evaluations
		
		\If {$\text{there is some for some }b', (b', \ell) \in L$}
		\State $\alpha \gets \texttt{ROM}(b, b', \ell)$
		\State $\overline{b} \gets \texttt{PlayGame}(b, b')$
		\State $L' \gets L \setminus \set{(b', \ell)}$
		\If $\ell_{\mathrm{stop}} = \ell - 1$
		\State \Return $L \bigcup \set{(\overline{b}, \ell - 1)}$
		\Else
		\State \Return $\texttt{EvalStream}(L', (\overline{b}, \ell - 1))$
		\EndIf
		\Else
		\State $L' \gets L \bigcup \set{(b, \ell)}$
		\State \Return $L'$
		\EndIf
	\end{algorithmic}
\end{algorithm}

\begin{enumerate}
	\item \textbf{Setup}: Same exact setup with TFHE in SSLE. Setup TFHE and distribute shares.
	      Each party also has some secret $s_i$ and commitment $\comm_i = \comm(s_i)$.
	\item \textbf{Publish Bid}: Each party samples $r_i \leftarrow U$ and publish $b_i = (\Enc(r_i), \Enc(\comm_i))$ each tagged with level $\log_2 n + 1$.
	\item \textbf{Evaluation}: This step is different than that in the simple sorting version (\cref{sec:simpleSort}).
	      Now, we can evaluate in a streaming fashion and even distribute evaluation (though we'll not go into details about the distributed implementation).
	      See \cref{alg:distrEval} for the algorithm. The idea is that we keep a list of bids and their levels;
	      when a new bid comes in, we greedily remove any bids from the list which we can. We can run \cref{alg:distrEval}
	      until there is only one bid left in the list or until all bids up to a level have been processed.
	      So, say that we have $k = 4$, we can run the algorithm until we are two levels away (in a tournament tree) from the final level.
	      More formally, when we have processed all bids up to level $\ell_{\mathrm{stop}} = \log_2 k + 2$ (and we only have bids at level $\log_2 k + 1$ in the list) we return the list and stop processing.
	      When there are $k$ elements in this list, all at level $\log_2 k + 1$, we are done.

		    \LS{I may have an off by one error here though I do not think so.}
	      
	\item \textbf{Decoding}: At least $t$ parties publish their threshold decryptions of the return from the evaluation
	\item \textbf{Proving}: For a party $i$ to prove that they are the selected leader for the $j$th slot, they prove that they know the secret to produce $\comm_i$
	      where $\comm_i$ is the decrypted commitment for the $i$th slot
\end{enumerate}

% \section{Oblivious PQ}
% \newcommand{\ct}{\texttt{ct}}
% https://eprint.iacr.org/2016/046.pdf
% and the functionality that it realizes is one that produces an encrypted random ordered list of the first k bla thingies
% We will start by describing the functionality of an MPC protocol for restricted computing environments which produces some random output list.

% TODO: ideal functionality
\subsection{Oblivious Priority Queue}
We will make extensive use of an oblivious priority queue which works over FHE.

\newcommand\compInd{\stackrel{\mathclap{\normalfont\mbox{c}}}{\approx}}
\newcommand{\PQInsert}{\texttt{PQInsert}}
\newcommand{\PQFront}{\texttt{PQFront}}
\newcommand{\PQExtractFront}{\texttt{PQExtractFront}}
\newcommand{\simulator}{\text{Sim}}
\newcommand{\simulatorPQ}{\text{Sim}_\text{PQ}}
\newcommand{\PQ}{\texttt{PQ}}
\newcommand{\hybrid}[1]{\mathrm{\textbf{Hyb}}_{#1}}
% \newcommand{\simulator}{\mathcal{S}}

\begin{definition}[Oblivious Priority Queue]
	An oblivious priority queue is a priority queue which can be ``run'' by a party oblivious to the actual
	messages of the priority queue. More formally, an oblivious priority queue has functionality
	\begin{itemize}
		\item $\PQInsert(\ct, \PQ) = \PQ'$ which takes in a cipher text $\ct$ and current priority queue and outputs $\PQ'$ where 
		the ciphertext is properly inserted into the priority queue according to the underlying plaintext of $\ct$.
		\item $\PQFront(\PQ) = \ct$ which takes as input a priority queue and outputs the ciphertext $\ct$ who's plaintext is the minimum element in the queue.
		\item $\PQExtractFront(\PQ) = \ct, \PQ'$ which takes as input a priority queue and outputs the ciphertext with minimum plaintext as well as an updated queue without $\ct$.
	\end{itemize}
	We also require that for $N$ elements in $\PQ$, operation $Q \in \set{\PQInsert, \PQFront, \PQExtractFront}$,
	and input $\ct$ into $Q$ such that there is some simulator, $\simulatorPQ$ where
	% TODO: latex crypto package
	\begin{equation}
		\{\simulatorPQ(1^\lambda, N, Q, \PQ) \} \compInd	\{Q(\PQ, \ct) \}.
	\end{equation}
	In words, $\simulatorPQ$ can simulate a distribution of the output $\PQ$ independent of the input element.
	\LS{I am not quite sure about the above: definitions that I saw are for adaptive adversaries, but we are not adaptive.}
\end{definition}

\subsection{Multi SLE Algorithm}

For $n$ parties, we want to output a list of $k$ elements where each element is chosen randomly (without replacement) from a set of messages submitted by each party.
Let the function for each party denote $f_i$ and we will say that $f_i(\Enc(b_1), \Enc(b_2), ..., \Enc(b_n)) = y$ where $y \in \set{\bot, 1, ..., k}$
such that if $y=a$ then party $i$ is the leader for the $a$th round and if not $y = \bot$.

To build this functionality, we will use an oblivious streaming sampler. We can construct one as follows, closely following \cite{shi2020path} but modifying randomness sampling to use a PRF.
\begin{itemize}
	\item When an item, $\ct$, arrives as a TFHE cipher text, we will produce a random value, $\Enc(\alpha)$ by evaluating $\texttt{PRF}(k, \ct)$ in FHE.
	\item If the item is the $m$-th item and $m \leq k$, insert the item into the queue along with $\alpha$ as its label via $\PQInsert(\ct)$.
	\item If $m > k$, we will evaluate the PRF (in FHE) to get an encoded random coin which is 1 with probability $1/m$ and 0 otherwise.
	Then, run \cref{alg:ObIns} which will
	\begin{itemize}
		\item replace the smallest labeled item in the queue with the new item if the coin is 1.
		\item not replace the smallest item in the queue if the coin is 0.
	\end{itemize}
	\item At the end of the stream, we output all the items in the priority queue one by one using $\PQExtractFront$.
\end{itemize}
The simulation proof follows almost exactly as in \cite{shi2020path} but now we have to deal with the randomness from 
the PRF rather than public randomness. We can follow the proof directly but add one more hybrid where we replace the output of the PRF with a random oracle in the FHE cipher text space.
Then, we can use the PRF indistinguishably property to show that the hybrid is indistinguishable from the previous hybrid.\\

We can then implement multi secret leader election via the following algorithm:
\begin{itemize}
	\item Each party $i$ will submit a message $\Enc(b_i)$ to the oblivious streaming sampler for some secret $b_i$ which is a commitment to a secret known only by party $i$.
	\item After all parties submitted there messages, the oblivious streaming sampler will output a list of $k$ messages: $\Enc(b_{a_1}), \Enc(b_{a_2}), ..., \Enc(b_{a_k})$.
	\item Then, at least $t$ parties will submit decryption shares to decrypt $\Enc(b_{a_1}), ..., \Enc(b_{a_k})$.
	\item Each party will then build check if they one an election by seeing if their commitment is in the list of decrypted messages.
	A party can then prove that they won an election $a$ by showing that they know the opening to $b_a$.
\end{itemize}

\begin{algorithm}
	\caption{Oblivious Gated Insert}
	\label{alg:ObIns}
	\begin{algorithmic}
		\Require $\Enc(\text{coin}), \PQ, \ct_a$
		\Ensure $\PQ$'
		\State $\ct_b, \PQ' \gets \PQExtractFront(\PQ)$
		\State $\ct_{\text{new}} = \Enc(\text{coin}) \cdot \ct_a + (\Enc(\text{coin}) - 1) \cdot \ct_b$
		\State \Return $\PQInsert(ct_{\text{new}})$
	\end{algorithmic}
\end{algorithm}


\subsection{Correctness}
For correctness, we want to show that for each of the $k$ elections, a unique leader is chosen for each round
and that the probability of being selected leader is uniform out of the potential leaders.

More formally, let $E_i$ be the event in which party $i$ is elected for some election and
$E_{a, i}$ be the event which party $i$ is elected for the $a$-th election. We want 
\begin{align}
	&\Pr\left[E_i\right] = \frac{k}{n},\\
	&\Pr\left[E_{a, i} \mid E_i\right] = \frac{1}{k},\\
	&\Pr\left[E_{b, i} \mid E_{a, i}\right] = 0.
\end{align}
where $b \neq a, b \in [k]$.
\LS{I think that here we can prove the uniqueness and fairness property of multi-leader election}
\begin{theorem}[Correctness]
We have correctness...	
\begin{proof}
	By definition... should be pretty easy to show.
\end{proof}
\end{theorem}
\subsection{Simulation Security}
We will show security in a semi-honest model via simulation. We will show that each user's view of the protocol can be simulated by a polynomial-time algorithm that only has access to the user's input.
Note that view of publishing a message is the same as the view of as each user publishing a random LWE ciphertext assuming the TFHE is secure.
\\Then, upon evaluation, we need to show that the oblivious queue can be simulated.
Note that every step in the oblivious queue algorithm which does not involve $b_i$ involves manipulating
TFHE samples under a secret key unknown to the simulator. Thus, we can simulate all steps not involving $b_i$.
Every step in the oblivious queue algorithm which involves $b_i$ and some $b_j$, for $i \neq j$ requires 
a homomorphic evaluation where one of the inputs in encrypted under a secret key unknown to the simulator.
Thus, by the security of TFHE, the output of any of these steps is indistinguishable from a using a $b_j$ encoding a different message.

\subsubsection*{Proof of Simulation Security}
More formally, $\forall i \in [n]$, we can create a simulator, $\simulator_i$, which takes as input 
the user's secret value $b_i$ and publishes random TFHE encrypted message, $\Enc(b_j)$ for all $j \neq i$, for all other parties such that
\begin{equation}
	\label{eq:simulator}
	\{\simulator_i(1^\lambda, s_i, \comm_i, r_i, f_i(\Enc(b_1), ..., \Enc(b_n))\} \compInd \{\texttt{view}^\pi_1(s_i, \comm_i, r_i, y)\} 
\end{equation}
with non-negligible probability.

To achieve \cref{eq:simulator}, the simulator simply follows a simple procedure
\begin{enumerate}
	\item The simulator randomly samples $b_j$ for all $j \neq i$
	\item The simulator uses $\simulatorPQ$ to simulate the oblivious streaming sampler for all inputs $\Enc(b_j)$ for $j \neq i$.
	For input $\Enc(b_i)$, the simulator honestly runs the oblivious streaming sampler.
	\item After all $n$ parties have submitted their inputs, the simulator runs the oblivious streaming sampler to get the outputs $\Enc(b_{a_1}),... \Enc(b_{a_k})$.
	\item If the output of the oblivious streaming sampler 
	\item If $y \neq \bot$, the simulator then ``decrypts'' $\Enc(b_{a_y})$ to get $b_i$ and also pretends to decrypt the rest of the $\Enc(b_{a_j})$'s by setting the decryption to a random value.
	It $y = \bot$, then the simulator will ``decrypt'' all $\Enc(b_{a_j})$'s to random values.
	\item The simulator simply outputs the view of the protocol with the ``decryptions'' of $\Enc(b_j)$'s.
\end{enumerate}

We will now show that the above indeed is a polynomial time simulator with overwhelming probability.
\begin{lemma}
	The above algorithm satisfies the computational indistinguishably requirment of \cref{eq:simulator}.
	\begin{proof}
		First we will show that $\mathcal{S}_i$ runs in polynomial time. There are only $k + 1$ possible outcomes for $y$,
		each with inverse polynomial probability. Thus, with overwhelming probability, the simulator can sample the $b_i$'s where the outcome is $y$.

		We now show simulatability via hybrids.
		Note that by the fairness property, the simulator has a non-negligible probability of being the true winner of the election.
		Thus, we will assume that the simulator is the true winner of election $y$ if $y \neq \bot$. If $y = \bot$, then we assume that $i$ did not win any election.
		Moreover, we also have that the encoded FHE elements (where party $i$ does not know the plaintext)
		are indistinguishable from encodings of random elements. We will now construct our hybrids:
		
		$\hybrid{1}$: Consider a hybrid where we replace the FHE encoded elements, $\Enc(b_j)$ for $j \neq i$ with random elements.
		By definition of IND-CPA security, this is indistinguishable from the original hybrid. We can then propogate the output of the oblivious priority queue
		to use these random elements.

		$\hybrid{2}$: Consider a hybrid where we replace the TFHE decoding of all elements which do not equal to $y$ with a random value.
		As the $b_j$ for $i \neq i$ are random, this hybrid is indistinguishable from the previous hybrid.

		$\hybrid{3}$: Consider a hybrid where we replace the the output of the priority queue with the output from simulating the priority queue with the random FHE encoded elements.
		
		Note that $\hybrid{3}$ is indistinguishable from the original hybrid and is the same as the output of the simulator.
	\end{proof}
\end{lemma}
\newcommand{\idealMSLE}{\mathcal{F}_{\text{MSLE}}}
\newcommand{\idealStreamSample}{\mathcal{F}_{\text{SAMPLE}}}
\newcommand{\hybrid}[1]{\mathrm{\textbf{Hyb}}_{#1}}
\newcommand{\eid}{{eid}}
\newcommand{\init}{\texttt{initialize}}
\newcommand{\register}{\texttt{register}}
\newcommand{\elect}{\texttt{elect}(\eid)}
\newcommand{\reveal}{\texttt{reveal}(\eid, \ell)}
\newcommand{\extractMin}{\texttt{ExtractMin}}
\newcommand{\getMin}{\texttt{GetMin}}
\newcommand{\ct}{\texttt{ct}}

\newcommand{\PQInsert}{\texttt{PQ.Insert}}
\newcommand{\PQExtract}{\texttt{PQ.ExtractFront}}
\newcommand{\PQFront}{\texttt{PQ.Front}}
\newcommand{\PQ}{\texttt{PQ}}

\newcommand{\EncPQInsert}{\texttt{EncPQ.Insert}}
\newcommand{\EncPQExtract}{\texttt{EncPQ.ExtractFront}}
\newcommand{\EncPQFront}{\texttt{EncPQ.Front}}


\newcommand{\EncPQ}{\texttt{EncPQ}}

\section{Data Independent Streaming Sampler}
\label{sec:streaming_sampler}
Our construction makes heavy use of a data independent streaming sampler which we will define below.
The streaming sampler relies on a data independent priority queue and in turn makes use of \LS{Cite Shi} protocol for an oblivious priority queue.

\subsection*{
	Encrypted Data Independent Queue
}
In this work, we will data independent queues as studied in \cite{toft2011secure, mitchell2014data, mazloom2023efficient}.
Data independent data structures are unique as their control flow and memory access do not depend on input data (\cite{mitchell2014data}).

They are especially useful as they allow for efficient computation within FHE as control flow is not dependent
on underlying ciphertexts. We use a data independent queue as outlined in \cite{mazloom2023efficient}
which allows for
\begin{itemize}
	\item $\PQInsert$: Inserts a tag and value, $(p, x)$ into $\PQ$ according to the tag's priority.
	\item $\PQExtract$: Removes and returns the $(p, y)$ with highest tag priority.
	\item $\PQFront$: Returns the $(p, y)$ with highest tag priority without removing the element.
\end{itemize}
Moreover, we note that the order is stable. I.e.\ the first inserted among equal tagged elements has a higher priority.

We will use the data independent queue within public key threshold FHE such that we now have an encrypted priority queue, $\EncPQ$
with functionality
\begin{itemize}
	\item $\EncPQInsert$: Inserts a tag and value, $\Enc(p, x)$ into $\PQ$ according to the tag's priority.
	\item $\EncPQExtract$: Removes and returns the $\Enc(p, y)$ with highest tag priority.
	\item $\EncPQFront$: Returns the $\Enc(p, y)$ with highest tag priority without removing the element.
\end{itemize}

We further require that for $\Enc(p, y) \gets \EncPQExtract$ (and $\EncPQFront$) and $x_1, ..., x_n$ drawn from a random distribution,
\begin{align}
	\label{eq:gameEncPQ}
	\Bigl\lvert \Pr[\advers\left(x_1, \dots x_n, \Enc(p_1, x_1), ..., \Enc(p_n, x_n), \Enc(p, y), i\right) = 1] \\
		- \Pr[\Dec(\Enc(p, y)) = (p_i, x_i)] \Bigr\rvert \leq \negl(\lambda) \notag
\end{align}
where $(p_i, x_i)$ is the $i$th submitted message to the priority queue.

\begin{lemma}
	Assuming the indistinguishablity of TFHE cipher texts, $\Enc(m_1), \Enc(m_2)$ where $m_1 \neq m_2$ and are known to the adversary, \cref{eq:gameEncPQ} holds.

	\begin{proof}
		Assume towards contradiction that there exists an adversary $\advers$ which can distinguish between a random 
		message to the priority queue and the output of $\EncPQExtract$ (resp. $\EncPQFront$).
		Then, we can construct an adversary $\advers'$ which can distinguish TFHE cipher text, $\ct_1 = \Enc(m_1), \ct_2 = \Enc(m_2)$, as follows:
		\begin{enumerate}
			\item $\advers'$ simulates an encrypted priority queue, $\EncPQ$.
			\item $\advers'$ inserts $x' = \Enc(\ct_1, r_1)$ into $\EncPQ$ and then $x = \Enc(\ct_2, r_2)$ into $\EncPQ$
			where $r_1, r_2$ are random.
			\item The adversary then calls $\Enc(p, y) \gets \EncPQExtract$ (resp. $\EncPQFront$) and uses $\advers$ to distinguish between $x$ and $x'$.
			\item If $\Enc(p, y) = x'$ output $m_1$, otherwise output $m_2$.
		\end{enumerate}
		Clearly, if the adversary can distinguish the priority queue outcomes, then using $\advers$, he can with non-negligible probability
		distinguish encryptions of $m_1$ and $m_2$ as, WLOG, $m_1 > m_2$ remains constant. An identical approach can then be taken
		to distinguish between encryptions of $m_1, ..., m_n$ except where the adversary inserts $m_1, ..., m_n$ into the queue and dequeues all items.
	\end{proof}
\end{lemma}


\subsection*{Encrypted Streaming Sampler}
A streaming sampler should output a random subset of size $k$ from a stream of $n$ elements with a random ordering.
We can build a streaming sampler from a priority queue as follows:


\begin{algorithm}
	\caption{Encrypted Gated Insert}
	\label{alg:EncIns}
	\begin{algorithmic}
		\Require $\Enc(\text{coin}), \EncPQ, \ct_a$
		\Ensure $\EncPQ$'
		\State $\ct_b \gets \EncPQExtract(\PQ)$
		\State $\ct_{\text{new}} = \Enc(\text{coin}) \cdot \ct_a + (\Enc(\text{coin}) - 1) \cdot \ct_b$
		\State $\EncPQInsert(ct_{\text{new}})$
		\State \Return $\EncPQ$
	\end{algorithmic}
\end{algorithm}

We will use a slightly modified version of a streaming sampler where randomness is submitted alongside the message.
We can then implement the algorithm as follows
\begin{itemize}
	\item $\texttt{Insert}(\ct_i, \Enc(e_i))$ where $\ct_i$ is a cipher text and $\Enc(e_i)$ is encrypted randomness
	\begin{itemize}
		\item If the item is the $m$-th item and $m \leq k$, insert the item into the queue along with $e_i$ as its tag via $\EncPQInsert(\Enc(e_i), \ct)$.
		\item If $m > k$, we will evaluate the PRF (in FHE) to get an encoded random coin which is 1 with probability $1/m$ and 0 otherwise.
		Then, run \cref{alg:EncIns} which will
		\begin{itemize}
			\item replace the smallest labeled item in the queue with the new item if the coin is 1.
			\item not replace the smallest item in the queue if the coin is 0.
		\end{itemize}
	\end{itemize}

	\item At the end of the stream, we output all the items in the priority queue, $\ct_{a_1} ... \ct_{a_k}$ in order
	 by repeatably calling $\EncPQExtract$.
	
\end{itemize}

We can formalize soundness as a game as follows, for all $\ell \in [k], i \in [n]$,
\begin{align}
	\Bigl\lvert 
		\Pr[\advers(\ct_1, \dots, \ct_n, e_1, \dots, e_n, \ct_{a_1}, \dots, \ct_{a_k}, \ell, i) = 1] \\
		- \Pr[\Dec(\ct_{a_\ell}) = \Dec(\ct_i)]
	\Bigr\rvert \leq \negl(\lambda) \notag
\end{align}
In words, the adversary should not be able to guess with any advantage which of the $\ell$-th 
outputted cipher text is associated with which inputted cipher text.

\begin{lemma}
	Our construction of a streaming sampler is sound.
	\begin{proof}
		% TODO: this should not be too bad
	\end{proof}
\end{lemma}

\section{MSLE Protocol}
\label{sec:msle_protocol}
We use a similar notion of ideal functionality for a multi-secret leader election from the ideal
functionality of single secret leader election of \LS{CITE}.

\begin{figure}[ht]
	\centering
	\fbox{
		\begin{minipage}{1\textwidth}
			\textbf{The MSLE functionality} $\idealMSLE$:
			Initialize $E, R \gets \emptyset, \gets 0$. Fix some $k \in \N$ to denote the number of rounds. Upon receiving,
			\begin{itemize}
				\item $\register$ from party $P_i$, set $R \gets R \cup \{(i, n\}$, broadcast $(\texttt{register}, i)$ to all parties and set $n \gets n + 1$
				\item $\elect$ $k$ leaders from all honest parties. If $|R| \geq k$ and $\eid$ was not requested before,
				      randomly sample $W^{\eid} \subseteq R$ where $|W^{\eid}| = k$.
				      Then, assign a random ordering to $W^\eid$ to get ordered set $E^\eid$.
				      Next, send $(\texttt{outcome}, \eid, a)$ to $P_j$ for all $E^\eid_a = (j, \cdot)$
				      and $(\texttt{outcome}, \eid, \bot)$ to $P_i$ if $(i, \cdot) \notin E^\eid$. Store $E \gets E \bigcup \{E^\eid\}$.
				\item $\reveal$ from $P_i$: for $E_\eid \in E$, if $i = E^\eid_a$, broadcast $(\texttt{result}, \eid, \ell, i)$.
				      Otherwise, broadcast $(\texttt{rejected}, \eid, \ell, i)$.
			\end{itemize}
		\end{minipage}
	}
	\caption{Description of the MSLE functionality, heavily based on the description of SSLE in \LS{CITE Dario and CFG21}}.
	\label{fig:my_label}
\end{figure}

\begin{figure}
	\centering
	\fbox{
		\begin{minipage}{1\textwidth}
			\textbf{The MSLE Protocol} $\pi_{\text{MSLE}}$:
			Initialize $n = 0$. Fix some $k \in \N$ to denote the number of rounds.
			Also, sample ;
			this will be the key FHE to the PRF.

			\begin{itemize}
				\item $\init$
				\begin{itemize}
					\item Set $n = 0$
					\item Sample some secret $s_{\texttt{TFHE}}$ and publicly publicly publish $\Enc(s_{\texttt{TFHE}})$.
					This will be the key to the PRF
					\item Initialize an empty set $C$, containing all party's registered commitments.
				\end{itemize}

				\item $\register(\Enc(c_i))$ from $P_i$ where $c_i = \comm(s_i, r_i)$ for secret $s_i$ and secret randomness $r_i$,
				\begin{itemize}
					\item If party $P$ does not already have a TFHE share, create a TFHE share for $P_i$.
					\item Add $\Enc(c_i)$ to $C$.
					\item $n \gets n + 1$
				\end{itemize}
				\item $\elect$ $k$  
					\begin{itemize}
						\item An encrypted streaming sampler will be publicly initialized
						\item Public randomness, $r$ will be sampled to and used in the encrypted PRF to get random $\Enc(e_1), ..., \Enc(e_n)= \Enc(\texttt{PRF}(r, s_\texttt{TFHE}))$
						\item All $c_i \in C$ will be fed to the encrypted streaming sampler along with randomness $\Enc(e_i)$ 
						\item The encrypted streaming sampler will output a list of $k$ messages: $\Enc(c_{a_1}), \Enc(c_{a_2}), ..., \Enc(c_{a_k})$.
						\item Then, at least $t$ parties will submit decryption shares to get $c_{a_1}, ..., c_{a_k}$.
						\item Each party will then check if they won an election by seeing if their commitment is in the list of decrypted messages.
					\end{itemize}
				\item $\texttt{reveal}(\eid, \ell, \Enc(c'_i))$ from $P_i$
					\begin{itemize}
						\item $P_i$ submits a zero knowledge proof that they know the opening to $c_{a_\ell}$. If this proof verifies,
						remove $\Enc(c_i)$ from $C$ and add $\Enc(c'_i)$. Then send out $(\texttt{result}, \eid, \ell, i)$ to all parties.
						Otherwise, send out $(\texttt{rejected}, \eid, \ell, i)$ to all parties.
					\end{itemize}
			\end{itemize}
		\end{minipage}
	}
	\caption{Description of the MSLE protocol}
	\label{fig:protocolMSLE}
\end{figure}

\subsection{Simulation Security}
To show simulation security, we will prove that, given a party's input and output in the ideal model,
a simulator can simulate the distribution of the view in the protocol for the party. Note that because the
protocol is deterministic, it suffices to prove simulation for only the party's output.

More formally we will show that for party $i$,
\begin{equation}
	\label{eq:simSecReg}
	\simulator_i\left(s_i, r_i, \idealMSLE.\register_i\right) \compInd \texttt{view}_{\register_i}((s_0, r_0), ..., (s_n, r_n)),
\end{equation}
\begin{equation}
	\label{eq:simSecElect}
	\simulator_i\left(s_i, b_i, r_i, \idealMSLE.\elect_i\right) \compInd \texttt{view}_{\elect_i}((s_0, b_0), ..., (s_n, b_n)),
\end{equation}
and
\begin{equation}
	\label{eq:simSecReveal}
	\simulator_i\left(b_{a, 1}, \dots, b_{a, k}, s_i, r_i, \idealMSLE.\reveal_i\right) \compInd \texttt{view}_{\reveal_i}(),
\end{equation}

\begin{lemma}
	We will first show that \cref{eq:simSecReg} is simulation secure.
	\begin{proof}

	\end{proof}
\end{lemma}


\begin{lemma}
	We will now show that \cref{eq:simSecElect} is simulation secure.
	\begin{proof}
		The view of each party $i$ for $\elect$ can be expressed as
		\begin{align*}
			(r_i,  s_i, \Enc(b_{1}), \dots \Enc(b_n), \Enc(r'_1), \dots, \Enc(r'_n), \\
				\Enc(b_{a_1}), \dots, \Enc(b_{a_k}), \sigma_1, \dots \sigma_t, b_1, \dots, b_n)
		\end{align*}
		where $r'_i$ is the randomness from the streaming sampler and $\sigma_1, \dots \sigma_t$ are the decryption shares.
		Then, we have the simulator proceed in the following manner:
		\begin{enumerate}
			\item The simulator sets a random, local tape
			% TODO: question, are these c_js part of the protocol/ commitment or is it valid to do these samples?
			% (I.e. they are not messages)...
			\item The simulator samples $c_j$ for all $j \neq i$ where $c_j$ is a commitment to a random value
			\item The simulator samples some randomness $r$ and computes $\Enc(e_1), \dots, \Enc(e_n) = \Enc(\texttt{PRF}(r, s_\texttt{TFHE}))$
			\item The simulator creates an encrypted priority queue $\EncPQ$ and simulates the encrypted streaming sampler for all inputs $\Enc(c_j)$ for $j \in [i]$.
			\item The simulator runs the encrypted streaming sampler to get the outputs $\Enc(c_{a_1}),... \Enc(c_{a_k})$.
			\item The simulator then chooses a random, ordered subset $S \subseteq [n]$ where $|S| = k$. 
			If there is some $w$ such that $S_w = i$, then set $y = w$. Otherwise, set $y = \bot$.
			\item The simulator then sets the decryptions of $\Enc(c_{a_\ell})$ for $\ell \neq y$ to $c_{S_\ell}$.
			The simulator also creates shares $\sigma_j$ such that $\Enc(c_{a_\ell})$ decrypts to $c_{S_\ell}$.
			\item The simulator then outputs $y$
			
		\end{enumerate}
		We now use a sequence of hybrids to show that the view of the real protocol is indistinguishable from
		that of the simulated one
		% \begin{itemize}
		% 	\item $\hybrid{0}$: The real protocol
		% 	\item $\hybrid{1}$: As $\hybrid{0}$ but $c_{a_\ell}$ are replaced with $ci'_j$, commitments to random values for $j \neq i$. And 
		% 	$\Enc(b_j)$ is replaced with $\Enc(b'_j)$.\\
		% 	% TODO: triple equal sign?
		% 	$\hybrid{1} \compInd \hybrid{0}$ by the hiding property of commitments.

		% 	\item $\hybrid{2}$: As $\hybrid{1}$ but $\Enc(b_{a_1}),... \Enc(b_{a_k})$ are replaced with
		% 	$\Enc(b'_{a_1}),... \Enc(b'_{a_k})$, the output of sampling the encrypted streaming sampler with $\Enc(b'_j)$.\\
		% 	$\hybrid{2} = \hybrid{1}$ as we are running the protocol honestly %TODO:?

		% 	\item $\hybrid{3}$: As $\hybrid{2}$ but, for all $j \neq y$, replace
		% 	$\Dec(\Enc(b_{a_1})) = b_{S_j}$ by updating $\sigma_1, ... \sigma_t$ where $S \subseteq [n]$,
		% 	$|S| = k$, and $S_y = i$ if $y \neq \bot$.\\
		% 	$\hybrid{3} \compInd \hybrid{2}$
		% 	\item $\hybrid{4}$: If $y \neq \bot$ update the threshold decryption such that $\Dec(\Enc(b_{a_y})) = b_i$. 
		% 	$\hybrid{4} = \hybrid{3}$
		% 	% TODO: better define threshold decryption as separate thing
		% 	% TODO: remove separate ticket registration
		% 	\item $\hybrid{5}$: The simulated protocol
		% \end{itemize}
		% TODO: can I have proof be within bullet points
			% Question: can a simulator have access to information from prior calls? 
			% Ohhhhhhhhhhhhhhhh... I see output $y$ has to follow distribution as well.


	\end{proof}
\end{lemma}

\begin{lemma}
	We will now show that \cref{eq:simSecReveal} is simulation secure.
	\begin{proof}

	\end{proof}
\end{lemma}



\bibliographystyle{alpha}
\bibliography{bib/ref}

\end{document}

