\documentclass[11pt]{article}
\usepackage{Custom}
% \usepackage{algorithmic}
\usepackage{CSTheoryToolkitCMUStyle}
\usepackage{cleveref}

\newcommand{\myname}{Authors here}

%%%%% Section-renaming code by egreg
\makeatletter
% we use \prefix@<level> only if it is defined
\renewcommand{\@seccntformat}[1]{%
  \ifcsname prefix@#1\endcsname
    \csname prefix@#1\endcsname
  \else
    \csname the#1\endcsname\quad
  \fi}
% Now we define our homework section prefixes
\makeatother
%%%%%

\begin{document}

\title{Multiple Secret Leaders}

\author{\myname}

\date{\today}
\maketitle

\newcommand{\npar}{n}
\newcommand{\kld}{k}
\newcommand{\comm}{\texttt{comm}}
\newcommand{\Enc}{\texttt{Enc}_{\mathrm{TFHE}}}
\newcommand{\Dec}{\texttt{Dec}_{\mathrm{TFHE}}}
% \begin{abstract}
% \end{abstract}
\section{Some Notation or Smthng}
\begin{enumerate}
	\item We will have $\npar$ parties
	\item We will have $\kld$ leaders elected
	\item We will have a ``bid'' published by a user $i \in [\npar]$ be denoted as $b_i$
	\item We will denote a commitment as $\comm_i$
	\item We will denote some generic CRHF as $h$
	\item We will say $\Enc$ and $\Dec$ for TFHE encoding and decoding respectively
\end{enumerate}
\section{Sketching shtuff out}
Say we want to elect $\kld$ leaders (maybe with or without repetition) secretly out of $\npar$
parties, we can run SSLE $\kld$ times but that'd be painful. Let's not do that.

\subsection{Simple sorting}
\label{sec:simpleSort}
A simple approach with runtime (assuming $\kld \leq \npar$) $O(\kld + \npar \log \npar)$
is to use sorting. At a high level, the idea is to somehow randomly sort the $\npar$ parties 
then elect the top $\kld$ parties in the sort.
To do this we can use TFHE in a very similar way to the SSLE paper.

\begin{figure}
	\begin{center}
		\fbox{\parbox{12cm}{
				\begin{algorithmic}
					\State $\alpha \gets \texttt{ROM}(b_1, b_2, ...)$
					\For{$i \in [n]$}
					\State $b'_i \gets (\Enc(r_i) - \alpha, \Enc(\comm_i))$
					\EndFor
					\State $\texttt{sorted} = [b'_{j_1}, b'_{j_2}, ...] \gets \texttt{Sort}_{\mathrm{TFHE}}([b'_1, b'_2, ...])$
					\State \Return first $k$ elements of $\texttt{sorted}$
				\end{algorithmic}}}
	\end{center}
	\caption{Simple Sorting Evaluation}
	\label{alg:simpleSortEval}
\end{figure}


\begin{enumerate}
	\item \textbf{Setup}: Same exact setup with TFHE in SSLE. Setup TFHE and distribute shares.
	      Each party also has some secret $s_i$ and commitment $\comm_i = \comm(s_i)$.
	\item \textbf{Publish Bid}: Each party samples $r_i \leftarrow U$ and publish $b_i = (\Enc(r_i), \Enc(\comm_i))$.
	\item \textbf{Evaluation}: This is done by one party whom may even be an external party. See \cref{alg:simpleSortEval} for details
	\item \textbf{Decoding}: At least $t$ parties publish their threshold decryptions of the return from the evaluation
	\item \textbf{Proving}: For a party $i$ to prove that they are the selected leader for the $j$th slot, they prove that they know the secret to produce $\comm_i$
	      where $\comm_i$ is the decrypted commitment for the $i$th slot
\end{enumerate}

\subsection{Distributed/ Streaming Simple Sorting}
Assume that $n, k$ are a power of 2. We will keep a lot of the same from sorting
but now look at selection as a sort of tournament. As a note, security is slightly reduced here as
a participating party can know that they did not win before the final outcome of the election. I do not know if this matters.

For completeness, we will write out each step again

\algrenewcommand\algorithmicrequire{\textbf{Input:}}
\algrenewcommand\algorithmicensure{\textbf{Output:}}

\begin{algorithm}
	\caption{$\texttt{PlayGame}$. Homomorphically plays a game to decide which incoming bid ``wins''}
	\label{alg:playGame}
	\begin{algorithmic}
		\Require Point $\alpha \in \F_q$, \\two bids, $b_1 = (\Enc(r_1), \Enc(\comm_1)), (\Enc(r_2), \Enc(\comm_2))$
		\Ensure Winning bid. The winning bid should be indistinguishable from a random bid to the non-players
    \State $\texttt{closer} \gets \Enc(r_1 - \alpha > r_2 - \alpha)$
    \State \Return $\texttt{closer} \cdot b_1 + (1 - \texttt{closer}) \cdot b_2$
	\end{algorithmic}
\end{algorithm}



\begin{algorithm}
	\caption{$\texttt{EvalStream}$. Evaluates a stream of bids as they come in.}
	\label{alg:distrEval}
	\begin{algorithmic}
		% TODO: how to name?
		% Name evalStream
		\Require $L$, set of $(\texttt{bid}, \texttt{level})$ pairs. New bid and level, $(b, \ell)$. Also, stop level $\ell_{\mathrm{stop}}$
		\Ensure New list $L$ with remaining evaluations
		
		\If {$\text{there is some for some }b', (b', \ell) \in L$}
		  \State $\alpha \gets \texttt{ROM}(b, b', \ell)$
		  \State $\overline{b} \gets \texttt{PlayGame}(b, b')$
		  \State $L' \gets L \setminus \set{(b', \ell)}$
      \If $\ell_{\mathrm{stop}} = \ell - 1$
        \State \Return $L \bigcup \set{(\overline{b}, \ell - 1)}$
      \Else
		    \State \Return $\texttt{EvalStream}(L', (\overline{b}, \ell - 1))$
    \EndIf
		\Else
		\State $L' \gets L \bigcup \set{(b, \ell)}$
		\State \Return $L'$
		\EndIf
	\end{algorithmic}
\end{algorithm}

\begin{enumerate}
	\item \textbf{Setup}: Same exact setup with TFHE in SSLE. Setup TFHE and distribute shares.
	      Each party also has some secret $s_i$ and commitment $\comm_i = \comm(s_i)$.
	\item \textbf{Publish Bid}: Each party samples $r_i \leftarrow U$ and publish $b_i = (\Enc(r_i), \Enc(\comm_i))$ each tagged with level $\log_2 n + 1$.
	\item \textbf{Evaluation}: This step is different than that in the simple sorting version (\cref{sec:simpleSort}).
	      Now, we can evaluate in a streaming fashion and even distribute evaluation (though we'll not go into details about the distributed implementation).
        See \cref{alg:distrEval} for the algorithm. The idea is that we keep a list of bids and their levels;
        when a new bid comes in, we greedily remove any bids from the list which we can. We can run \cref{alg:distrEval} 
        until there is only one bid left in the list or until all bids up to a level have been processed.
        So, say that we have $k = 4$, we can run the algorithm until we are two levels away (in a tournament tree) from the final level.
        More formally, when we have processed all bids up to level $\ell_{\mathrm{stop}} = \log_2 k + 2$ (and we only have bids at level $\log_2 k + 1$ in the list) we return the list and stop processing.
        When there are $k$ elements in this list, all at level $\log_2 k + 1$, we are done.
        \begin{remark}
          I may have an off by one error here.
        \end{remark}
	      
	\item \textbf{Decoding}: At least $t$ parties publish their threshold decryptions of the return from the evaluation
	\item \textbf{Proving}: For a party $i$ to prove that they are the selected leader for the $j$th slot, they prove that they know the secret to produce $\comm_i$
	      where $\comm_i$ is the decrypted commitment for the $i$th slot
\end{enumerate}

\bibliographystyle{alpha}
\bibliography{bib/ref}

\end{document}

